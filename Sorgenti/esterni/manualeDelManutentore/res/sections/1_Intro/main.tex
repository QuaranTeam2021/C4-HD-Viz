\section{Introduzione}

% inserire tutte il codice della sezione in questo file ed eliminare cartella subsections
% OPPURE inserire più file nella sottocartella subsections e includere i file 
% come mostrato sotto:

% es: per includere 2 files chiamati 
% nomefile1.tex, nomefile2.tex
% \subimport{subsections/}{nomefile1.tex}
% \subimport{subsections/}{nomefile2.tex}

% oppure
% \subsectionInFile{nomefile1.tex}
% \subsectionInFile{nomefile2.tex}

\subsection{Scopo del documento}
Questo documento ha lo scopo di fornire una guida agli sviluppatori interessati ad estendere o mantenere l'applicazione \textit{HD Viz}. Il documento contiene l'analisi dell'architettura realizzata e delle scelte progettuali adottate.

\subsection{Scopo del prodotto}
\textit{HD Viz} è un'applicazione web avente scopo di fornire uno strumento per la visualizzazione di dati con molte dimensioni a supporto della fase esplorativa dell'analisi dei dati.
\textit{HD Viz} è in grado di rappresentare dati che possono avere almeno 15 dimensioni e fornisce 4 diversi tipi di visualizzazione a tale scopo.

\subsection{Glossario}

Il documento contiene termini che possono presentare significati ambigui. Viene quindi fornito un glossario individuabile nell'appendice §A, all'interno del documento, contenente tutti i termini definiti ambigui e la loro spiegazione. Nel documento vengono identificati con una G a pedice. 

\subsection{Riferimenti}
\subsubsection{Riferimenti normativi}
\begin{itemize}
	\item \textbf{\glo{Capitolato} d'appalto C4 - \textit{HD Viz}:} \\
	\url{https://www.math.unipd.it/~tullio/IS-1/2020/Progetto/C4.pdf}.
\end{itemize}

\subsubsection{Riferimenti informativi}
\begin{itemize}
	\item \textbf{Slide del corso di Ingegneria del Software - Diagrammi delle classi} \\
	\url{https://www.math.unipd.it/~rcardin/swea/2021/Diagrammi%20delle%20Classi_4x4.pdf};	
	\item \textbf{Slide del corso di Ingegneria del Software - Diagrammi dei package} \\
	\url{https://www.math.unipd.it/~rcardin/swea/2021/Diagrammi%20dei%20Package_4x4.pdf};	
	\item \textbf{Slide del corso di Ingegneria del Software - Diagrammi di attività} \\
	\url{https://www.math.unipd.it/~rcardin/swea/2021/Diagrammi%20di%20Attivit%c3%a0_4x4.pdf}; 
	\item \textbf{Slide del corso di Ingegneria del Software - Principi di programmazione SOLID} \\
	\url{https://www.math.unipd.it/~rcardin/swea/2021/SOLID%20Principles%20of%20Object-Oriented%20Design_4x4.pdf};
	\item \textbf{Slide del corso di Ingegneria del Software - Design pattern comportamentali} \\
	\url{https://www.math.unipd.it/~rcardin/swea/2021/Design%20Pattern%20Comportamentali_4x4.pdf};

\item \textbf{Libri di Testo}: 
    \begin{itemize}
    \item Software Engineering (10th edition) - Ian Sommerville - Pearson Education - Global Edition\\
	    Sezioni:
	    \begin{itemize}
			\item §6 - Architectural Design
		\end{itemize}
    \item UML Distilled (3rd edition) - Martin Fowler - Addison Wesley\\
	    Sezioni:
	    \begin{itemize}
			\item §3 - Class Diagrams: The Essentials;
			\item §4 - Sequence Diagrams;
			\item §5 - Class Diagrams: Advanced Concepts;
			\item §7 - Package Diagrams.
		\end{itemize}
	\end{itemize}
\end{itemize}