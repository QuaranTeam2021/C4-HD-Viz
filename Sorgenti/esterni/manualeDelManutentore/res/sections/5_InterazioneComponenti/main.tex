\section{Interazioni fra le componenti}

\subsection{Caricamento file da locale}
\begin{figure}[H]
	\begin{center}
		\includegraphics[width=1.0\textwidth]{img/loadFile.png} \\
		\caption{Diagramma di sequenza per il caricamento di un file da locale}
	\end{center}
\end{figure}

Al verificarsi dell'evento \textit{onChangeInsert} la componente \textit{BuildGraph} della \glo{UI} invoca il metodo \textit{parse()} del controller \textit{LocalLoaderController} del quale possiede un'istanza. \\
Il file caricato viene controllato in dimensione, subisce il parsing e viene memorizzato all'interno di un array. Le features individuate nei dati vengono salvare in una struttura dati \textit{mappa <String, String>} per memorizzare la tipologia di ogni dimensione. \\
Successivamente il controller invoca \textit{loadData()} della componente \textit{Store}. Lo \textit{Store} notifica tutti i sui osservatori grazie all'implementazione del pattern \glo{observer} realizzata con la libreria \glo{MobX}.

\subsection{Creazione di un grafico con riduzione dimensionale}
\begin{figure}[H]
	\begin{center}
		\includegraphics[width=1.0\textwidth]{img/createGraph.png} \\
		\caption{Diagramma di sequenza per la creazione di un grafico per il quale è richiesta riduzione dimensionale}
	\end{center}
\end{figure}
Al verificarsi dell'evento \textit{onClickConfirm}, la componente \textit{BuildGraph} della \glo{UI} invoca il metodo \textit{buildGraph()} del controller \textit{TsneController} del quale possiede un'istanza. In questo caso i controller derivano tutti da un'unica classe astratta \textit{ReductionController}, e sfruttano così il polimorfismo per l'invocazione dei metodi. \\
Il controller crea l'istanza \textit{TsneParameters} che viene passata come parametro per calcolare l'algoritmo. Successivamente viene chiamato \textit{calculateReduction()} della classe \textit{Store} per ottenere i dati ridotti da inserire nel grafico. Viene creata l'istanza per contenere un grafico \textit{StandardGraph()} e successivamente si passa questa istanza come parametro del metodo \textit{addGraph()} della classe \textit{Store}.\\
Il procedimento rimane invariato per la creazione di un qualunque altro grafico o algoritmo di riduzione.
La componente \textit{Visualization} è un observer dello \textit{Store}, e verrà quindi notificata all'inserimento di un nuovo grafico.

%\subsection{Modifica dataset visualizzato Scatter plot Matrix}
%\begin{figure}[H]
%	\begin{center}
%		\includegraphics[width=1.0\textwidth]{img/modScpMatrix.pdf} \\
%		\caption{Diagramma di sequenza per la modifica di un dataset dovuta ad un cambio di algoritmo}
%	\end{center}
%\end{figure}