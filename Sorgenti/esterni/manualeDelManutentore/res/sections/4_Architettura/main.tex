\section{Architettura}
La base dell'architettura per l'applicazione web \textit{HD Viz} è rappresentata da \glo{MobX}. Il vantaggio offerto da \glo{MobX} è la gestione automatica dello stato con un set minimale di istruzioni da implementare. \glo{MobX} e React, che è stato utilizzato per lo sviluppo della \glo{UI}, sono comunemente usati insieme e per questo \glo{MobX} supporta una specifica integrazione con React che mette a disposizione un framework minimale per l'implementazione dell'\textit{\glo{observer} pattern}. \newline
Le componenti principali sono le seguenti:
\begin{itemize}
	\item \textbf{Domain store}: i domain stores sono degli archivi che contengono i dati dell'applicazione. Nel nostro caso lo \textit{Store} principale contiene dati che vengono caricati da database o da file locale. La componente \textit{Store} è poi composta da altri oggetti di dominio al suo interno, come l'array di istanze \textit{Graphs} che archivia lo stato di ogni diverso grafico richiesto sulla \glo{UI}. Lo \textit{Store} sarà quindi la componente di architettura che verrà resa \textit{observable} (\textit{subject}) da \glo{MobX}; 
	
	\item \textbf{Components}: queste sono le componenti React che compongono la \glo{UI}. Sono organizzate secondo una struttura gerarchica, che rappresenta in quale punto vengono inserite nell'interfaccia. Il livello immediatamente sottostante alla componente radice, \textit{App}, permette inoltre di individuare le diverse funzioni che l'applicazione fornisce (esplorazione di dati, gestione dei dataset salvati e manuale utente). La parte di esplorazione dei dati, funzionalità principale dell'applicazione, contiene il maggior numero di sotto-componenti.
	È proprio in queste che avviene l'integrazione tra le due librerie. Con il modulo \textit{observer} è possibile inglobare specifiche componenti che necessitano di osservare lo stato dello \textit{Store}. Non è quindi necessario rendere observer l'intera vista, ma solo le componenti necessarie;
	
	\item \textbf{Controllers}: diversamente da quanto previsto dall'utilizzo standard di \glo{MobX} con React abbiamo introdotto delle componenti controller che servono per trasformare gli input utente che arrivano dalle componenti React in dati che possano essere archiviati nello \textit{Store}. Le componenti controller sono state divise per ruolo in modo da renderle più semplici e minimali senza includere funzionalità superflue. 
\end{itemize}
Per rendere il tutto funzionante si fa utilizzo dei \textit{React context}. È possibile vedere l'organizzazione delle componenti React come una struttura ad albero. Un contesto è visibile da tutte le componenti dell'applicazione una volta che viene passato grazie ad un \textit{Provider} alla componente radice, che la rende disponibile a tutte le componenti discendenti da quest'ultima. Per utilizzare un contesto, che rimane quindi disponibile per tutta l'esecuzione, si utilizza \textit{useContext()} offerta da React. \\
React mette a disposizione anche l'hook \textit{useEffect} che può essere utilizzato per impostare dei \textit{side effect} che devono verificarsi e che sono legati al ciclo di vita della componente React. L'utilizzo di \textit{useEffect} richiede la specifica delle dipendenze, che possono essere lo \textit{store context} o un suo particolare attributo, in modo tale che la componente reagisca al loro cambiamento di stato. La possibilità di definire come dipendenze gli attributi è fondamentale perché permette alle componenti di ri-renderizzarsi solo in seguito a specifici cambiamenti. \\
In sintesi nell'architettura descritta è stato implementato un \textit{\glo{observer} pattern} nel quale lo \textit{Store} è un \textit{subject} e diverse componenti di React sono \textit{observers} che reagiscono al cambiamento di stato dello \textit{Store}. \newline
\subsectionInFile{Architettura.tex}