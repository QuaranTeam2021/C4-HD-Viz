\section{Server}
La parte server (opzionale per il funzionamento dell'applicazione) gestisce le query verso il database PostgreSQL ed il caricamento o la rimozione di dataset da esso, sfruttando la libreria node-postgres. Il framework Express.js gestisce le richieste \glo{HTTP} inviate dalla web app.

\subsection{Gestione delle richieste}

\begin{figure}[H]
	\begin{center}
		\includegraphics[width=1.0\textwidth]{images/server1.jpg} \\
		\caption{Schema di interazione con la parte server}
	\end{center}
\end{figure}

La gestione di una richiesta \glo{HTTP} inviata dalla web app comprende i seguenti passaggi:
\begin{enumerate}
	\item Le routes definite indirizzano le richieste della web app chiamando un'opportuna funzione \glo{middleware};
	\item La funzione \glo{middleware} opera dei controlli sui parametri passati dal client tramite la richiesta \glo{HTTP}. Nel caso di esito negativo provvede a ritornare un messaggio di errore significativo al client, in caso di esito positivo sfrutta la libreria node-postgres per effettuare le operazioni sul database collegato;
	\item In base all'esito della query o dell'operazione sul database effettuata, viene ritornato il contenuto richiesto dal client o un messaggio di errore significativo.
\end{enumerate}

\subsection{Routing}
Vengono utilizzati i seguenti percorsi di route:
\begin{itemize}
	\item \textbf{(GET)  /api/tableslist}: ritorna la lista dei dataset memorizzati nel database;
	\item \textbf{(GET)  /api/getcontent/:table}: ritorna il dataset memorizzato nella tabella \texttt{table};
	\item \textbf{(GET)  /api/getcolnames/:table}: ritorna i nomi delle dimensioni del dataset \texttt{table};
	\item \textbf{(GET)  /api/getselectedcol/:table}: ritorna le dimensioni selezionate del dataset \texttt{table};
	\item \textbf{(POST)  /api/upload/:table}: carica nel database il dataset \texttt{table} estraendolo da un file in formato \glo{CSV} o \glo{TSV}. Nel caso in cui non fosse presente il parametro \texttt{table}, il dataset viene salvato con il nome del file contenente i dati;
	\item \textbf{(DELETE)  /api/delete/:table}: elimina dal database il dataset \texttt{table};
	\item \textbf{(GET)  /gettoken}: ritorna un token (\glo{JWT}) alla componente client.
\end{itemize}

\subsection{Sicurezza}
La protezione dei percorsi di routes esposti viene garantita tramite l'uso di \glo{JWT}, attraverso i seguenti passaggi:
\begin{enumerate}
	\item La componente client dell'applicazione provvede ad effettuare una richiesta \glo{HTTP} alla route \texttt{/gettoken}, passando come parametri delle credenziali che ne consentano l'autenticazione;
	\item La componente \glo{middleware} richiamata dalla route effettua un controllo sulle credenziali, se il controllo ha esito positivo provvede a generare un nuovo token che viene ritornato al client;
	\item Il client salva il token che verrà quindi incluso nelle successive richieste \glo{HTTP};
	\item Una funzione \glo{middleware} effettua un controllo sulle route \texttt{/api/}, relativamente alla presenza e alla validità del token, in caso di esito negativo provvede a ritornare un messaggio di errore al client, invece in caso di esito positivo ridirige la richiesta alla route opportuna.
\end{enumerate}








