\section{Tecnologie utilizzate}
In questa sezione sono descritte brevemente le tecnologie utilizzate per lo sviluppo di \textit{HD Viz}.

\subsection{Linguaggi}
Segue un elenco dei linguaggi di programmazione utilizzati:
\begin{itemize}
	\item \textbf{\glo{JavaScript}}: questo è il linguaggio alla base dell'applicazione essendo il linguaggio di base per qualunque sito web.
\end{itemize}

\subsection{Framework e librerie}
\begin{itemize}
	\item \textbf{React.js}: libreria \glo{JavaScript} per la creazione di interfacce utente. Permette la renderizzazione delle componenti, rendendo semplice l'applicazione di modifiche ai grafici visualizzati;\\
	Link: \url{https://it.reactjs.org/}
	\item \textbf{Material-UI}: libreria di componenti utilizzata per facilitare la realizzazione di funzionalità lato front-end. Permette di personalizzare le componenti di React adottando come linea guida il linguaggio di design Material Design di Google;\\
	Link: \url{https://material-ui.com/}
	\item \textbf{D3.js}: questa libreria costituisce il cuore della visualizzazione dei grafici. D3.js è una libreria \glo{JavaScript} per la manipolazione di documenti basati sui dati (\textit{Data-driven documents}), combinando componenti di visualizzazione guidate dai dati con la manipolazione del \glo{DOM};\\
	Link: \url{https://d3js.org/}
	\item \textbf{DruidJS}: questa libreria è stata scelta perché racchiude al suo interno numerosi algoritmi di riduzione e ha delle buone prestazioni se confrontata con altre librerie che si pongono lo stesso obiettivo;\\
	Link: \url{https://renecutura.eu/pdfs/Druid.pdf} \\ 
	Link: \url{https://saehm.github.io/DruidJS/index.html}
	\item \textbf{\glo{MobX.js}}: utilizzato per l'implementazione del pattern \glo{observer}, rende semplice, trasparente e scalabile la gestione dello stato di una applicazione web. \glo{MobX} fornisce una specifica integrazione per React;\\
	Link: \url{https://mobx.js.org/about-this-documentation.html}
	\item \textbf{Node.js}: framework che permette di utilizzare \glo{JavaScript} come linguaggio per la creazione di un server con cui interfacciarsi col database. Uno dei vantaggi di Node.js è la sua caratteristica di essere guidato da eventi asincroni non bloccanti;	\\
	Link: \url{https://nodejs.org/it/}
	\item \textbf{Express}: framework per applicazioni web Node.js. Facilita la creazione di un server \glo{HTTP} e la definizione di \glo{middleware};\\
	Link: \url{https://expressjs.com/it/}
	\item \textbf{Node-postgres}: libreria utilizzata per interfacciarsi ad un database PostgreSQL.\\
	Link: \url{https://node-postgres.com/}
\end{itemize}

\subsection{Database}
\begin{itemize}
	\item \textbf{PostgreSQL}: database relazionale open-source.
	Link: \url{https://www.postgresql.org/}
\end{itemize}