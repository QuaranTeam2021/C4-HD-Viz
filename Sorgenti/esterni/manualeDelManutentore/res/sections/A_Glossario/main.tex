\section{Glossario}

% semplicemente:
% \subsection*{B}
%	\textbf{termine}: definizione
%	\textbf{termine}: definizione
% \subsection*{C}
% ...
\subsection*{C}
	\paragraph*{CSV}
	Acronimo di Comma-Separated Values, è un formato di file basato su file di testo utilizzato per l'importazione ed esportazione di una tabella di dati.
\subsection*{D}
	\paragraph*{DOM}
	Acronimo di Document Object Model, è una forma di rappresentazione dei documenti strutturati come modello orientato agli oggetti.
\subsection*{G}
	\paragraph*{GitHub} 
	Servizio di hosting per progetti software, che implementa lo strumento di controllo versione distribuito Git.
\subsection*{H}
	\paragraph*{HTTP} 
	Acronimo di HyperText Transfer Protocol. È un insieme di regole che un server deve seguire quando si tratta della trasmissione di file attraverso il World Wide Web.
\subsection*{J}
	\paragraph*{JavaScript} 
	Linguaggio di scripting orientato agli oggetti e agli eventi, utilizzato nella programmazione web sia lato client che lato server.
	\paragraph*{JSON} 
	Acronimo di JavaScript Object Notation. È un semplice formato per lo scambio di dati. Per le persone è facile da leggere e scrivere, mentre per le macchine risulta facile da generare e analizzarne la sintassi.
	\paragraph*{JWT} 
	Acronimo di JSON Web Token, è un sistema di cifratura e di contratto in formato \glo{JSON} per lo scambio di informazioni tra i vari servizi di un server. Vengono utilizzati nei web services e nelle web app per l'autenticazione dei client.
\subsection*{M}
	\paragraph*{Middleware} 
	Si intende il software che rende accessibile sul web risorse hardware o software che prima erano disponibili solo localmente o su reti non internet.
	\paragraph*{MobX} 
	Libreria di gestione dello stato. Proprio come React, che utilizza un \glo{DOM} virtuale per eseguire il rendering degli elementi dell'interfaccia utente nei browser, riducendo il numero di mutazioni \glo{DOM}, MobX fa la stessa cosa ma nello stato dell'applicazione.
\subsection*{O}
	\paragraph*{Observer} 
	Pattern che definisce una dipendenza di tipo $1..n$ fra oggetti riflettendo la 
	modifica di un oggetto su di quelli che sono a lui relazionati. Questo serve a mantenere consistenza nel sistema facendo in modo che in seguito ad una segnalazione avvengano degli aggiornamenti.
	\paragraph*{Open-closed principle}
	Principio della programmazione orientata agli oggetti che afferma che le entità (classi, moduli, funzioni) software dovrebbero essere aperte all'estensione, ma chiuse alle modifiche; in maniera tale che un'entità possa permettere che il suo comportamento sia modificato senza alterare il suo codice sorgente.
\subsection*{S}
	\paragraph*{Separation of Concerns} 
	Principio di progettazione che prevede la separazione di un programma in sezioni distinte.
	\paragraph*{Strategy} 
	Pattern che definisce una famiglia di algoritmi che possono essere fra di loro 
	interscambiabili. Viene utilizzato quando si hanno tipi diversi che differiscono tra di loro per il 
	comportamento ma non per l'interfaccia.
\subsection*{T}
	\paragraph*{TSV} 
	Formato di file il cui nome sta per "Valori separati da tabulazione", e questi TSV file vengono creati e utilizzati da molte applicazioni per fogli di calcolo.
\subsection*{U}
	\paragraph*{UI} 
	Acronimo di User Interface. L'interfaccia utente è l'interfaccia visuale con la quale l'uomo interagisce con un computer o una macchina per portare a termine un'attività.