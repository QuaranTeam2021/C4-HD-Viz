 \newcommand{\TitoloDoc}{Glossario}
\newcommand{\Verificatori}{\COF}
\newcommand{\Redattori}{\VEL}
\newcommand{\Approvatore}{\REL}
\newcommand{\Distribuzione}{\Committente{} \\& \Proponente{} \\& \Gruppo{}}
\newcommand{\Uso}{Esterno}
\newcommand{\Stato}{Approvato}
\newcommand{\DescrizioneDoc}{Questo documento racchiude tutti i termini ed acronimi utilizzati dal gruppo \textit{\Gruppo} all'interno della documentazione per lo sviluppo del progetto \textit{\NomeProgetto}.}
\newcommand{\pathimg}{../../immagini}
\newcommand{\VersioneDoc}{4.0.0}


% info generali 
\newcommand{\NomeProgetto}{HD Viz}

% fornitore
\newcommand{\Gruppo}{QuaranTeam}
\newcommand{\Mail}{quaranteam2021@gmail.com}

% committenti
\newcommand{\Committente}{\VT{}\\& \CR{}}
\newcommand{\VT}{Prof. Vardanega Tullio}
\newcommand{\CR}{Prof. Cardin Riccardo}

% proponenti
\newcommand{\Proponente}{Zucchetti S.p.A.}
\newcommand{\PG}{Piccoli Gregorio}

% QuaranTeam member
\newcommand{\CHF}{Chiarello Federico}
\newcommand{\COF}{Consalvo Federico}
\newcommand{\GIA}{Gibellato Alice}
\newcommand{\MAD}{Mason Damiano}
\newcommand{\REL}{Rech Elia}
\newcommand{\SIM}{Sinigaglia Matteo}
\newcommand{\VEL}{Veronese Luca}

% ruoli
\newcommand{\Responsabile}{Responsabile di Progetto}
\newcommand{\Amministratore}{Amministratore di Progetto}

% documenti
\newcommand{\SdF}{Studio di Fattibilità}
\newcommand{\SdFv}[1]{\textit{Studio di Fattibilità {#1}}}
\newcommand{\PdQ}{Piano di Qualifica}
\newcommand{\PdQv}[1]{\textit{Piano di Qualifica {#1}}}
\newcommand{\PdP}{Piano di Progetto}
\newcommand{\PdPv}[1]{\textit{Piano di Progetto {#1}}}
\newcommand{\NdP}{Norme di Progetto}
\newcommand{\NdPv}[1]{\textit{Norme di Progetto {#1}}}
\newcommand{\AdR}{Analisi dei Requisiti}
\newcommand{\AdRv}[1]{\textit{Analisi dei Requisiti {#1}}}
\newcommand{\Glossario}{Glossario}
\newcommand{\Glossariov}[1]{\textit{Glossario {#1}}}
\newcommand{\MM}{Manuale Manutentore}
\newcommand{\MMv}[1]{\textit{Manuale Manutentore {#1}}}
\newcommand{\MU}{Manuale Utente}
\newcommand{\MUv}[1]{\textit{Manuale Utente {#1}}}

% comandi generali
\newcommand{\glo}[1]{#1\textsubscript{\textit{G}}}
\newcommand{\qm}[1]{``#1''}

\newcommand{\defaultfooter}[1]{
	\rowcolor{white}
	\multicolumn{#1}{|c|}{\textit{La tabella continua a pagina seguente.}}\\
    \hline
    \endfoot
    \endlastfoot
}


\newcommand{\versionMU}{2.0.0}
\newcommand{\versionMM}{2.0.0}
\newcommand{\versionSdF}{1.0.0}
\newcommand{\versionPdQ}{4.0.0}
\newcommand{\versionPdP}{4.0.0}
\newcommand{\versionNdP}{3.0.0}
\newcommand{\versionAdR}{3.0.0}
\newcommand{\versionGlossario}{4.0.0}

\documentclass{../../Utility/stdDocument}

\begin{document}
% deve essere presente res/registro.tex per compilare
\glossarioConfig
	\section{A}
		\subsection{Amazon CloudWatch}
		Servizio di monitoraggio e osservazione costruito per raccogliere, accedere e mettere in relazione su un'unica \glo{piattaforma} i dati provenienti da tutte le risorse, le applicazioni e i servizi eseguiti su \glo{AWS} e sui \glo{server} locali.
		\subsection{Ambiente di sviluppo} 
		Un ambiente di sviluppo integrato (o IDE - Integrated Development Environment), è un \glo{software} che, in fase di programmazione, aiuta il programmatore nello sviluppo del codice sorgente fornendo strumenti e funzionalità di supporto allo sviluppo, soprattutto nel debugging e nel controllo della sintassi.
		\subsection{Analizzatore} 
		Macchina o programma che effettua un'analisi dei dati o delle strutture informative.
		\subsection{Angular}
		\glo{Framework} \glo{open-source} per lo sviluppo di applicazioni e pagine web.
		\subsection{Anti-Collision System} 
		Sistema automatico di sicurezza, utilizzato per ridurre la gravità di un incidente in caso di impatto imminente.
		\subsection{Apache Kafka} 
		\glo{Piattaforma} \glo{open-source} scritta in \glo{Java} e \glo{Scala} usata principalmente per tutte le applicazioni di elaborazioni di stream di dati in tempo reale.
		\subsection{Apache Tomcat}
		Vedere \glo{Tomcat}.
		\subsection{API} 
		Application Programming Interface, insieme di \glo{procedure} con i quali vengono realizzati e integrati \glo{software} applicativi.
		\subsection{API REST}
		Sistema di trasmissione dati che sfrutta HTTP ed i suoi protocolli.
		\subsection{Apprendimento approfondito}
		Vedere \glo{Deep Learning}.
		\subsection{Apprendimento automatico}
		Vedere \glo{Machine Learning}.
		\subsection{Attore} 
		Nel contesto del diagramma dei \glo{casi d'uso}, si riferisce ad un utente (persona fisica o sistema esterno) che svolge un ruolo nell'interazione con il sistema principale.
		\subsection{AWS} 
		Amazon Web Services, Inc. è un'azienda statunitense di proprietà del gruppo Amazon, che fornisce servizi di cloud computing.
		\subsection{AWS AppSync} 
		Servizio di \glo{AWS} che permette l'accesso offline a dati di app e sincronizza automaticamente gli aggiornamenti avvenuti quando il dispositivo non era connesso.
		\subsection{AWS GameLift} 
		Servizio che sfrutta l'infrastruttura globale di \glo{AWS} per gestire servizi di gioco.
		\subsection{AWS Lambda} 
		\glo{Piattaforma} di computing \glo{serverless} ad eventi, fornita da Amazon come parte di \glo{AWS}.
	
	\newpage
	\section{B}
		\subsection{Back-end} 
		Con il termine \glo{back-end} si indica l'interfaccia con la quale il gestore di un sito web dinamico ne gestisce i contenuti e le funzionalità. L'accesso al \glo{back-end} è riservato agli amministratori del sito web.
		\subsection{Bitbucket} 
		Strumento di gestione del codice \glo{Git}. Fornisce ai \glo{team} di lavoro uno spazio in cui pianificare i progetti, collaborare su codici, effettuare test e distribuirne i risultati.
		\subsection{Blockchain} 
		Struttura dati condivisa ed immutabile, gestita come un registro digitale di blocchi concatenati, ordinati cronologicamente e crittografati. Il suo principale impiego è la gestione di criptovaluta ed eventuali transazioni di quest'ultima tra soggetti della rete.
		\subsection{Bug} 
		Guasto che porta al malfunzionamento del \glo{software}, tipicamente dovuto ad un errore nella scrittura del codice sorgente di un programma.
		\subsection{Business logic} 
		L'espressione logica di business si riferisce all'algoritmica che gestisce lo scambio di informazioni tra l'interfaccia utente ed eventualmente una sorgente dati.
		
	\newpage
	\section{C}
		\subsection{Capitolato} 
		Un capitolato d'appalto è un insieme di clausole di un contratto, in cui una delle parti è il \glo{committente}. Nel capitolato viene presentato il prodotto insieme ai suoi vincoli e \glo{requisiti}.
		\subsection{Caso d'uso} 
		Insieme di scenari impiegati per descrivere situazioni nelle quali il sistema viene utilizzato per soddisfare uno o più bisogni dell'utente. Sono utilizzati all'interno degli Use Case Diagram di \glo{UML}. 
		\subsection{Chai} 
		\glo{Libreria} di asserzioni e testing per \glo{Node.js} e browser. 
		\subsection{Cloud} 
		Con il termine inglese cloud si indica un paradigma di erogazione di servizi offerti su richiesta da un \glo{fornitore} ad un cliente finale attraverso la rete Internet.
		\subsection{Cloud-based} 
		Un \glo{server} cloud-based è un \glo{server} in cui i dati del \glo{cloud} vengono memorizzati in un sistema di archiviazione professionale.
		\subsection{Code coverage} 
		Nel testing è la misura di quante linee/blocchi/archi del codice vengono eseguiti mentre i test automatici sono in esecuzione.
		\subsection{Codifica}
		Processo che permette di rappresentare delle informazioni mediante un certo codice. 
		\subsection{Commit} 
		All'interno di un sistema di \glo{versionamento}, è l'operazione che trasmette le ultime modifiche apportate al codice sorgente.
		\subsection{Committente} 
		Figura fisica o giuridica, che ordina ad altri l'esecuzione di un lavoro o di una prestazione, per suo conto. Ha il potere decisionale di spesa, relativo alla gestione del lavoro commissionato.
		\subsection{CommonJs}
		CommonJS è un progetto con l'obiettivo di stabilire convenzioni sull'ecosistema dei moduli per \glo{JavaScript} al di fuori del browser web.
		\subsection{Consistente}
		Sinonimo di solido, resistente.
		\subsection{ConvNet.js}
		\glo{Framework} \glo{JavaScript} per il \glo{Deep Learning}.
		\subsection{Covid-19} 
		Acronimo dell'inglese COronaVIrus Disease 19, è una malattia respiratoria causata dal virus SARS-CoV-2. Diffusasi dapprima in Cina e in seguito in tutto il resto del mondo, è stata identificata in Italia a partire da febbraio 2020 e ha portato il governo ad attuare delle misure di restrizione stravolgendo la quotidianità dei cittadini e coinvolgendo quasi tutti i settori lavorativi.
		\subsection{CSS} 
		Cascading Style Sheets, è un linguaggio usato per definire la presentazione (ovvero l'aspetto grafico) delle pagine web e lavora in combinazione con l'\glo{HTML}.
		\subsection{CSV} 
		Comma-Separated Values, è un formato di file basato su file di testo utilizzato per l'importazione ed esportazione di una tabella di dati.
	
	\newpage
	\section{D}
		\subsection{D3.js} 
		\glo{Libreria} \glo{JavaScript} per la manipolazione dei documenti basati su dati. Utilizzato per portare in vita i dati attraverso \glo{HTML}, \glo{SVG} e \glo{CSS}. 
		\subsection{Dashboard} 
		Interfaccia che permette di monitorare in tempo reale l'andamento dei report e delle metriche aziendali.
		\subsection{Data mining} 
		Insieme di tecniche e metodologie che hanno per oggetto l'estrazione di informazioni utili da grandi quantità di dati attraverso metodi automatici o semi-automatici.
		\subsection{Database} 
		Insieme organizzato di dati, gestito da un DBMS (DataBase Management System).
		\subsection{Datadog} 
		Servizio di monitoraggio per le applicazioni cloud-scale, \glo{database}, servizi, strumenti e \glo{server}.
		\subsection{Dbscan}
		Algoritmo di clustering e stima la densità attorno a ciascun punto (item) contando il numero di punti in un intorno specificato dall'utente.
		\subsection{Deep learning} 
		Per apprendimento profondo o approfondito si intende un insieme di tecniche basate su reti neurali artificiali organizzate in diversi strati, dove ogni strato calcola i valori per quello successivo affinché l'informazione venga elaborata in maniera sempre più completa.
		\subsection{Dendrogramma}
		Il dendrogramma è un albero utilizzato nelle tecniche di clustering per fornire una rappresentazione grafica del processo di raggruppamento delle istanze.
		\subsection{Design pattern} 
		Per design pattern si intende una descrizione o modello logico da applicare per la risoluzione di un problema che può presentarsi in diverse occasioni durante le fasi di progettazione e sviluppo del \glo{software}.
		\subsection{diagrams.net}
		\glo{Software} online per la creazione di diagrammi.
		\subsection{Difetto} 
		Con difetto si intende un errore nella scrittura del codice sorgente di un programma \glo{software} o nel testo di un documento.
		\subsection{Discord}
		Piattaforma di VoIP e messaggistica istantanea. Gli utenti comunicano con chiamate vocali, videochiamate, messaggi di testo, media e file in chat private o canali tematici dedicati.
		\subsection{Docker} 
		\glo{Software} per la creazione, sviluppo ed esecuzione di applicazione attraverso l'uso di container. I container permettono di incorporare un'applicazione e tutte le parti necessarie ad essa in un unico pacchetto.
		\subsection{DOM}
		Acronimo di Document Object Model, è una forma di rappresentazione dei documenti strutturati come modello orientato agli oggetti. È lo standard ufficiale del \glo{W3C} per la rappresentazione di documenti strutturati in maniera da essere neutrali sia per la lingua che per la \glo{piattaforma}.
		\subsection{Driver} 
		Insieme di \glo{procedure} \glo{software}, spesso scritte in linguaggio assembly, che danno la possibilità ad un sistema operativo di pilotare un dispositivo hardware.

	\newpage
	\section{E}
		\subsection{E-commerce} 
		Attività di acquisto e vendita di servizi e prodotti attraverso Internet. 
		\subsection{ESLint} 
		Strumento di analisi statica del codice per l'identificazione di problematiche relative ai modelli trovate nel codice \glo{JavaScript}.
		\subsection{Express.js} 
		\glo{Framework} per applicazioni web \glo{Node.js}, facilita la creazione di un \glo{server} http rispetto al pacchetto di default.
	
	\newpage
	\section{F}
		\subsection{FASTMAP} 
		Algoritmo di riduzione dimensionale non lineare adatto a compiti di indicizzazione, \glo{data mining} e visualizzazione di dataset.
		\subsection{Force Field}
		Termine utilizzato nell'ambito dei grafici disponibili per il progetto \textit{\NomeProgetto}. Il grafico Force Field traduce le distanze nello spazio a molte dimensioni i forze di attrazione e repulsione tra i punti proiettati nello spazio bidimensionale (o anche tridimensionale).
		\subsection{Fornitore} 
		Individuo, \glo{team} o azienda incaricato di realizzare il prodotto richiesto da un \glo{proponente}.
		\subsection{Framework} 
		Architettura logica di supporto su cui un \glo{software} può essere progettato e realizzato.	
		\subsection{Front-end} 
		Parte di applicazione con la quale l'utente interagisce direttamente, responsabile dell'acquisizione dei dati di ingresso e per la loro elaborazione. 
	
	\newpage
	\section{G}
		\subsection{GGobi} 
		Programma visivo \glo{open-source} che permette di esplorare e analizzare dati multi-dimensionali.
		\subsection{Git} 
		Sistema di \glo{versionamento} distribuito multi-piattaforma. Permette di versionare i sorgenti \glo{software}, documenti di testo e di collaborare alla loro realizzazione.
		\subsection{GitHub} 
		Servizio di hosting per progetti \glo{software}, che implementa lo strumento di controllo versione distribuito \glo{Git}.
		\subsection{GitLab} 
		\glo{Piattaforma} web \glo{open-source} che permette la gestione di \glo{repository} \glo{Git} e di funzioni trouble ticket.
		\subsection{Gmail} 
		Servizio di posta elettronica fornito da Google.
		\subsection{Google Drive} 
		Servizio di memorizzazione e sincronizzazione online fornito da Google, che permette la condivisione di file e la modifica collaborativa di documenti. 
		\subsection{GPS} 
		Global Positioning System, è un sistema di posizionamento satellitare che permette in ogni istante di localizzare la latitudine e la longitudine di un oggetto, grazie all'utilizzo di satelliti che stazionano nell'orbita terrestre.
		\subsection{GraphQL} 
		Linguaggio di interrogazione lato \glo{server} per interfacce di programmazione delle applicazioni (\glo{API}), in grado di fornire ai client unicamente i dati di cui hanno bisogno.
		\subsection{gRPC} 
		\glo{Framework} RPC \glo{open-source} sviluppato inizialmente da Google che sta prendendo una fetta di mercato sempre più importante per quanto riguarda le comunicazioni in architetture di microservizi, mobile e siti web.
		\subsection{GUI} 
		Graphical User Interface, è un tipo di interfaccia utente che consente l'interazione uomo-macchina in modo visuale utilizzando rappresentazioni grafiche piuttosto che utilizzando i comandi tipici di un'interfaccia a riga di comando.
	
	\newpage
	\section{H}
		\subsection{Heat Map} 
		Mappa di calore, è una rappresentazione grafica dei dati dove i singoli valori contenuti in una matrice sono rappresentati da colori.
		\subsection{Hover}
		Hover si verifica quando l'utente interagisce con un elemento utilizzando una periferica di puntamento, ma non necessariamente lo attiva. È generalmente innescato quando l'utente passa sopra un elemento con il cursore.
		\subsection{HTML} 
		Linguaggio a marcatori per ipertesti che gestisce la strutturazione delle pagine web.
			
	\newpage
	\section{I}
		\subsection{IaaS} 
		Infrastructure as a service, offerta di \glo{cloud} computing in cui un venditore fornisce agli utenti l'accesso alle risorse di calcolo, ad esempio \glo{server}, archiviazione e connessione di rete.
		\subsection{Incapsulamento} 
		Tecnica utilizzata per nascondere il funzionamento interno e la struttura interna di una parte di un programma, in modo da proteggere le altre parti del programma da eventuali cambiamenti.
		\subsection{Indice di Gulpease}
		L'Indice di Gulpease è un indice di leggibilità di un testo tarato sulla lingua italiana che considera due variabili linguistiche: la lunghezza della parola e la lunghezza della frase rispetto al numero delle lettere.
		\subsection{Information Hiding} 
		Letteralmente \qm{occultamento delle informazioni} indica il principio teorico dell'\glo{incapsulamento} dove gli attributi e i metodi dell'oggetto possono anche essere nascosti all'interno dell'oggetto stesso.
		\subsection{Inspection} 
		Tecnica di analisi statica che consiste nell'analizzare il prodotto solo nelle parti in cui si prevede che ci possano essere dei \glo{difetti}.
		\subsection{ISOMAP} 
		Algoritmo di riduzione dimensionale non lineare, consigliato quando si ha già qualche idea sulla struttura dei dati.
		\subsection{Issue} 
		All'interno di un \glo{Issue Tracking System} rappresenta un evento da gestire.
		\subsection{Issue Tracking System} 
		Un Issue Tracking System indica un sistema informatico il cui scopo è tracciare richieste di assistenza o problemi.

	\newpage
	\section{J}
		\subsection{Java} 
		Linguaggio di programmazione orientato agli oggetti a tipizzazione statica, progettato per essere il più possibile indipendente dalla \glo{piattaforma} di esecuzione.
		\subsection{Java Servlet} 
		Oggetti scritti in linguaggio \glo{Java} che operano all'interno di un \glo{server} web oppure un \glo{server} per applicazioni permettendo la creazione di applicazione web.
		\subsection{JavaScript} 
		Linguaggio di scripting orientato agli oggetti e agli eventi, utilizzato nella programmazione web sia lato client che lato \glo{server}.
		\subsection{Jest} 
		Progetto \glo{open-source} composto da una \glo{libreria} per testare codice \glo{JavaScript}. Può testare qualsiasi codice \glo{JavaScript} anche se si presta particolarmente per i test sui codici \glo{React}.
		\subsection{JSON}
		JSON (JavaScript Object Notation) è un semplice formato per lo scambio di dati. Per le persone è facile da leggere e scrivere, mentre per le macchine risulta facile da generare e analizzarne la sintassi.

	\newpage
	\section{K}
		\subsection{Keras} 
		\glo{Libreria} \glo{open-source} per l'\glo{apprendimento automatico} e le reti neurali, scritta in \glo{Python}.
		\subsection{Kotlin} 
		Linguaggio di programmazione \glo{open-source} e multi-paradigma creato con l'obiettivo di integrarsi con l'ambiente \glo{Java} e superando le limitazioni e criticità che il linguaggio \glo{Java} stesso ha.
		\subsection{Kubernetes} 
		\glo{Piattaforma} \glo{open-source} per automatizzare la distribuzione, la scalabilità e la gestione di applicazioni containerizzate.
		
	\newpage
	\section{L}
		\subsection{LaTeX}
		Programma di composizione tipografica indicato soprattutto per scrivere documenti di eccellente qualità.
		\subsection{Layout} Per layout si intende l'impaginazione e la struttura grafica di un elemento che viene visualizzato a video.
		\subsection{Leaflet} 
		\glo{Libreria} \glo{open-source} \glo{JavaScript} utilizzata per lo sviluppo di mappe interattive.
		\subsection{Learning Vector Quantization} 
		LVQ, è un algoritmo di classificazione supervisionato basato su prototipi. È inoltre la controparte supervisionata dei sistemi di quantizzazione vettoriale.
		\subsection{Libreria} 
		Raccolta di componenti che offrono servizi ad un livello di astrazione piuttosto basso, ovvero assemblare componenti semplici e predefiniti per ottenere strutture complesse specializzate.
		\subsection{LLE}
		Acronimo di Locally-linear embedding, algoritmo di riduzione dimensionale non lineare utile quando i dati sono disposti in una struttura simile a un manifold.
		\subsection{LowerCamelCase} 
		Deriva dalla pratica CamelCase che consiste nella scrittura di parole composte o frasi unendo tutte le parole tra loro, ma lasciando le loro iniziali maiuscole. LowerCamelCase è un nome convenzionale per indicare che la prima lettera deve essere minuscola (lower-case).
	
	\newpage
	\section{M}
		\subsection{Machine learning} 
		Branca dell'intelligenza artificiale che utilizza metodi statistici per migliorare progressivamente la performance di un algoritmo nell'identificare modelli nei dati.
		\subsection{Manifold} 
		\glo{Spazio topologico} che localmente è simile a uno \glo{spazio topologico} ben conosciuto (ad esempio lo spazio euclideo $n$-dimensionale), ma che globalmente può avere proprietà geometriche differenti.
		\subsection{MAPI} 
		Messaging Application Programming Interface, è un'architettura di messaggistica e un Component Object Model, basato sulle \glo{API} per Microsoft Windows. MAPI permette ai programmi client di diventare capaci di inviare un messaggio di posta elettronica, una in chiaro, o basato su chiamate RPC a un sottosistema di routine MAPI che interfaccia con alcuni \glo{server} di messaggistica.
		\subsection{Markup}
		Un linguaggio marcatore basato su un meccanismo sintattico che consente di definire e controllare il significato degli elementi contenuti in un documento o in un testo.
		\subsection{MiKTeX}
		MiKTeX è un elaboratore di testi basato su \glo{LaTeX}.
		\subsection{ml.js}
		ml.js è una \glo{libreria} \glo{JavaScript} volta a fornire funzionalità di \glo{machine learning} per applicazioni web.
		\subsection{MobX}
		MobX è una \glo{libreria} di gestione dello stato. Proprio come \glo{React}, che utilizza un \glo{DOM} virtuale per eseguire il rendering degli elementi dell'interfaccia utente nei browser, riducendo il numero di mutazioni \glo{DOM}, MobX fa la stessa cosa ma nello stato dell'applicazione.
		\subsection{Mocha} 
		Mocha è un \glo{framework} di test \glo{JavaScript} eseguito su \glo{Node.js} e su browser per lo svolgimento di test asincroni. Utilizzato con la \glo{libreria} di testing \glo{Chai}.
		\subsection{Modulo} 
		Un modulo è un singolo blocco di codice che fornisce funzionalità specifiche e strettamente collegate. I moduli definiscono e impongono limiti logici nel codice. 
		\subsection{MQTT} 
		Message Queue Telemetry Transport, è un protocollo ISO standard (ISO/IEC PRF 20922) di messaggistica leggero progettato per le situazioni in cui è richiesto un basso impatto e dove la banda è limitata.
		\subsection{MVC} 
		Model-view-controller è un pattern architetturale molto diffuso nello sviluppo di sistemi \glo{software} in grado di separare la logica di presentazione dei dati dalla \glo{logica di business}.
	
	\newpage
	\section{N}
		\subsection{Next.js} 
		\glo{Framework} che consente un facile rendering \glo{React} sul lato \glo{server} e fornisce molte altre funzionalità.
		\subsection{Node.js} 
		\glo{Framework} \glo{open-source} multi-piattaforma orientato agli eventi per l'esecuzione di codice \glo{JavaScript}. 
		\subsection{NoSQL} 
		Movimento che promuove sistemi \glo{software} dove la persistenza dei dati è in generale caratterizzata dal fatto di non utilizzare il modello relazionale.
		\subsection{NumPY} 
		\glo{Libreria} \glo{open-source} per il linguaggio di programmazione \glo{Python}, che aggiunge supporto a grandi matrici e array multi-dimensionali insieme a una vasta collezione di funzioni matematiche di alto livello per poter operare efficientemente su queste strutture dati.
	
	\newpage
	\section{O}
		\subsection{Open-source} 
		Termine utilizzato per riferirsi ad un \glo{software} in cui gli \glo{attori} rendono pubblico il codice sorgente. Tutto questo viene regolamentato tramite l'applicazione delle licenze d'uso. Il vantaggio è quello di permettere a programmatori distanti di coordinarsi e lavorare tutti allo stesso progetto.
		\subsection{OpenShift} 
		OpenShift è un \glo{Platform-as-a-Service} per applicazioni \glo{cloud} che rende semplice lo sviluppo, il deploy e la scalabilità di applicazioni \glo{cloud}.
	
	\newpage
	\section{P}
		\subsection{PaaS} 
		Platform as a Service, è un ambiente di sviluppo e distribuzione completo nel \glo{cloud}. Come \glo{IaaS}, include l'infrastruttura ma anche middleware, strumenti di sviluppo, sistemi di gestione dei \glo{database} e molto altro.
		\subsection{Pandas} 
		Pandas è una \glo{libreria} \glo{software} scritta per il linguaggio di programmazione \glo{Python} e impiegata per la manipolazione e analisi dei dati.
		\subsection{PCA}
		L'analisi delle componenti principali (principal component analysis) è una tecnica per la semplificazione dei dati utilizzata nell'ambito della statistica multi variata. Lo scopo della tecnica è quello di ridurre il numero più o meno elevato di variabili che descrivono un insieme di dati.
		\subsection{Piattaforma} 
		Una piattaforma è una struttura di base hardware e/o \glo{software} su cui si sviluppano le applicazioni e si eseguono programmi.
		\subsection{PoC}
		Vedere \glo{Proof of Concept}.
		\subsection{POI} 
		Point of Interest, o punto di interesse. I POI sono luoghi che potrebbero risultare utili o interessanti quando si è alla guida.
		\subsection{PostgreSQL}
		PostgreSQL è un \glo{database} ad oggetti e relazionale \glo{open-source}.
		\subsection{Power-up} 
		Un power-up o potenziamento, nel campo dei videogiochi, è un oggetto mostrato a video che conferisce una particolare abilità temporanea al giocatore o ne incrementa le statistiche quando raccolto.
		\subsection{Procedura} 
		Sequenza di operazioni da effettuare ordinatamente per raggiungere un certo scopo.
		\subsection{Product Baseline}
		Illustra la baseline architetturale del prodotto, in coerenza con la \glo{Technology Baseline}. Definisce eventuali \glo{design pattern}, diagrammi delle classi, diagrammi dei package e diagrammi delle attività.
		\subsection{Proiezione Lineare Multi Asse}
		Grafico utilizzato nel progetto \NomeProgetto. La Proiezione Lineare Multi Asse posiziona i punti dello spazio multi-dimensionale in un piano cartesiano, riducendo a 2 dimensioni anche dati con molte più dimensioni.
		\subsection{Proof of Concept}
		Per Proof of Concept si intende la dimostrazione pratica dei funzionamenti di base di un applicativo \glo{software}.
		\subsection{Proponente} 
		Figura fisica o giuridica, che presenta una proposta di \glo{capitolato}.
		\subsection{Protocollo asincrono}
		Un protocollo di comunicazione è un insieme di regole formalmente descritte che definiscono le modalità di comunicazione tra due o più entità. Per protocollo asincrono si intende un protocollo in cui l'intervallo tra il bit di stop di un carattere e il bit di start del carattere successivo è indefinito.
		\subsection{Python} 
		Linguaggio di programmazione dinamico orientato agli oggetti utilizzabile per molti tipi di sviluppo \glo{software}. Offre un forte supporto all'integrazione con altri linguaggi e programmi.

	\newpage
	\section{Q}
		\subsection{Query} 
		Interrogazione da parte di un utente di un \glo{database} per estrarre o aggiornare i dati che soddisfano un certo criterio di ricerca.
		\subsection{Qt} 
		\glo{Libreria} multipiattaforma per lo sviluppo di programmi con interfaccia grafica tramite l'uso di widget.
	
	\newpage
	\section{R}
		\subsection{Rancher} 
		\glo{Piattaforma} di gestione per container \glo{Docker}. Oltre a includere una distribuzione \glo{Kubernetes} e l'opzione tra Docker Swarm e Apache Mesos, include anche servizi di infrastruttura modulare tra cui networking, bilanciamento del carico, rilevamento di servizi, monitoraggio e ripristino.
		\subsection{React} 
		Libreria \glo{JavaScript} per la creazione di interfacce utente interattive.
		\subsection{Real-time} 
		In informatica, un sistema real-time è un calcolatore in cui la correttezza del risultato delle sue computazioni dipende non solo dalla correttezza logica ma anche dal quella temporale. Quest'ultima è spesso espressa come tempo massimo di risposta.
		\subsection{Repository} 
		Archivio o sito web nel quale sono raccolti e conservati dati ed informazioni in formato digitale.
		\subsection{Requisito} 
		Una condizione o capacità che deve essere verificata o posseduta da un sistema o un componente di un sistema per soddisfare un contratto, uno standard, una specifica o qualsiasi altro documento formalmente specificato.
		\subsection{Revisione} 
		Nuovo esame volto a controllare, correggere, modificare ed accertare i risultati dell'esame già consegnato.
		\subsection{RFID} 
		Radio-frequency identification, è una tecnologia per l'identificazione e/o memorizzazione automatica di informazioni inerenti a oggetti, animali o persone basata sulla capacità di memorizzazione di dati da parte di particolari etichette elettroniche, chiamate tag.
	
	\newpage
	\section{S}
		\subsection{Scala}
		SCAlable LAnguage, è un linguaggio di programmazione general-purpose studiato per integrare le caratteristiche e funzionalità dei linguaggi orientati agli oggetti e dei linguaggi funzionali.
		\subsection{Scatter plot}
		Chiamato anche grafico di dispersione, è un tipo di grafico in cui due variabili di un set di dati sono riportate su uno spazio cartesiano.
		\subsection{Scatter plot Matrix}
		Termine utilizzato nell'ambito dei grafici disponibili per il progetto \NomeProgetto. Un grafico a matrice \glo{scatter plot} è uno strumento di esplorazione dei dati che consente di confrontare diversi set di dati per cercare modelli e relazioni. \\
		Il grafico ha due componenti principali: una matrice di piccoli \glo{scatter plot} per ciascuno dei campi e una finestra di anteprima più grande che mostra lo \glo{scatter plot} per una coppia selezionata di campi in modo più dettagliato. È inoltre possibile abilitare il plottaggio degli istogrammi, mostrando la distribuzione dei valori per ciascuno dei campi.
		\subsection{Spazio topologico} 
		In matematica, il concetto di spazio topologico è un concetto molto generale di spazio, accompagnato da una nozione di "vicinanza" definita nel modo più debole possibile.
		\subsection{SceneKit} 
		\glo{Framework} grafico ad alto livello, permette di creare scene animate ed effetti in 3D.
		\subsection{SciKit-learn} 
		\glo{Libreria} \glo{open-source} di \glo{apprendimento automatico} per il linguaggio di programmazione \glo{Python}.
		\subsection{Self Organizing Map} 
		Tipo di organizzazione di processi di informazione in rete analoga alle reti neurali artificiali. Sono utili per la visualizzazione di dati di dimensione elevata.
		\subsection{Server}
		Componente o sottosistema informatico di elaborazione e gestione del traffico di informazioni che fornisce, a livello logico e fisico, un qualunque tipo di servizio ad altre componenti che ne fanno richiesta attraverso una rete di computer, all'interno di un sistema informatico o anche direttamente in locale su un computer.
		\subsection{Serverless} 
		Con il termine serverless si intende un network la cui gestione non viene incentrata su dei \glo{server}, ma viene dislocata fra i vari utenti che utilizzano il network stesso, quindi il lavoro necessario di gestione del network viene eseguito dagli stessi utilizzatori.
		\subsection{Servlet}
		Vedere \glo{Java Servlet}.
		\subsection{Skype} 
		Programma di messaggistica che permette di effettuare chiamate e videochiamate di gruppo, mantenuto da Microsoft.
		\subsection{Snapshot}
		Immagine corrispondente a ciò che viene visualizzato in un determinato istante sullo schermo di un monitor, di un televisore o di un qualunque dispositivo video. 
		\subsection{Software} 
		Insieme delle \glo{procedure} e delle istruzioni in un sistema di elaborazione dati.
		\subsection{SonarQube}
		\glo{Piattaforma} per il controllo della qualità del codice tramite l'esecuzione di revisioni automatiche ed analisi statica.
		\subsection{SPICE}
		ISO/IEC 15504, anche conosciuta come SPICE (Software Process improvement and Capability Determination), è un insieme di nove documenti di standard tecnici relativi ai processi di sviluppo del \glo{software} e relative funzioni di business e, in particolare, alla loro valutazione.
		\subsection{SpriteKit} 
		\glo{Framework} che facilita la creazione di giochi 2D. Si integra con \glo{SceneKit}.
		\subsection{SQL} 
		Acronimo di Structured Query Language, è un linguaggio di interrogazione usato per l'interazione con i principali Database Management Systems (DBMS) soprattutto relazionali.
		\subsection{Stakeholder} 
		Insieme di soggetti, individui od organizzazioni attivamente coinvolti nel ciclo di vita del \glo{software} avendo influenza sul prodotto o sul processo.
		\subsection{Stub} 
		I test stub sono programmi in grado di simulare il comportamento di componenti o moduli \glo{software} del prodotto che si vuole testare.
		\subsection{SVG} 
		Acronimo di Scalable Vector Graphics, indica un particolare formato che è in grado di visualizzare oggetti di grafica vettoriale e quindi di salvare immagini in modo che siano ridimensionate a piacere senza perdere in risoluzione grafica.
		\subsection{Swift} 
		Linguaggio di programmazione orientato agli oggetti utilizzato per lo sviluppo di sistemi e applicativi Apple.
		\subsection{SwiftUI} 
		\glo{Framework} per lo sviluppo delle User Interface dei sistemi Apple.
	
	\newpage
	\section{T}
		\subsection{t-SNE} 
		t-distributed Stochastic Neighbor Embedding, è un algoritmo di riduzione della dimensionalità ampiamente utilizzato come strumento di \glo{apprendimento automatico} in molti ambiti di ricerca.
		\subsection{Team}
		Gruppo di persone che collaborano a uno stesso lavoro o per uno stesso fine.
		\subsection{Technology Baseline}
		Baseline tecnologica che motiva le tecnologie, i \glo{framework} e le \glo{librerie} selezionate per la realizzazione del prodotto. Ne dimostra l'adeguatezza e fattibilità tramite un \glo{Proof of Concept} che rappresenta la baseline per lo sviluppo.
		\subsection{Telegram} 
		Servizio di messaggistica istantanea e broadcasting basato su \glo{cloud}.
		\subsection{TensorFlow} 
		\glo{Libreria} \glo{software} \glo{open-source} per l'\glo{apprendimento automatico}, che fornisce moduli sperimentati e ottimizzati, utili nella realizzazione di algoritmi per diversi tipi di compiti percettivi e di comprensione del linguaggio.
		\subsection{Texmaker}
		Texmaker è un editor di testo \glo{open-source} utilizzato per sviluppare documenti con \glo{LaTeX}.
		\subsection{TeXstudio}
		TeXstudio è un \glo{ambiente di sviluppo} che permette di gestire file \glo{LaTeX} con grande immediatezza d'uso. 
		\subsection{Token} 
		Letteralmente \qm{gettone}, termine che si inserisce nel contesto dei diagrammi \glo{UML}. Un token rappresenta il comportamento dei processi.
		\subsection{Tomcat} 
		\glo{Server} web nella forma di contenitore \glo{servlet} \glo{open-source}. Implementa le specifiche JavaServer Pages e \glo{servlet}, fornendo quindi una \glo{piattaforma} \glo{software} per l'esecuzione di applicazioni web sviluppate in linguaggio \glo{Java}.
		\subsection{Truffle} 
		\glo{Framework} per lo sviluppo ed il testing di codice di una \glo{blockchain}.
		\subsection{TSV}
		Il formato di file TSV sta per "Valori separati da tabulazione", e questi TSV file vengono creati e utilizzati da molte applicazioni per fogli di calcolo.
		\subsection{Typescript} Linguaggio di programmazione \glo{open-source} che punta ad estendere \glo{JavaScript}, aggiungendo tipi al linguaggio.
	
	\newpage
	\section{U}
		\subsection{Uber} 
		Azienda che fornisce un servizio di trasporto automobilistico privato attraverso un'applicazione mobile che mette in collegamento diretto passeggeri e autisti.
		\subsection{UMAP} 
		Uniform Manifold Approximation and Projection, è una tecnica di riduzione dimensionale che può essere utilizzata per la visualizzazione dei dati multi-dimensionali, simile a \glo{t-SNE}.	
		\subsection{UML} 
		Unified Modeling Language, linguaggio che permette, tramite l'utilizzo di modelli visuali, di analizzare, descrivere, specificare e documentare un sistema \glo{software} anche complesso.
		\subsection{Unità} 
		Insieme di uno o più componenti di un programma dotato di funzionamento autonomo. Nella definizione di architettura \glo{software} per unità si intende una piccola parte su cui effettuare dei test. Un'unità si occupa di soddisfare uno o più \glo{requisiti}.
		\subsection{UpperCamelCase} 
		Deriva dalla pratica CamelCase che consiste nella scrittura di parole composte unite, ma lasciando le loro iniziali in maiuscolo.
	\newpage
	\section{V}
		\subsection{Validatore}
		Un validatore è uno strumento in grado di controllare la conformità di un file rispetto a uno standard.
		\subsection{Validazione}
		La validazione è un’attività di controllo mirata a confrontare il risultato di una fase del
		processo di sviluppo con i requisiti del prodotto.
		\subsection{Vanilla}
		Aggettivo che, riferito a un linguaggio, indica l'utilizzo di un linguaggio di programmazione senza l'utilizzo di altri \glo{framework}.
		\subsection{Versionamento} 
		Si tratta di una pratica che tiene traccia e permette di controllare i cambiamenti al codice sorgente prodotti da ciascun sviluppatore, condividendone allo stesso tempo la versione più aggiornata o modificata da ciascuno mostrando lo stato di avanzamento del lavoro di sviluppo.
		
	\newpage
	\section{W}
		\subsection{W3C}
		Organizzazione non governativa internazionale che ha come scopo quello di sviluppare tutte le potenzialità del World Wide Web.
		\subsection{Walkthrough} 
		Tecnica di verifica attraverso la quale il \glo{verificatore} esegue una lettura ed un controllo completo dell'intero documento alla ricerca di eventuali errori.
		\subsection{WebKit}
		WebKit è un motore di rendering per browser web utilizzato per il rendering delle pagine web.
		\subsection{Wizard}
		Wizard indica una \glo{procedura}, generalmente inglobata in un'applicazione più complessa, che permette all'utente di eseguire determinate operazioni tramite una serie di passi successivi (es. installazione o configurazione di un applicativo).
	
	\newpage
	\section{X}
		\subsection{XML} 
		XML (eXtensible Markup Language) è un metalinguaggio per la definizione di linguaggi di \glo{markup}.
	
	\newpage
	\section{Z}
		\subsection{Zoom} 
		\glo{Piattaforma} \glo{cloud} che offre strumenti per videoconferenze.		
		
\end{document}