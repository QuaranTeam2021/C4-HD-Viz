\section{Introduzione}

% inserire tutte il codice della sezione in questo file ed eliminare cartella subsections
% OPPURE inserire più file nella sottocartella subsections e includere i file 
% come mostrato sotto:

% es: per includere 2 files chiamati 
% nomefile1.tex, nomefile2.tex
% \subimport{subsections/}{nomefile1.tex}
% \subimport{subsections/}{nomefile2.tex}

% oppure
% \subsectionInFile{nomefile1.tex}
% \subsectionInFile{nomefile2.tex}

\subsection{Scopo del documento}
Questo documento presenta la descrizione di tutti i \glo{casi d'uso} e l'analisi dei \glo{requisiti} individuati a seguito dell'analisi del \glo{capitolato} C4 proposto dalla \textit{\Proponente}. Tutto il materiale presente nel documento è stato prodotto a seguito dello studio e della comprensione del \glo{capitolato} C4, degli incontri con il \glo{proponente} e di discussioni interne al \glo{team}.

\subsection{Glossario}
Viene fornito il \Glossariov{v\versionGlossario{}}, una raccolta di tutti i termini con un significato particolare, che vengono definiti e descritti al fine di evitare ambiguità. In tutti i documenti i termini definiti nel \Glossariov{v\versionGlossario{}} saranno identificati con una G a pedice. 

\subsection{Riferimenti}
\subsubsection{Riferimenti normativi}
\begin{itemize}
	\item \textbf{\NdP}: \NdPv{v\versionNdP{}};
	\item \textbf{\glo{Capitolato} d'appalto C4 - \textit{HD Viz}:} \\
	\url{https://www.math.unipd.it/~tullio/IS-1/2020/Progetto/C4.pdf}.
\end{itemize}

\subsubsection{Riferimenti informativi}
\begin{itemize}
	\item \textbf{Slide del corso di Ingegneria del Software - Analisi dei requisiti:} \\
	\url{https://www.math.unipd.it/~tullio/IS-1/2020/Dispense/L07.pdf};
	\item \textbf{Slide del corso di Ingegneria del Software - Diagrammi dei casi d'uso:} \\
	\url{https://www.math.unipd.it/~rcardin/swea/2021/Diagrammi\%20Use\%20Case\_4x4.pdf}.	
\end{itemize}