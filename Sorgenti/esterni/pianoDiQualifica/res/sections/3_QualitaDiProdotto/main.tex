\section{Qualità di prodotto}

\subsection{Introduzione}
Nel valutare la qualità si prende come riferimento lo standard \textbf{ISO/IEC 9126}, dove sono definiti i parametri cui attenersi per ottenere un prodotto di qualità. Questi parametri quantificano il grado di raggiungimento della qualità di prodotto. I parametri definiti nello standard non sono gli unici utilizzati, si è deciso di utilizzarne anche altri per quantificare la qualità dei documenti redatti.
La presentazione delle metriche adottate e degli strumenti atti a garantirne un valore entro una soglia è riportata nel documento \NdPv{v\versionNdP{}}, mentre i valori ritenuti accettabili e ottimali sono riportati in seguito per ogni metrica adottata.

\subsection{Monitoraggio delle caratteristiche del prodotto}
Le caratteristiche monitorate sono:
\begin{itemize}
    \item \textbf{QC-1} Manutenibilità;
    \item \textbf{QC-2} Usabilità;
    \item \textbf{QC-3} Efficienza;
    \item \textbf{QC-4} Funzionalità;
    \item \textbf{QC-5} Comprensibilità.
\end{itemize}

\subsubsection{QC-1 Manutenibilità}
Le metriche di manutenibilità sono descritte nella sezione §2.2.3.2 delle \NdPv{v\versionNdP{}}.

\paragraph{Metriche utilizzate}
\begin{itemize}
	\item \textbf{QM-PROD-1}: Complessità ciclomatica (CC);
	\item \textbf{QM-PROD-2}: Densità di duplicazione (DD);
	\item \textbf{QM-PROD-3}: Numero di bug (NB);
	\item \textbf{QM-PROD-4}: Numero di code smell (NCS).
	\item \textbf{QM-PROD-5}: Comprensione del codice (CDC).
\end{itemize}

\paragraph{Indici di qualità}
\begin{tabmetriche}{Indici di qualità per le metriche di manutenibilità del prodotto}{Indici QC1 - Manutenibilità del prodotto}
  QM-PROD-1 & Complessità ciclomatica (CC) & $\leq$ $20$ & $\leq$ $10$ \\ 
  \hline
  QM-PROD-2 & Densità di duplicazione (DD) & $\leq$ $15\%$ & $0\%$ \\
  \hline
  QM-PROD-3 & Numero di bug (NB) & $\leq$ $10$ & $\leq$ $4$ \\ 
  \hline
  QM-PROD-4 & Numero di code smell (NCS) & $\leq$ $15$ & $\leq$ $8$ \\
  \hline
  QM-PROD-5 & Comprensione del codice (CDC) & $\geq$ $60\%$ & $\geq$ $80\%$ \\
\end{tabmetriche}

\subsubsection{QC-2 Usabilità}
Le metriche di usabilità sono descritte nella sezione §2.2.3.3 delle \NdPv{v\versionNdP{}}.

\paragraph{Metriche utilizzate}
\begin{itemize}
	\item \textbf{QM-PROD-6}: Click necessari (CN).
\end{itemize}

\paragraph{Indici di qualità}
\begin{tabmetriche}{Indici di qualità per le metriche di usabilità del prodotto}{Indici QC2 - Usabilità del prodotto}
	QM-PROD-6 & Click necessari (CN) & $\leq$ $8$ & $\leq$ $6$ \\ 
\end{tabmetriche}

\subsubsection{QC-3 Efficienza}
Le metriche di efficienza sono descritte nella sezione §2.2.3.4 delle \NdPv{v\versionNdP{}}.

\paragraph{Metriche utilizzate}
\begin{itemize}
	\item \textbf{QM-PROD-7}: Tempo medio di risposta (TMR).
\end{itemize}

\paragraph{Indici di qualità}
\begin{tabmetriche}{Indici di qualità per le metriche di efficienza del prodotto}{Indici QC3 - Efficienza del prodotto}
	QM-PROD-7 & Tempo medio di risposta (TMR) & $5$ $secondi$ & $3$ $secondi$ \\ 
\end{tabmetriche}

\subsubsection{QC-4 Funzionalità}
Le metriche di funzionalità sono descritte nella sezione §2.2.3.5 delle \NdPv{v\versionNdP{}}.

\paragraph{Metriche utilizzate}
\begin{itemize}
	\item \textbf{QM-PROD-8}: Errori di utilizzo (EDU).
\end{itemize}

\paragraph{Indici di qualità}
\begin{tabmetriche}{Indici di qualità per le metriche di funzionalità del prodotto}{Indici QC4 - Funzionalità del prodotto}
	QM-PROD-8 & Errori di utilizzo (EDU) & $0$ & $0$ \\ 
\end{tabmetriche}

\subsubsection{QC-5 Comprensibilità}
Le metriche di comprensibilità sono descritte nella sezione §3.1.3.1 delle \NdPv{v\versionNdP{}}.

\paragraph{Metriche utilizzate}
\begin{itemize}
	\item \textbf{QM-PROD-9}: \glo{Indice di GULPEASE} (IDG);
	\item \textbf{QM-PROD-10}: correttezza ortografica (CO).
\end{itemize}

\paragraph{Indici di qualità}
\begin{tabmetriche}{Indici di qualità per le metriche di comprensione del prodotto}{Indici QC5 - Comprensibilità del prodotto}
  QM-PROD-9 & \glo{Indice di GULPEASE} (IDG) & $\leq$ $80$ & $\leq$ $60$ \\ 
  \hline
  QM-PROD-10 & Correttezza ortografica (CO) & $0$ & $0$ \\
\end{tabmetriche}