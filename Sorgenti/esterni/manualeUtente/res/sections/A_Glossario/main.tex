\section{Glossario}
\label{sec:glossario}

% no indentazione paragrafo
\setlength{\parindent}{0pt}
Di seguito non sono riportate le definizioni di funzioni matematiche e algoritmi per i quali si è ritenuto poco utile fornire una definizione concisa e povera di formalismi matematici che si adattasse a un glossario.

Si rimanda dunque a §\ref{sub:riferimenti} - Riferimenti per la definizione dei seguenti termini:
\begin{itemize}
 \item \textsc{algoritmi di \glo{riduzione dimensionale}}: FASTMAP, ISOMAP, LLE, t-SNE, UMAP;
 \item \textsc{funzioni di calcolo delle \glo{distanze}}: distanza di Canberra, distanza di Chebyshev, distanza Cosine, distanza Euclidea, distanza Euclidea Quadrata, distanza di Manhattan.
\end{itemize}


\subsection*{B}
\paragraph*{Browser}
Applicazione per la fruizione di contenuti web.
\paragraph*{Brush}
In uno Scatter plot Matrix, l'operazione di Brush permette di selezionare un insieme di punti in uno Scatter plot e di vedere evidenziati quei punti in tutti i restanti Scatter plot della matrice.
\paragraph*{Bug}
Guasto che porta al malfunzionamento del software, tipicamente dovuto ad un errore nella scrittura del codice sorgente di un programma.
\paragraph*{Build}
Trasformazione del codice sorgente in codice eseguibile.
\subsection*{C}
\paragraph*{Checkbox}
Controllo grafico con cui l'utente può effettuare selezioni multiple. 
\paragraph*{Cluster}
Gruppo di oggetti di un insieme. Può essere dato in partenza o può essere ricavato da particolari algoritmi di apprendimento non supervisionato, detti di \textit{clustering}. 
\paragraph*{CSV}
Acronimo di \textit{Comma-Separated Values}, è un formato di file utilizzato per la rappresentazione di una tabella di dati, che utilizza il carattere delimitatore virgola ',' o punto e virgola ';' per separare le celle. 
\subsection*{D}
\paragraph*{Database}
Insieme organizzato di dati, gestito da un DBMS (DataBase Management System).
\paragraph*{Dati multi-dimensionali}
Dataset in cui il numero di features per ogni record del dataset è $>> 3$. 
\paragraph*{Diagonale}
In una matrice $A$, la diagonale è l'insieme di elementi $A_{i,j}$ tali che $i = j$.
\paragraph*{Distanza}
Misura della "lontananza" tra due punti di un insieme al quale si possa attribuire qualche carattere spaziale.
\paragraph*{DSV}
Acronimo di \textit{Delimiter-Separated Values}, è una famiglia di formati di file utilizzati per la rappresentazione di una tabelle di dati, che utilizzano un carattere delimitatore per separare le celle. \glo{CSV} e \glo{TSV} sono formati DSV.
\subsection*{E}
\paragraph*{EDA}
Acronimo di \textit{Exploratory Data Analysis}, parte dell'attività di Data Analysis, nella quale l'analista ricerca associazioni, sequenze ripetute nascoste o pattern nei dati. 
\subsection*{F}
\paragraph*{Forza}
Grandezza fisica vettoriale che si manifesta nell'interazione reciproca di due o più corpi sia a livello macroscopico, sia a livello delle particelle elementari. 
\subsection*{G}
\paragraph*{Grafo}
In Matematica Discreta, un grafo è una tripla $(V,E,f)$, dove $V$ è detto insieme dei nodi, $E$ è detto insieme degli archi e $f$ è una funzione che associa ad ogni arco $e$ in $E$ due vertici $u,v$ in $V$. 
\paragraph*{Git}
Sistema di \glo{versionamento} distribuito multi-piattaforma. Permette di versionare i sorgenti \glo{software}, documenti di testo e di collaborare alla loro realizzazione. 
\paragraph*{GitHub}
Servizio di hosting per progetti \glo{software}, che implementa lo strumento di controllo versione distribuito \glo{Git}. 
\subsection*{I}
\paragraph*{Integratore di Verlet}
Metodo numerico per il calcolo delle Equazioni del moto di Newton.
\subsection*{M}
\paragraph*{Matrice delle distanze}
Matrice quadrata contenente le distanze tra ogni coppia di un insieme. 
\subsection*{N}
\paragraph*{Node.js}
Framework open-source multi-piattaforma orientato agli eventi per l'esecuzione di codice JavaScript. 

\paragraph*{npm}
Acronimo di \textit{Node.js Package Manager}, gestore di pacchetti predefinito per Node.js. 
\subsection*{O}
\paragraph*{Origine}
Nel sistema di riferimento cartesiano, il punto nel quale si intersecano tutte le rette orientate dette \textit{assi}. 
\subsection*{P}
\paragraph*{PostgreSQL}
Database ad oggetti e relazionale \glo{open-source}, focalizzato sulla scalabilità e sulla massima aderenza allo standard SQL. 
\paragraph*{Proiezione}
Trasformazione lineare definita da uno spazio vettoriale in sé stesso. 
\subsection*{R}
\paragraph*{Radio button}
Controllo grafico che consente all'utente di effettuare una scelta singola esclusiva nell'ambito di un insieme predefinito di opzioni o possibili scelte. 
\paragraph*{Repository}
Archivio o sito web nel quale sono raccolti e conservati dati ed informazioni in formato digitale. 
\paragraph*{Riduzione dimensionale}
Algoritmo di trasformazione di dati da uno spazio 	\glo{multi-dimensionale} ad uno spazio con meno dimensioni. Un algoritmo di riduzione dimensionale cerca di mantenere le caratteristiche dello spazio \glo{multi-dimensionale} anche nello spazio con meno dimensioni. 
\subsection*{S}
\paragraph*{Scatter plot}
Tipo di grafico in cui due variabili di un set di dati sono riportate su uno spazio cartesiano. Viene chiamato anche grafico di dispersione. \\
Scatter plot Matrix e Proiezione Lineare multi asse sono varianti di Scatter plot. 
\paragraph*{Server}
Componente o sottosistema informatico di elaborazione e gestione del traffico di informazioni che fornisce, a livello logico e fisico, un qualunque tipo di servizio ad altre componenti che ne fanno richiesta attraverso una rete di computer, all'interno di un sistema informatico o anche direttamente in locale su un computer. 
\paragraph*{Slider}
Componente grafico con il quale un utente può impostare un valore muovendo un indicatore, solitamente con uno spostamento orizzontale. 
\paragraph*{SVG}
Acronimo di \textit{Scalable Vector Graphics}, indica un particolare formato che è in grado di visualizzare oggetti di grafica vettoriale e quindi di salvare immagini in modo che siano ridimensionate a piacere senza perdere in risoluzione grafica. 
\subsection*{T}
\paragraph*{TSV}
Acronimo di \textit{Tab-Separated Values}, è un formato di file utilizzato per la rappresentazione di una tabella di dati, che utilizza il carattere delimitatore tabulazione per separare le celle.
\subsection*{V}
\paragraph*{Validatore}
Un validatore è uno strumento in grado di controllare la conformità di un file rispetto a uno standard. 