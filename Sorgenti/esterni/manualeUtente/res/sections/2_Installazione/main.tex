\section{Installazione}
\label{sec:installazione}

% inserire tutte il codice della sezione in questo file ed eliminare cartella subsections
% OPPURE inserire più file nella sottocartella subsections e includere i file 
% come mostrato sotto:

% es: per includere 2 files chiamati 
% nomefile1.tex, nomefile2.tex
% \subimport{subsections/}{nomefile1.tex}
% \subimport{subsections/}{nomefile2.tex}

% oppure
% \subsectionInFile{nomefile1.tex}
% \subsectionInFile{nomefile2.tex}


%probabili sezioni:

% * prerequisiti minimi di sistema, clonazione git o download zip, npm start o build... 

\subsection{Requisiti}
\paragraph*{Sistema operativo}
\begin{itemize}
	\item Windows 10;
	\item Ubuntu 20.04 o superiore, o distribuzioni Linux derivate.
\end{itemize}

\paragraph*{Browser}
\begin{itemize}
	\item \textbf{Mozilla Firefox} v85.0 o superiore;
	\item \textbf{Google Chrome} v87.0 o superiore. 
\end{itemize}
È consigliabile avere impostato uno dei due \glo{browser} elencati come \glo{browser} predefinito, perché la procedura indicata in §\ref{subsubsec:avvio_client} (Installazione - Avvio del client) apre in automatico una pagina del \glo{browser} predefinito. 
È comunque possibile aprire l'applicazione da un \glo{browser} non predefinito in seguito, al seguente URL:
\begin{center}
	\texttt{http:localhost/\textit{<numero\_porta>}}
\end{center}
Non è garantita la piena compatibilità per \glo{browser} derivati dai suddetti o per versioni precedenti dei suddetti.

\paragraph*{Requisiti software}
\begin{itemize}
	\item \glo{\textbf{Node.js}} 14.16.0 o superiore;
	\item \glo{\textbf{npm}} 7.6.3 o superiore;
	\item \glo{\textbf{PostgreSQL}} 13.2 o superiore (opzionale\footnote{necessario solo per il collegamento con il \glo{database}}). 
\end{itemize}

\paragraph*{Requisiti hardware}

\begin{itemize}
	\item \textbf{RAM}: necessari almeno 4 GB, consigliati almeno 8 GB;
	\item \textbf{connessione a Internet}. 
\end{itemize}

\subsection{Installazione e avvio}
\subsubsection{Preliminari}
\begin{enumerate}
\item Aprire la pagina del \glo{repository} \glo{GitHub} di \textit{HD Viz} al seguente link:
\begin{center}
	\url{https://github.com/QuaranTeam2021/HD-Viz}
\end{center} 
\item Selezionare \textbf{Code > Download Zip};
\item Salvare l'archivio \textit{.zip} ed estrarre i file nella locazione desiderata;
\item Aprire la cartella \texttt{HD-Viz-main}.
\end{enumerate}
In alternativa, in un sistema con \texttt{wget} e \texttt{unzip} installati\footnote{installati di default in ogni distribuzione Linux, necessaria installazione manuale per Windows} aprire il terminale\footnote{terminale per Linux o Windows PowerShell per Windows} e digitare le seguenti istruzioni:
\begin{lstlisting}
wget https://github.com/QuaranTeam2021/HD-Viz/archive/main.zip
unzip main.zip
cd HD-Viz-main
\end{lstlisting}

\subsubsection{Avvio del client}
\label{subsubsec:avvio_client}
Una volta posizionati all'interno della cartella \texttt{HD-Viz-main}:
\begin{enumerate}
	\item Aprire la cartella \texttt{client} e aprire un terminale posizionato nella cartella \texttt{client} (se non già aperto);
	\item Se si cerca di avviare per la prima volta \textit{HD Viz}, digitare il comando:
	\begin{lstlisting}
npm install
	\end{lstlisting}
	\item Eseguire la \glo{build} dell'applicazione con il comando:
	\begin{lstlisting}
npm run build
	\end{lstlisting}
	\item Se si cerca di avviare per la prima volta \textit{HD Viz}, digitare il comando:
	\begin{lstlisting}
npm install -g serve
	\end{lstlisting}
	\item Digitare infine il comando:
	\begin{lstlisting}
serve -s build -l 3000
	\end{lstlisting}
	% sostituendo \texttt{<numero_porta>} con 3000
	\item Attendere l'apertura automatica del \glo{browser} predefinito.
\end{enumerate}

\subsubsection{Configurazione ed avvio del server (opzionale)}
Il \glo{server} viene sfruttato per interagire con un \glo{database} esterno, pertanto la sua installazione ed esecuzione è opzionale.
Il \glo{database} utilizzato è \texttt{PostgreSQL}, scaricabile al seguente link:
\begin{center}
	\url{https://www.postgresql.org/download/}
\end{center}
\label{subsubsec:avvio_server}
Una volta posizionati all'interno della cartella \texttt{HD-Viz-main}:
\begin{enumerate}

	\item Se si cerca di avviare per la prima volta la componente server, aprire un terminale nella cartella \texttt{HD-Viz-main} e digitare il comando:
	\begin{lstlisting}
npm install
	\end{lstlisting}

	\item Aprire la cartella \texttt{server} e creare un file \texttt{.env} rispettando la seguente notazione: 
	\begin{lstlisting}
HDVIZ_USER = postgres
HDVIZ_PASSWORD = postgres
HDVIZ_HOST = localhost
HDVIZ_PORT = 5432
HDVIZ_DATABASE = hdviz
	\end{lstlisting}
	Inserire come valore dei campi HDVIZ\_USER, HDVIZ\_PASSWORD e HDVIZ\_DATABASE rispettivamente il nome utente, la password e il nome del \glo{database} a cui ci si vuole connettere. \\

	\item Spostarsi con il terminale nella cartella \texttt{server} e avviare il \glo{server} con il comando:
	\begin{lstlisting}
node server
	\end{lstlisting}
\end{enumerate}
Il \glo{server} sarà in esecuzione alla porta \texttt{5000}.