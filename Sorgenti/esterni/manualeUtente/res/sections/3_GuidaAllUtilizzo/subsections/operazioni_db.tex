\newpage
\subsection{Operazioni sul database}

\subsubsection{Importazione dati}
\label{sub:importazione_db}

Dopo aver cliccato sul pulsante "DATABASE" come modalità di importazione dei dati scelta, si aprirà una finestra con i seguenti due menù  (figura \ref{fig:import_db}):
\begin{itemize}
	\item \textbf{Tabelle}: in cui è possibile selezionare uno dei dataset presenti nel database;
	\item \textbf{Colonne}: in cui è possibile selezionare le dimensioni del dataset selezionato che si vogliono importare.
\end{itemize}
Nel caso in cui non vengano selezionate delle colonne tramite il menù "Colonne", verrà importato l'intero dataset con tutte le dimensioni presenti nel \glo{database}. \\
Al termine delle selezioni è sufficiente cliccare il pulsante "SCEGLI QUESTI DATI" per completare l'operazione di importazione.

\begin{figure}[htpb]
	\centering
	\begin{subfigure}[c]{.75\linewidth}
	\includegraphics[width=\linewidth]{images/import_db.png}
	\end{subfigure}
\caption{Importazione di un dataset dal database}
\label{fig:import_db}
\end{figure}

\subsubsection{Aggiunta di un dataset}

Accedendo alla sezione "Gestisci Database" è possibile caricare i dati contenuti in un file \glo{CSV} o \glo{TSV} nel \glo{database} esterno tramite i seguenti passaggi (figura \ref{fig:gestionedb}):
\begin{enumerate}
	\item Cliccare sul pulsante di aggiunta (+) e scegliere il file .\glo{csv} o .\glo{tsv} contenente i dati che vogliamo caricare nel \glo{database}. 
	Per ulteriori informazioni su caratteristiche obbligatorie dei file .\glo{csv} o .\glo{tsv} si rimanda a §\ref{sub:caricamento_dati};
	\item Inserire il nome con cui si vuole salvare il dataset nel \glo{database} nell'area di testo apposita.
	Nel caso in cui non venga esplicitamente inserito un nome verrà utilizzato il nome del file caricato;
	\item Confermare l'operazione premendo il tasto "INVIA".
\end{enumerate}

\subsubsection{Eliminazione di un dataset}

Accedendo alla sezione "Gestisci Database" è possibile eliminare un dataset contenuto nel \glo{database} tramite i seguenti passaggi (figura \ref{fig:gestionedb}) cliccando sul pulsante con l'icona di eliminazione situato a fianco del nome del dataset che vogliamo eliminare.

\begin{figure}[htpb]
	\begin{subfigure}[c]{\linewidth}
	\includegraphics[width=\linewidth]{images/gestionedb.png}
	\end{subfigure}
\caption{Operazioni con il database}
\label{fig:gestionedb}
\end{figure}
