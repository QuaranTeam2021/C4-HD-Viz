\subsection{Navigazione nel sito}
All'apertura dell'applicazione viene visualizzata la schermata sottostante:
\begin{figure}[htpb]
	\includegraphics[width=\linewidth]{images/home.png}
	\caption{Schermata principale}
\label{fig:home}
\end{figure}

Ogni pulsante svolge la funzione brevemente descritta sopra di esso. Una volta selezionata una pagina i pulsanti appaiono in alto nella nuova pagina, come mostrato in figura \ref{fig:ins_gr}. È sempre possibile tornare alla pagina principale premendo il tasto "Home".

%selezione files
%selezione dati da DB
%selezione grafico
\subsection{Operazioni sui grafici}
\subsubsection{Inserimento}
%selezione files
%selezione dati da DB
%selezione grafico
%selezione opzioni
È possibile importare un dataset dalla memoria locale o dal \glo{database}, scegliendo tra i due pulsanti mostrati in figura \ref{fig:ins_gr}.
\begin{itemize}
	\item[\textbf{1.a)}] Cliccare sul pulsante "IMPORTA" e scegliere il file .\glo{csv} o .\glo{tsv} da cui prelevare i dati. 
	Per ulteriori informazioni su caratteristiche obbligatorie dei file .\glo{csv} o .\glo{tsv} si rimanda a §\ref{sub:caricamento_dati}.
\end{itemize}
Oppure
\begin{itemize}
	\item[\textbf{1.b)}] Cliccare sul pulsante "DATABASE". Nella finestra che si aprirà saranno presenti i seguenti due menù a tendina:
	\begin{itemize}
		\item \textbf{Tabelle}: in cui è possibile selezionare uno dei dataset presenti nel \glo{database};
		\item \textbf{Colonne}: in cui è possibile selezionare le dimensioni del dataset selezionato che si vogliono importare.
	\end{itemize}
	Per ulteriori dettagli sull'importazione di dataset dal \glo{database} si rimanda a §\ref{sub:importazione_db}.
\end{itemize}

Una volta importato un dataset, è possibile selezionare una visualizzazione ed eventualmente impostare le opzioni di manipolazione dati disponibili, attraverso l'interfaccia di figura \ref{fig:ins_gr}:
\begin{itemize}
	% \setlength\itemsep{1pt}
	\item[\textbf{2)}] Selezionare una o più colonne da visualizzare;
	\item[\textbf{3)}] Selezionare una o nessuna colonna per il raggruppamento;
	\item[\textbf{4)}] Selezionare una tra le possibili visualizzazioni;
	\item[\textbf{5)}] Scegliere se applicare o meno la normalizzazione ai dati (si veda §\ref{subsec:nrml} - Normalizzazione);
	\item[\textbf{6)}] Selezionare le opzioni per la manipolazione dati. Tali opzioni dipendono dal grafico scelto al punto 4. La gamma di opzioni presenti per ogni grafico è riassunta nella tabella \ref{tab:manip_graf}.
	\item[\textbf{7)}] Premere il pulsante "CONFERMA" per aggiungere il grafico alla pagina di visualizzazione ed essere rimandati a tale pagina.
\end{itemize}
\begin{figure}[htpb]
	\begin{subfigure}[c]{\linewidth}
	\includegraphics[width=\linewidth]{images/inserimento.png}
	\end{subfigure}
\caption{Inserimento di un grafico}
\label{fig:ins_gr}
\end{figure}
\begin{floattable}{Manipolazioni possibili per ogni grafico}{manip_graf}{R{4cm}|P{2.5cm}|c|c|P{3,5cm}}
\topline
 & \textbf{Scatter plot Matrix} & \textbf{Force Field} & \textbf{Heat Map} & \textbf{Proiezione Lineare Multi Asse} \\ \capsep
Normalizzazione di dati & Sì & Sì & Sì & Sì \\ \rowsep
Visualizzazione di una riduzione dimensionale & Sì & No & No & Sì \\ \rowsep
Visualizzazione di una matrice delle distanze & No & Sì & Sì & No \\ \rowsep
Visualizzazione di dati non manipolati & Sì & No & No & Sì \\
\end{floattable}

\newpage
\subsubsection{Modifiche al grafico}
\paragraph{Modifica titolo}
\begin{itemize}
	\item[\textbf{1)}] Aprire il pannello delle opzioni di un grafico;
	\item[\textbf{2)}] Cliccare sul pulsante con l'icona di modifica (figura \ref{fig:name_change}.a);
	\item[\textbf{3)}] Inserire il nome desiderato per il grafico;
	\item[\textbf{4)}] Premere l'icona di conferma (figura \ref{fig:name_change}.b).
\end{itemize}
A questo punto il nuovo titolo viene visualizzato. È possibile modificare nuovamente il titolo.
\begin{figure}[htpb]
	\begin{subfigure}[b]{.5\linewidth}
	\includegraphics[width=\linewidth]{images/before_name.png}
	\caption{Nome di un grafico}\label{fig:before_name}
	\end{subfigure}
	\begin{subfigure}[b]{.5\linewidth}
	\includegraphics[width=\linewidth]{images/after_name.png}
	\caption{Modifica del nome}\label{fig:after_name}
	\end{subfigure}
\caption{Modifica al nome di un grafico}
\label{fig:name_change}
\end{figure}

\paragraph{Interazioni con il grafico}

Uno dei punti di forza di \textit{HD Viz} è la presenza di grafici \textit{dinamici}, che rispondono alle interazioni con l'utente.
Le opzioni dipendono dallo specifico tipo di grafico, pertanto si rimanda alla sezione §\ref{sub:grafici} che illustra in dettaglio le caratteristiche e le funzionalità presenti per ogni grafico. Un riassunto delle interazioni possibili è il seguente:
\begin{floattable}{Interazioni possibili per ogni grafico}{Interazioni possibili per ogni grafico}{R{4cm}|P{2.5cm}|c|c|P{3,5cm}}
\topline
 & \textbf{Scatter plot Matrix} & \textbf{Force Field} & \textbf{Heat Map} & \textbf{Proiezione Lineare Multi Asse} \\ \capsep
Evidenziamento di dati & Sì & No & No & No \\ \rowsep
Aggiunta dimensione & Sì & No & No & Sì \\ \rowsep
Rimozione dimensione & Sì & No & No & Sì \\ \rowsep
Ordinamento elementi & No & No & Sì & No \\ \rowsep
Trascinamento nodo & No & Sì & No & No \\ \rowsep
Modifica intensità della forza applicata & No & Sì & No & No \\ \rowsep
Modifica intervallo di distanze visualizzate & No & Sì & Sì & No \\
\end{floattable}
Il procedimento per l'applicazione di modifiche è identico per tutti i grafici:
\begin{itemize}
	\item[\textbf{1)}] Aprire il pannello delle opzioni di un grafico;
	\item[\textbf{2)}] Modificare una o più opzioni tra quelle disponibili;
	\item[\textbf{3)}] Premere il pulsante di conferma.
\end{itemize}
\subsubsection{Rimozione di un grafico}

È possibile rimuovere un grafico dalla lista dei grafici visualizzati:
\begin{itemize}
	\item[\textbf{1)}] Aprire il pannello delle opzioni di un grafico;
	\item[\textbf{2)}] Premere il pulsante "RIMUOVI".
\end{itemize}

\begin{figure}[htpb]
	\includegraphics[width=\linewidth]{images/visualizations.png}
	\caption{Modifiche ai grafici}
\label{fig:modifiche}
\end{figure}

\subsection{Visualizzazione manuale e guida utente}

Per visualizzare il \textit{Manuale Utente} (il presente documento) selezionare dal menù in alto la voce "Manuale Utente".\\
\begin{figure}[htpb]
	\includegraphics[width=\linewidth]{images/manuale.png}
	\caption{Visualizzazione Manuale Utente}
\label{fig:manutente}
\end{figure}
Per visualizzare la Guida Utente selezionare dal menù principale la voce "Guida".\\
\begin{figure}[htpb]
	\includegraphics[width=\linewidth]{images/guida.png}
	\caption{Visualizzazione Guida Utente}
\label{fig:guidautente}
\end{figure}

