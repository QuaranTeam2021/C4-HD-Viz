\subsection{Caricamento dati}
\label{sub:caricamento_dati}
L'utente preme sul pulsante importa file e seleziona un file con estensione \glo{\textit{.csv}} o \glo{\textit{.tsv}}. \\
Questo file deve contenere \textit{obbligatoriamente}:
\begin{itemize}
	\item \textbf{una e una sola} riga di intestazione;
	\item \textbf{una o più} righe di dati.
\end{itemize}
Un esempio di file \glo{\textit{.csv}} ben formato è il seguente:

\begin{lstlisting}[caption={formattazione corretta CSV}]
sepal_length,sepal_width,petal_length,petal_width,species
5.1,3.5,1.4,0.2,setosa
4.9,3.0,1.4,0.2,setosa
4.7,3.2,1.3,0.2,setosa
4.6,3.1,1.5,0.2,setosa
5.0,3.6,1.4,0.2,setosa
\end{lstlisting}
Un esempio di file \glo{\textit{.tsv}} ben formato è il seguente:

\begin{lstlisting}[caption={formattazione corretta TSV}]
sepal_length	sepal_width	petal_length	petal_width	species
5.1	3.5	1.4	0.2	setosa
4.9	3.0	1.4	0.2	setosa
4.7	3.2	1.3	0.2	setosa
4.6	3.1	1.5	0.2	setosa
5.0	3.6	1.4	0.2	setosa
\end{lstlisting}
Il contenuto del file in input deve sottostare alle seguenti regole:

\begin{itemize}
	\item il carattere delimitatore per un file \glo{\textit{.csv}} deve essere una virgola '\textbf{,}' o un punto e virgola '\textbf{;}' ;
	\item il carattere delimitatore per un file \glo{\textit{.tsv}} deve essere una tabulazione;
	\item il delimitatore per le cifre decimali deve essere un punto;
	\item non deve essere presente un delimitatore per raggruppare le migliaia. Ciò implica che:
	\begin{itemize}
		\item 10000 è accettato e considerato come $10^4$,
		\item 10.000 è accettato e considerato come $10^1$,
		\item 10'000 non è accettato,
		\item 10,000 viene interpretato come il contenuto di due celle contigue;
	\end{itemize}
	\item ogni colonna deve contenere dati dello stesso tipo;
	\item il formato di codifica accettato è UTF-8.
\end{itemize}
Una volta importati i dati l'utente può selezionare tra le colonne numeriche quelle da considerare per la visualizzazione senza manipolazioni, per la \glo{riduzione dimensionale} o per il calcolo della \glo{matrice delle distanze}. \\
In seguito è possibile selezionare la colonna (numerica o non numerica) per il raggruppamento. \\
È possibile validare ed eventualmente correggere un file \glo{\textit{.csv}} all'indirizzo
\begin{center}
	\url{http://www.fixcsv.com/}
\end{center}