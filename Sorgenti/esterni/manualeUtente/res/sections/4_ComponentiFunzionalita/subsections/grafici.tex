\subsection{Funzionalità dei grafici}
\label{sub:grafici}
	%TODO chiarire la conferma
\subsubsection{Force Field}
Il grafico Force Field (figura \ref{fig:ffparams}) permette di visualizzare un integratore numerico di velocità \glo{Verlet} per la simulazione di forze fisiche sulle particelle, utile per lo studio di reti.

\paragraph{Modalità di interazione}
È possibile \textit{trascinare} e \textit{fissare} il nodo in un punto dello schermo, \textbf{mantenendo premuto} il cursore sopra un nodo e \textbf{spostando} il nodo nel punto desiderato. Al termine dell'operazione il bordo del nodo assume il colore nero. \\
È possibile \textit{rilasciare} un nodo precedentemente fissato, \textbf{cliccando} sul nodo stesso. Al termine dell'operazione il bordo del nodo assume il colore bianco. \\
È possibile \textit{visualizzare informazioni} associate a un nodo, \textbf{posizionando il cursore} sopra il nodo stesso.
	%TODO sostituire
\paragraph{Parametri della visualizzazione}
Attraverso il pannello di controllo della visualizzazione si possono modificare i seguenti parametri:
\begin{itemize}
	\item \textbf{Intensità della \glo{forza}}: l'utente può impostare l'attrazione reciproca esercitata dai nodi. Se impostata ad un valore positivo i nodi si attraggono tra loro, se impostata ad un valore negativo i nodi si respingono. Il suo valore di default è $-30$, l'intervallo chiuso nel quale può spaziare è $[-150,50]$.
	% Viene modificata attraverso 
	%TODO aggiornare
	% lo slider #1 in figura;
	\item \textbf{\glo{Distanza} massima}: \glo{distanza} massima tra i nodi affinché l'arco che collega la coppia di nodi venga inserito nella simulazione di forza e venga visualizzato. Il suo valore di default è pari alla distanza massima tra coppie di nodi della grafo. 
	% Viene modificata attraverso 
	%TODO aggiornare{}
	% lo slider #3 in figura;
	\item \textbf{\glo{Distanza} minima}: \glo{distanza} minima tra i nodi affinché l'arco che collega la coppia di nodi venga inserito nella simulazione di forza e venga visualizzato. Il suo valore di default è pari alla distanza minima tra coppie di nodi della grafo. 
	% Viene modificata attraverso 
	%TODO aggiornare
	% lo slider #2 in figura;
\end{itemize}
	%TODO sostituire
\begin{figure}[htpb]
	\centering
	\begin{subfigure}[b]{.31\linewidth}
	\includegraphics[width=\linewidth]{images/ff_originale.png}
	\caption{Force Field}\label{fig:ff_originale}
	\end{subfigure}
	\begin{subfigure}[b]{.31\linewidth}
	\includegraphics[width=\linewidth]{images/ff_threshold.png}
	\caption{Distanze modificate}\label{fig:ff_threshold}
	\end{subfigure}
	\begin{subfigure}[b]{.31\linewidth}
	\includegraphics[width=\linewidth]{images/strengthAugmeted.png}
	\caption{Forza aumentata e nodo trascinato}\label{fig:strengthAugmeted}
	\end{subfigure}
\caption{Modifica parametri Force Field}
\label{fig:ffparams}
\end{figure}

\subsubsection{Heat Map}
Il grafico Heat Map (figura \ref{fig:hmparams}) permette di visualizzare una matrice bidimensionale delle distanze, contenente rettangoli colorati. \\
Ogni rettangolino posizionato \textit{al di fuori} della \glo{diagonale} rappresenta un arco di un \glo{grafo}. La riga e la colonna di appartenenza indicano ciascuna un nodo che è estremo dell'arco, il cui nome è riportato sopra la riga o la colonna. Ogni rettangolino ha un colore tendente al viola, di opacità proporzionale all'intensità del collegamento tra i nodi rappresentanti riga e colonna.\\
Ogni rettangolino posizionato \textit{nella} \glo{diagonale} \textit{non} rappresenta un arco di un \glo{grafo}, bensì il nodo rappresentante la riga e la colonna (che hanno lo stesso indice, trattandosi di elementi sulla diagonale). È riempito con un colore opaco diverso dal viola, che rappresenta il gruppo di appartenenza del nodo.\\
Ciascun collegamento tra nodi è presente due volte nel grafico Heat Map: una volta a destra della \glo{diagonale}, un'altra volta a sinistra della \glo{diagonale}.
	%TODO sostituire
\paragraph{Modalità di interazione}
\textit{Posizionando} il cursore \textit{sopra} un rettangolino è possibile \textbf{evidenziare} a lato del grafico il nome dei nodi corrispondenti agli estremi dell'arco rappresentato dal rettangolino.\\
\textit{Posizionando} il cursore \textit{sopra} il nome di una riga o di una colonne è possibile \textbf{visualizzare informazioni} sul nodo.\\


\paragraph{Parametri della visualizzazione}
Attraverso il pannello di controllo della visualizzazione si possono modificare i seguenti parametri:
\begin{itemize}
\item \textbf{Ordinamento}: è possibile \textit{ordinare} gli elementi della matrice, selezionando la modalità di ordinamento dall'apposito menu a tendina. \\
	Le opzioni selezionabili sono:
	\begin{itemize}
		\item \textbf{ordinamento per \glo{cluster}};
		\item \textbf{ordinamento originale}.
	\end{itemize}
	\item \textbf{\glo{Distanza} massima}: \glo{distanza} massima tra i nodi affinché venga visualizzato nel grafico il rettangolino corrispondente all'arco che collega la coppia di nodi. Il suo valore di default è pari alla distanza massima tra coppie di nodi della grafo. 
	% Viene modificata attraverso 
	%TODO aggiornare
	% lo slider #2 in figura;
	\item \textbf{\glo{Distanza} minima}: \glo{distanza} minima tra i nodi affinché venga visualizzato nel grafico il rettangolino corrispondente all'arco che collega la coppia di nodi. Il suo valore di default è pari alla distanza minima tra coppie di nodi della grafo. 
	% Viene modificata attraverso 
	%TODO aggiornare
	% lo slider #1 in figura;
\end{itemize}

	%TODO sostituire
\begin{figure}[htpb]
	\begin{subfigure}[b]{.32\linewidth}
	\includegraphics[width=\linewidth]{images/hm_original.png}
	\caption{Originale}\label{fig:hm_originale}
	\end{subfigure}
	\begin{subfigure}[b]{.32\linewidth}
	\includegraphics[width=\linewidth]{images/hm_threshold.png}
	\caption{Distanze modificate}\label{fig:hm_threshold}
	\end{subfigure}
	\begin{subfigure}[b]{.32\linewidth}
	\includegraphics[width=\linewidth]{images/hm_clusters.png}
	\caption{Ord. per cluster}\label{fig:hm_clusters}
	\end{subfigure}
\caption{Modifica parametri Heat Map}
\label{fig:hmparams}
\end{figure}

\subsubsection{Scatter plot Matrix}
Il grafico Scatter plot Matrix (figura \ref{fig:scptinter}) permette di visualizzare una griglia di \glo{Scatter plot}, ciascuno avente come assi una coppia di dimensioni del dataset. 
Ciò significa che, preso un dataset con $k$ dimensioni, lo Scatter plot Matrix avrà $k$ righe e $k$ colonne, e il grafico alla $i$-esima riga e alla $j$-esima colonna è uno \glo{Scatter plot} con la $i$-esima dimensione nell'asse delle ascisse e la $j$-esima dimensione nell'asse delle ordinate. \\
È possibile visualizzare fino ad un massimo di 5 dimensioni nello Scatter plot Matrix.\\
\textbf{N.B.} Ogni coppia di dimensioni distinte appare in due diversi \glo{Scatter plot} della matrice, in posizioni speculari rispetto alla diagonale, con assi delle ascisse e delle ordinate invertiti.

\paragraph{Modalità di interazione}
L'operazione di \textit{brushing} permette di selezionare un insieme di punti in uno \glo{Scatter plot} e di vedere evidenziati quei punti in tutti i restanti \glo{Scatter plot} della matrice. \\
È possibile \textit{effettuare il \glo{brush}} dei nodi \textbf{selezionando un area rettangolare} in uno \glo{Scatter plot} della griglia. In seguito a questa operazione in ogni \glo{Scatter plot} sono distinguibili:
\begin{itemize}
	\item i punti selezionati, nelle dimensioni e nei colori originari; 
	\item i punti non selezionati, in dimensioni nettamente ridotte e in colore nero.
\end{itemize}
È possibile \textit{spostare} il rettangolo di selezione precedentemente disegnato, \textbf{trascinandolo} con il mouse nella posizione desiderata. L'evidenziamento dei dati nei restanti Scatter plot viene aggiornato in tempo reale. \\
È possibile \textit{annullare il \glo{brush}} della selezione precedentemente creata, \textbf{cliccando in un punto} del grafico \textbf{al di fuori} del rettangolo di selezione. In seguito a questa operazione tutti i punti vengono visualizzati nella loro forma originaria e non sono più visibili aree di selezione nel grafico. \\
È possibile \textit{aggiungere o rimuovere} una dimensione selezionandola dal pannello in parte al grafico. Ad ogni aggiunta o rimozione di una dimensione l'eventuale \glo{brush} applicato viene rimosso.
\begin{figure}[htpb]
\centering
	\begin{subfigure}[c]{.4\linewidth}
	\includegraphics[width=\linewidth]{images/scpt_brush.png}
	\caption{fig. a con brush}\label{fig:brush}
	\end{subfigure}
	\begin{subfigure}[c]{.4\linewidth}
	\includegraphics[width=\linewidth]{images/scpt_colonne.png}
	\caption{fig. a con selezione colonne}\label{fig:scpt_colonne}
	\end{subfigure}
\caption{Interazione con Scatter plot Matrix}
\label{fig:scptinter}
\end{figure}
\subsubsection{Proiezione Lineare Multi Asse}
La Proiezione Lineare Multi Asse (figura \ref{fig:malpinter}) è una variante dello \glo{Scatter plot} tradizionale, nella quale gli assi sono visibili e trascinabili. \\
Il trascinamento di un asse mostra a schermo una differente \glo{proiezione} bi-dimensionale del dataset \glo{multi-dimensionale} originario. \\
L'asse delle ascisse e delle ordinate rappresentano le componenti principali 1 e 2.
Una griglia posizionata in secondo piano rispetto al grafico aiuta a dedurre le coordinate dei punti rispetto alle due componenti principali.
L'\glo{origine} è rappresentata dall'intersezione degli assi.

\paragraph{Modalità di interazione}

È possibile \textit{visualizzare una diversa \glo{proiezione}} del dataset originario \textbf{trascinando} l'estremità di un asse in un altro punto del grafico avente \glo{distanza} uguale (o simile) dall'\glo{origine}. \\
È possibile \textit{comprimere o allungare} un asse \textbf{trascinando} l'estremità dell'asse in un altro punto del grafico più vicino o lontano dall'\glo{origine}, senza effettuare rotazioni. \\
Un'operazione di trascinamento può produrre uno solo o entrambi gli effetti descritti sopra, poiché l'utente non è limitato a muovere un'estremità di un asse lungo una stessa orbita circolare o lungo la stessa direzione più o meno lontana dall'\glo{origine}. \\
È possibile \textit{aggiungere o rimuovere} un asse selezionandolo dal pannello in parte al grafico. Ad ogni aggiunta o rimozione di un asse la \glo{proiezione} viene ricalcolata, dunque gli assi già presenti o rimasti potrebbero assumere una posizione differente.


\begin{figure}[htpb]
	\begin{subfigure}[b]{.32\linewidth}
	\includegraphics[width=\linewidth]{images/malp_originale.png}
	\caption{Originale}\label{fig:malp_originale}
	\end{subfigure}
	\begin{subfigure}[b]{.32\linewidth}
	\includegraphics[width=\linewidth]{images/malp_colonne.png}
	\caption{fig. a con selezione colonne}\label{fig:malp_colonne}
	\end{subfigure}
	\begin{subfigure}[b]{.32\linewidth}
	\includegraphics[width=\linewidth]{images/malp_ruotato.png}
	\caption{fig. b con rotazione asse}\label{fig:malp_ruotato}
	\end{subfigure}
\caption{Interazione con Proiezione Lineare Multi Asse}
\label{fig:malpinter}
\end{figure}

\subsection{Complementi ai grafici}
\subsubsection{Legenda}
Al fine di semplificare l'attività esplorativa sui dati ogni grafico possiede una propria legenda, visibile di default e nascondibile. \\
Per ogni tipologia di grafico la legenda mostra le \textit{categorie} associate ai dati visualizzati nel grafico.\\ Nei grafici \textit{Force Field} e \textit{Heat Map} vengono mostrate inoltre la \textit{scala della distanza} adottata, il \textit{numero di archi} del grafo \textit{compresi tra il valore della distanza minima e massima} e, se sono presenti archi con valore della distanza nullo, il \textit{numero di archi} del grafo aventi valore della \textit{distanza nullo} (una funzione di calcolo delle distanze può produrre valori nulli, si veda §\ref{subsec:distcalc} - Scelta di una funzione per il calcolo delle distanze).\\
Nei grafici \textit{Scatter plot Matrix} e \textit{Proiezione Lineare Multi Asse} viene visualizzato il numero di elementi del dataset non mostrati poiché contenenti valori nulli (una funzione di calcolo delle distanze può produrre valori nulli, si veda §\ref{subsec:dimred} - Scelta di una funzione di riduzione dimensionale).\\
