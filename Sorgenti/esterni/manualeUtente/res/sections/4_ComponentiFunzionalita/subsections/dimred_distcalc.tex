\subsection{Scelta di una \glo{riduzione dimensionale}}
\label{subsec:dimred}
Dopo aver scelto una visualizzazione Scatter plot Matrix o Proiezione Lineare Multi Asse, è possibile selezionare \textbf{una sola o nessuna} delle seguenti \glo{riduzioni dimensionali} e le relative opzioni:
\begin{itemize}
	\item FASTMAP, con opzioni:
	\begin{itemize}
		\item numero di dimensioni in output;
		\item funzione di calcolo della \glo{distanza} associata;
	\end{itemize}
	\item ISOMAP, con opzioni:
	\begin{itemize}
		\item numero di dimensioni in output;
		\item funzione di calcolo della \glo{distanza} associata;
		\item parametro \textit{neighbors};
	\end{itemize}
	\item t-SNE, con opzioni:
	\begin{itemize}
		\item numero di dimensioni in output;
		\item funzione di calcolo della \glo{distanza} associata;
		\item parametro \textit{neighbors};
		\item parametro \textit{perlexity};
		\item parametro \textit{epsilon};
	\end{itemize}
	\item LLE, con opzioni:
	\begin{itemize}
		\item numero di dimensioni in output;
		\item funzione di calcolo della \glo{distanza} associata;
		\item parametro \textit{neighbors};
	\end{itemize}
	\item UMAP, con opzioni:
	\begin{itemize}
		\item numero di dimensioni in output;
		\item parametro \textit{neighbors}.
	\end{itemize}
\end{itemize}
Le opzioni possibili sono riassunte nella tabella in seguito:
\begin{floattable}{Opzioni disponibili per ogni riduzione dimensionale}{Opzioni disponibili per ogni riduzione dimensionale}{R{5.8cm}|c|c|c|c|c}
\topline
 & \textbf{FASTMAP} & \textbf{ISOMAP} & \textbf{t-SNE} & \textbf{LLE} & \textbf{UMAP} \\ \capsep
Numero di dimensioni in output & Sì & Sì & Sì & Sì & Sì \\ \rowsep
Funzione di calcolo della distanza & Sì & Sì & Sì & Sì & No \\ \rowsep
Parametro \textit{neighbors} & No & Sì & Sì & Sì & Sì \\ \rowsep
Parametro \textit{perplexity} & No & No & Sì & No & No \\ \rowsep
Parametro \textit{epsilon} & No & No & Sì & No & No \\ 
\end{floattable}
Le varie opzioni appaiono e scompaiono a video dinamicamente in seguito alla selezione della \glo{riduzione dimensionale}. Tutte le opzioni sono selezionabili tramite \glo{slider} numerico, ad eccezione della scelta della funzione di calcolo della \glo{distanza} associata, selezionabile tramite lo stesso widget utilizzato per la scelta di una funzione per il calcolo delle \glo{distanze} utilizzato per il calcolo della \glo{matrice delle distanze}. \\
È consentito selezionare un numero di dimensioni in output minore, uguale o maggiore del numero di dimensioni in input. Alcune riduzioni dimensionali possono generare dei valori nulli in presenza di particolari configurazioni del dataset.\\
Per una trattazione approfondita dei vari algoritmi di \glo{riduzione dimensionale} si rimanda a §\ref{sub:riferimenti} - Riferimenti.

\subsection{Scelta di una funzione per il calcolo delle distanze}
\label{subsec:distcalc}
Dopo aver selezionato una visualizzazione Force Field o Heat Map è necessario scegliere \textbf{una e una sola} tra le funzioni disponibili per il calcolo delle \glo{distanze}. \\
Le opzioni possibili sono le seguenti:
\begin{itemize}
	\item \glo{distanza} Euclidea;
	\item \glo{distanza} di Manhattan;
	\item \glo{distanza} Cosine;
	\item \glo{distanza} di Chebyshev;
	\item \glo{distanza} Euclidea quadrata;
	\item \glo{distanza} di Canberra.
\end{itemize}
Sono selezionabili tramite menu a tendina. \\
Il calcolo della distanza può produrre valori non numerici in presenza di alcune configurazioni del dataset (es. il calcolo della distanza di Canberra comporta una divisione con denominatore potenzialmente uguale a 0). In caso vengano prodotti valori non validi questi non vengono mostrati nella visualizzazione, e viene mostrato un avviso nella legenda del grafico contenente il numero di valori non numerici individuati. \\
Per una trattazione approfondita delle varie funzioni di calcolo della \glo{distanza} si rimanda a §\ref{sub:riferimenti} - Riferimenti
%TODO aggiornare
% in figura

\subsection{Selezione della normalizzazione}
\label{subsec:nrml}
HD Viz permette di scegliere se applicare o meno l'operazione di \textit{normalizzazione} sui dati, prima di applicare un'eventuale manipolazione su questi. L'operazione di \textit{normalizzazione} individua e applica una funzione per ogni colonna del dataset, avente come dominio l'insieme dei valori della colonna del dataset e come codominio l'intervallo chiuso $[0,1]$. Il valore più alto del dominio (che chiamiamo $m$) viene associato a 1, l'immagine ($x_i$) di ogni altro valore ($x_d$) del dominio assume il valore ($\frac{x_d}{m}$).

Tale operazione serve ad equiparare il contributo delle colonne del dataset nella visualizzazione, a prescindere dall'ampiezza del dominio in cui i valori sono distribuiti. La \textit{normalizzazione} può essere desiderabile o meno.