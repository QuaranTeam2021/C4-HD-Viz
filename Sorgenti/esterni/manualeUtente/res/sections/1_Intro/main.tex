\section{Introduzione}

\subsection{Scopo del documento}
Il presente documento rappresenta il manuale utente per l’applicazione web \textit{HD Viz}. 
Vengono descritte dettagliatamente tutte le caratteristiche dell'applicativo utilizzabili dall'utente. \\
Tale manuale è suddiviso in sezioni:
\begin{itemize} 
	\item in §\ref{sec:installazione} vengono riportate le istruzioni per l'installazione;
	\item in §\ref{sec:utilizzo} vengono descritte le operazioni principali eseguibili dall'utente di \textit{HD Viz};
	\item in §\ref{sec:componenti} il focus è rivolto verso le componenti visibili dell'applicazione, quindi vengono descritte le loro caratteristiche, le loro funzionalità e le loro modalità di interazione.
\end{itemize} 
Il manuale sarà consultabile alla sezione "Aiuto" accessibile dal menu della pagina principale di \textit{HD Viz}.

\subsection{Scopo del prodotto}
\textit{HD Viz} è un'applicazione web avente scopo di fornire uno strumento per la visualizzazione di dati con molte dimensioni a supporto della fase esplorativa dell'analisi dei dati.
\textit{HD Viz} è in grado di rappresentare dati che possono avere almeno 15 dimensioni e fornisce 4 diversi tipi di visualizzazione a tale scopo.

\subsection{Glossario}

Il documento contiene termini che possono presentare significati ambigui. Viene quindi fornito un glossario individuabile nell'appendice §\ref{sec:glossario}, all'interno del documento, contenente tutti i termini definiti ambigui e la loro spiegazione. Nel documento vengono identificati con una G a pedice.

\subsection{Riferimenti}
\label{sub:riferimenti}
Per una trattazione più approfondita di funzioni matematiche, algoritmi e grafici sono state selezionate le seguenti fonti:
\begin{itemize}
	\item \textbf{Funzioni di \glo{riduzione dimensionale}}:
	\begin{itemize}
		\item FASTMAP:\\
		\url{https://dl.acm.org/doi/10.1145/568271.223812}
		\item ISOMAP:\\
		\url{https://science.sciencemag.org/content/290/5500/2319}
		\item t-SNE:\\
		\url{https://www.jmlr.org/papers/v9/vandermaaten08a.html}
		\item LLE:\\
		\url{https://science.sciencemag.org/content/290/5500/2323}
		\item UMAP:\\
		\url{https://arxiv.org/abs/1802.03426}
	\end{itemize}

	\item \textbf{Funzioni di calcolo delle distanze}:
	\begin{itemize}
		\item \glo{Distanza} di Canberra: \\
			\url{https://en.wikipedia.org/wiki/Canberra_distance}
		\item \glo{Distanza} Euclidea e Euclidea Quadrata:\\
			\url{https://xlinux.nist.gov/dads/HTML/euclidndstnc.html}
		\item \glo{Distanza} di Chebyshev:\\
			\url{https://en.wikipedia.org/wiki/Chebyshev_distance}
		\item \glo{Distanza} Cosine:\\
			\url{https://www.itl.nist.gov/div898/software/dataplot/refman2/auxillar/cosdist.htm}
		\item \glo{Distanza} di Manhattan:\\
			 \url{https://xlinux.nist.gov/dads/HTML/manhattanDistance.html}
	\end{itemize}

	\item \textbf{Grafici}:
	\begin{itemize}
		\item Force Field:\\
			\url{https://observablehq.com/@d3/force-directed-graph}
		\item Heat Map:\\
			\url{https://observablehq.com/@bstaats/matrix-diagram}
		\item Scatter plot Matrix:\\
			\url{https://observablehq.com/@d3/scatterplot-matrix}
		\item Proiezione Lineare Multi Asse:\\
			\url{https://orange3.readthedocs.io/projects/orange-visual-programming/en/latest/widgets/visualize/freeviz.html}
	\end{itemize}
\end{itemize}


\subsection{Supporto tecnico}

Qualora venissero riscontrati \glo{bug} o malfunzionamenti, si prega di segnalarli per mail all'indirizzo:
\begin{center}
	\textcolor{blue}{\href{mailto:quaranteam2021@gmail.com}{quaranteam2021@gmail.com}}
\end{center}
È preferibile l'utilizzo della seguente struttura:
\vspace{10pt}\\
\textbf{Oggetto}: HD Viz - bug - $\langle$ Evento da segnalare $\rangle$;\\
\textbf{Allegato} (opzionale): è possibile allegare immagini o video esplicativi;\\
\textbf{Corpo}: 
\begin{itemize}
	\item \glo{browser} e sistema operativo utilizzati;
	\item descrizione del problema;
	\item versione di HD Viz utilizzata.
\end{itemize}