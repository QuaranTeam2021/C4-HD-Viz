\section{Diagrammi delle classi}

% inserire tutte il codice della sezione in questo file ed eliminare cartella subsections
% OPPURE inserire più file nella sottocartella subsections e includere i file 
% come mostrato sotto:

% es: per includere 2 files chiamati 
% nomefile1.tex, nomefile2.tex
% \subimport{subsections/}{nomefile1.tex}
% \subimport{subsections/}{nomefile2.tex}

% oppure
% \subsectionInFile{nomefile1.tex}
% \subsectionInFile{nomefile2.tex}

\begin{figure}[H]
	\begin{center}
		\includegraphics[width=1.0\textwidth]{img/Progettazione.png} \\
		\caption{Diagramma delle classi}
	\end{center}
\end{figure}

\subsection{Design pattern adottati}
Come citato nella sezione §2 è stata utilizzata la libreria MobX per l'implementazione del pattern \textit{observer} tra lo \texttt{Store} e le componenti React.
L'implementazione degli algoritmi di riduzione è stata effettuata utilizzando il \textit{Template method pattern}. La classe \texttt{Algorithm} è astratta e definisce i passi per eseguire la riduzione all'interno del metodo \texttt{compute}, nel cui corpo viene richiamato il metodo \texttt{setAlgorithm} che è definito nelle sottoclassi concrete. Esiste una sottoclasse concreta per ogni algoritmo di riduzione e ciascuna di essa implementa il metodo \texttt{setAlgorithm}, avente lo scopo di creare l'oggetto della libreria \textit{DruidJS} da utilizzare per il calcolo della riduzione dimensionale.
