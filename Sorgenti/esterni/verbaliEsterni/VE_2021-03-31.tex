\newcommand{\Data}{2021-03-31}
\newcommand{\TitoloDoc}{V.E. del \Data}
\newcommand{\Redattori}{\MAD}
\newcommand{\Verificatori}{\SIM}
\newcommand{\Approvatore}{\GIA}
\newcommand{\Distribuzione}{\Committente{} \\& \Proponente{} \\& \Gruppo{}}
\newcommand{\Uso}{Esterno}
\newcommand{\Stato}{Approvato}
\newcommand{\DescrizioneDoc}{Riassunto della riunione tra il gruppo \textit{QuaranTeam} ed il \glo{proponente} tenutosi il 2021-03-31.}
\newcommand{\pathimg}{../../immagini}
\newcommand{\VersioneDoc}{1.0.0}

% info generali 
\newcommand{\NomeProgetto}{HD Viz}

% fornitore
\newcommand{\Gruppo}{QuaranTeam}
\newcommand{\Mail}{quaranteam2021@gmail.com}

% committenti
\newcommand{\Committente}{\VT{}\\& \CR{}}
\newcommand{\VT}{Prof. Vardanega Tullio}
\newcommand{\CR}{Prof. Cardin Riccardo}

% proponenti
\newcommand{\Proponente}{Zucchetti S.p.A.}
\newcommand{\PG}{Piccoli Gregorio}

% QuaranTeam member
\newcommand{\CHF}{Chiarello Federico}
\newcommand{\COF}{Consalvo Federico}
\newcommand{\GIA}{Gibellato Alice}
\newcommand{\MAD}{Mason Damiano}
\newcommand{\REL}{Rech Elia}
\newcommand{\SIM}{Sinigaglia Matteo}
\newcommand{\VEL}{Veronese Luca}

% ruoli
\newcommand{\Responsabile}{Responsabile di Progetto}
\newcommand{\Amministratore}{Amministratore di Progetto}

% documenti
\newcommand{\SdF}{Studio di Fattibilità}
\newcommand{\SdFv}[1]{\textit{Studio di Fattibilità {#1}}}
\newcommand{\PdQ}{Piano di Qualifica}
\newcommand{\PdQv}[1]{\textit{Piano di Qualifica {#1}}}
\newcommand{\PdP}{Piano di Progetto}
\newcommand{\PdPv}[1]{\textit{Piano di Progetto {#1}}}
\newcommand{\NdP}{Norme di Progetto}
\newcommand{\NdPv}[1]{\textit{Norme di Progetto {#1}}}
\newcommand{\AdR}{Analisi dei Requisiti}
\newcommand{\AdRv}[1]{\textit{Analisi dei Requisiti {#1}}}
\newcommand{\Glossario}{Glossario}
\newcommand{\Glossariov}[1]{\textit{Glossario {#1}}}
\newcommand{\MM}{Manuale Manutentore}
\newcommand{\MMv}[1]{\textit{Manuale Manutentore {#1}}}
\newcommand{\MU}{Manuale Utente}
\newcommand{\MUv}[1]{\textit{Manuale Utente {#1}}}

% comandi generali
\newcommand{\glo}[1]{#1\textsubscript{\textit{G}}}
\newcommand{\qm}[1]{``#1''}

\newcommand{\defaultfooter}[1]{
	\rowcolor{white}
	\multicolumn{#1}{|c|}{\textit{La tabella continua a pagina seguente.}}\\
    \hline
    \endfoot
    \endlastfoot
}


\newcommand{\versionMU}{2.0.0}
\newcommand{\versionMM}{2.0.0}
\newcommand{\versionSdF}{1.0.0}
\newcommand{\versionPdQ}{4.0.0}
\newcommand{\versionPdP}{4.0.0}
\newcommand{\versionNdP}{3.0.0}
\newcommand{\versionAdR}{3.0.0}
\newcommand{\versionGlossario}{4.0.0}

\documentclass{../../Utility/stdDocument}

\begin{document}
\verbaleFronte
		
\section*{Registro delle modifiche}
\begin{RegistroModifiche}
 1.0.0 & 2021-04-03 & Approvazione del documento. & \GIA & \\
\hline
 0.1.0 & 2021-04-01 & Stesura del documento. & \MAD & \SIM \\
\end{RegistroModifiche}

\verbaleIndice

\section{Informazioni generali}
\begin{itemize}
	\item[•] \textbf{Luogo:} \glo{Skype};
	\item[•] \textbf{Data:} 2021-03-31;
	\item[•] \textbf{Ora di inizio:} 12:00;
	\item[•] \textbf{Ora di fine:} 13:00;
	\item[•] \textbf{Partecipanti del gruppo:}
	\begin{enumerate}
		\item [-] Federico Consalvo; 
		\item [-] Alice Gibellato; 
		\item [-] Matteo Sinigaglia;
		\item [-] Luca Veronese;
		\item [-] Damiano Mason.
	\end{enumerate} 
	\item[•] \textbf{Assenti:}
	\begin{enumerate}
		\item [-] Federico Chiarello;
		\item [-] Elia Rech.
	\end{enumerate}
	\item[•] \textbf{Segretario:} Damiano Mason.
\end{itemize}

\section{Ordine del giorno}
\begin{itemize}
	\item [•] Chiarimenti sulla gestione del confronto tra grafici;
	\item [•] Chiarimenti sull'esplorazione dei grafici.
\end{itemize}

\section{Resoconto}
L'incontro è stato proposto dai membri del gruppo per chiarire alcuni dubbi relativi al confronto tra più grafici e all'implementazione di alcune visualizzazioni.
\subsection{Chiarimenti sulla gestione del confronto tra grafici}
Il gruppo ha avanzato una proposta sull'organizzazione visiva della pagina che limitava l'utente nell'inserimento dei grafici.
Il dott. \PG{} ha però fatto presente al gruppo che il confronto tra tipi di grafico uguale è una funzionalità molto utile. Dopodiché ha ulteriormente ribadito l'importanza di lasciare all'utente la libertà di sfruttare una funzionalità appieno, una volta che questa viene implementata, fatti salvo i vincoli tecnici. \\
Il gruppo ha colto il messaggio di pensare bene all'implementazione dei requisiti funzionali, senza dibattiti squisitamente interni sull'effettiva utilità di questi: sarà l'utente stesso a decidere se utilizzare o meno le funzionalità richieste.

\subsection{Chiarimenti sull'implementazione dei grafici}
Per quanto riguarda l'implementazione dei grafici è stato chiesto come implementare il calcolo del dendrogramma, e il dott. \PG{} ha suggerito l'algoritmo \textit{hierarchical clustering}.
È stata avanzata una proposta per l'ampliamento dell'applicazione, accantonata poiché secondo il dott. \PG{} non costituisce valore aggiunto per l'applicazione. 
\section{Riepilogo delle decisioni}
\begin{RiepilogoDecisioni}
	VE\_2021-03-31.1 & Si è stabilito di adottare lo \textit{hierarchical clustering} per il calcolo del \glo{dendrogramma} nel grafico \glo{Heatmap}, qualora il suddetto requisito facoltativo venisse implementato. \\
	\hline
	VE\_2021-03-31.2 & È stata accantonata la proposta di ampliamento dell'applicazione. \\
	\hline
	VE\_2021-03-31.3 & Si è deciso di non introdurre vincoli che limitino l'esperienza d'uso dell'utente nell'inserimento, visualizzazione ed esplorazione dei grafici; fatti salvo i vincoli tecnici. \\
	\hline
\end{RiepilogoDecisioni}
\end{document}