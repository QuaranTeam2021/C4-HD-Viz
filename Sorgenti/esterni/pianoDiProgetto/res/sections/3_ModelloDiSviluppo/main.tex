\section{Modello di sviluppo}

Aderire ad un modello di sviluppo comporta vincoli sulla pianificazione e gestione del corrispondente progetto. La scelta del gruppo è stata quella di adottare il \textbf{modello incrementale}.
\subsection{Modello incrementale}
Il modello di sviluppo incrementale prevede lo sviluppo del prodotto tramite incrementi multipli e successivi, ossia dei rilasci con un incremento di funzionalità, che vengono integrati nel sistema.
I requisiti sono classificati e trattati in base alla loro importanza. I primi incrementi mirano a soddisfare i requisiti più importanti sul piano strategico, gli incrementi successivi coprono invece requisiti meno importanti, che rappresentano però un valore aggiunto al prodotto.
La modifica, l'aggiunta o la cancellazione di requisiti sono consentite solamente previa discussione con il \glo{proponente} e sua approvazione, non sono tuttavia permesse durante la fase di sviluppo dell'incremento corrente.
L'adozione di questo modello comporta i seguenti vantaggi:
\begin{itemize}
	\item le funzionalità principali hanno la priorità nello sviluppo, essendo legate ai requisiti più importanti;
	\item ogni incremento produce valore aggiunto rendendo disponibili delle nuove funzionalità;
	\item è possibile ricevere il feedback da parte del \glo{proponente} più frequentemente, anche in una fase iniziale dello sviluppo;
	\item ad ogni incremento si riducono i rischi legati all'incertezza e si chiariscono meglio i requisiti per gli incrementi successivi;
	\item le attività di verifica e test sono facilitate perché legate all'incremento.
\end{itemize}
Il gruppo \textit{QuaranTeam} ha scelto il modello di sviluppo incrementale poiché:
\begin{itemize}
 \item nel progetto scelto è semplice identificare requisiti di diversa priorità che si possono implementare in rilasci successivi;
 \item il modello incrementale permette di effettuare adeguamenti in corso d'opera con un dispendio di tempo e risorse relativamente basso;
 \item il modello incrementale permette di sottoporre più volte al giudizio del proponente un prodotto che sia sempre funzionante e conforme alle loro aspettative, e nel tempo sempre più completo.
\end{itemize}

\subsection{Incrementi previsti}
I requisiti riportati in tabella comprendono anche tutti i requisiti figli. I requisiti funzionali considerati sono definiti nel documento \AdRv{v\versionAdR}.

  \begin{center}
  \begin{LongTable}{Tabella degli incrementi previsti}{Tabella degli incrementi previsti}{x{.20\hsize}|p{.36\hsize}|x{.36\hsize}}
 \topline
  \rowcolor{headercolour}
  \textbf{Nome}       &   \textbf{Descrizione}  &   \textbf{Requisiti}    \\ \capsep
  \endfirsthead
  \continuedcaption
  \topline
   \rowcolor{headercolour}
  \textbf{Nome}       &   \textbf{Descrizione}  &   \textbf{Requisiti}    \\ \capsep
  \endhead
  \rowsep
	\defaultfooter{3}
	%\rowcolor{violet!25}
%\ohdrx{3}{Progettazione e codifica del Proof of Concept e funzionalità essenziali}\\\hline	
	%\rowcolor{violet!25}
%\ohdrx{3}{}\\\hline	
	Incremento I  &  Importazione dati a partire da un file locale al sistema e visualizzazione basilare dei dati.  &  RF-O-4 \newline RF-O-6  \\	\hline	
	Incremento II  &  Visualizzazione dei grafici ed esplorazione dei dati.  &  RF-O-8 \newline RF-O-9  \\	\hline	
	Incremento III  &  Esplorazione dei dati e gestione degli errori.  &  RF-O-7 \newline RF-D-10 \newline RF-D-11  \\	\hline	
	Incremento IV  &  Interazioni con il database.  &  RF-F-5  \\	\hline	
	Incremento V  &  Salvataggio sessione.  &  RF-O-12 \newline RF-D-13  \\	\hline	
	Incremento VI  &  Istruzioni utente.  &  RF-O-1 \newline  RF-F-2  \newline  RF-F-3  \\	\hline	
  \bottomline
  \end{LongTable}
	\end{center}

















