\subsection{Fase di progettazione e codifica della \glo{Technology Baseline}}
\subsubsection{Incremento I}
\paragraph{Prospetto orario}
Durante il primo incremento la distribuzione oraria è la seguente:
\begin{tabOrariRuolo}{Incremento I - Divisione oraria}{Tabella della divisione oraria}
		{\SIM} & 0 & 4 & 0 & 5 & 2 & 4 & 15 \\
		
		{\GIA} & 0 & 0 & 0 & 5 & 4 & 6 & 15 \\	
		
		{\VEL} & 0 & 4 & 4 & 3 & 4 & 0 & 15 \\
		
		{\MAD} & 0 & 3 & 0 & 5 & 4 & 3 & 15 \\
		
		{\CHF} & 0 & 0 & 3 & 5 & 5 & 2 & 15 \\
		
		{\COF} & 5 & 0 & 5 & 2 & 0 & 3 & 15 \\	
		
		{\REL} & 3 & 0 & 0 & 5 & 3 & 4 & 15 \\ \hline
		
		{\textbf{Ore totali ruolo}} & 8 & 11 & 12 & 30 & 22 & 22 & 105  \\                       
\end{tabOrariRuolo}
I dati ottenuti si possono riassumere nel seguente istogramma:
\begin{figure}[H]
	\centering 
	\includegraphics[width=0.9\textwidth]{../images/ist_inc1.pdf}\\
	\caption{Istogramma della ripartizione ore/ruolo per il primo incremento}
	\label{IstogrammaPrimoIncremento}
\end{figure}
\subsubsection{Costo risultante}
Il costo per ogni ruolo, circoscritto a questo periodo, è il seguente:
\begin{tabCostiRuolo}{Incremento I - Costi per ruolo}{Tabella dei costi per ruolo}
		{Responsabile} & 8 & 240,00 \\
		
		{Amministratore} & 11 & 220,00 \\	
		
		{Analista} & 12 & 300,00 \\
		
		{Progettista} & 30 & 660,00 \\
		
		{Programmatore} & 22 & 330,00 \\
		
		{Verificatore} & 22 & 330,00 \\ \hline
		
		{\textbf{Totale}} & 105 & 2080,00 \\
\end{tabCostiRuolo}
\begin{figure}[H]
	\centering 
	\includegraphics[width=0.7\textwidth]{../images/areo_inc1.pdf}\\
	\caption{Areogramma della ripartizione ore/ruolo per il primo incremento}
	\label{AreogrammaPrimoIncremento}
\end{figure}
\subsubsection{Incremento II}
\paragraph{Prospetto orario}
Durante il secondo incremento la distribuzione oraria è la seguente:
\begin{tabOrariRuolo}{Incremento II - Divisione oraria}{Tabella della divisione oraria}
		{\SIM} & 0 & 3 & 0 & 3 & 4 & 5 & 15 \\
		
		{\GIA} & 0 & 0 & 0 & 4 & 4 & 7 & 15 \\	
		
		{\VEL} & 0 & 4 & 4 & 0 & 3 & 4 & 15 \\
		
		{\MAD} & 0 & 3 & 4 & 2 & 3 & 3 & 15 \\
		
		{\CHF} & 0 & 0 & 3 & 4 & 4 & 4 & 15 \\
		
		{\COF} & 3 & 0 & 3 & 5 & 4 & 0 & 15 \\	
		
		{\REL} & 4 & 0 & 0 & 3 & 3 & 5 & 15 \\ \hline
		
		{\textbf{Ore totali ruolo}} & 7 & 10 & 14 & 21 & 25 & 28 & 105  \\                      
\end{tabOrariRuolo}
I dati ottenuti si possono riassumere nel seguente istogramma:
\begin{figure}[H]
	\centering 
	\includegraphics[width=0.9\textwidth]{../images/ist_inc2.pdf}\\
	\caption{Istogramma della ripartizione ore/ruolo per il secondo incremento}
	\label{IstogrammaSecondoIncremento}
\end{figure}
\subsubsection{Costo risultante}
Il costo per ogni ruolo, circoscritto a questo periodo, è il seguente:
\begin{tabCostiRuolo}{Incremento II - Costi per ruolo}{Tabella dei costi per ruolo}
		{Responsabile} & 7 & 210,00 \\
		
		{Amministratore} & 10 & 200,00 \\	
		
		{Analista} & 14 & 350,00 \\
		
		{Progettista} & 21 & 462,00 \\
		
		{Programmatore} & 25 & 375,00 \\
		
		{Verificatore} & 28 & 420,00 \\ \hline
		
		{\textbf{Totale}} & 105 & 2017,00	\\
\end{tabCostiRuolo}
I dati ottenuti si possono riassumere nel seguente areogramma:
\begin{figure}[H]
	\centering 
	\includegraphics[width=0.7\textwidth]{../images/areo_inc2.pdf}\\
	\caption{Areogramma della ripartizione ore/ruolo per il secondo incremento}
	\label{AreogrammaSecondoIncremento}
\end{figure}