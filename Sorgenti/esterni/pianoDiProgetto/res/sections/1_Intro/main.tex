\section{Introduzione}

% inserire tutte il codice della sezione in questo file ed eliminare cartella subsections
% OPPURE inserire più file nella sottocartella subsections e includere i file 
% come mostrato sotto:

% es: per includere 2 files chiamati 
% nomefile1.tex, nomefile2.tex
% \subimport{subsections/}{nomefile1.tex}
% \subimport{subsections/}{nomefile2.tex}

% oppure
% \subsectionInFile{nomefile1.tex}
% \subsectionInFile{nomefile2.tex}

\subsection{Scopo del documento}
Lo scopo del \textit{\PdP} è gestire le risorse disponibili ed effettuare la ripartizione dei compiti tra i membri del gruppo al fine di perseguire gli obiettivi di efficienza ed efficacia. Vengono inoltre presentate l'analisi dei rischi e una descrizione del modello di sviluppo adottato per il progetto.

\subsection{Scopo del prodotto}
Il \glo{capitolato} C4 ha per obiettivo lo sviluppo di un'applicazione web chiamata \textit{HD Viz}, il cui scopo è fornire una visualizzazione di dati con molte dimensioni a supporto della fase esplorativa dell'analisi dei dati.
\textit{HD Viz} dovrà essere in grado di rappresentare dati che potranno avere almeno fino a 15 dimensioni e fornire minimo 4 diversi tipi di visualizzazione.

\subsection{Glossario}
Viene fornito il \Glossariov{v\versionGlossario{}}, una raccolta di tutti i termini con un significato particolare, che vengono definiti e descritti al fine di evitare ambiguità. I termini definiti nel \Glossariov{v\versionGlossario{}} saranno identificati con una G a pedice. 

\subsection{Riferimenti}

	\subsubsection{Riferimenti normativi}
	\begin{itemize}
		\item \textbf{Norme di Progetto}: \NdPv{v\versionNdP};
		\item \textbf{Regolamento Organigramma e Offerta Tecnico-Economica}: \\
			\url{https://www.math.unipd.it/~tullio/IS-1/2020/Progetto/RO.html};
		\item \textbf{Slide del corso Ingegneria del Software - Regolamento del Progetto Didattico}: \\
			\url{https://www.math.unipd.it/~tullio/IS-1/2020/Dispense/P1.pdf}.
	\end{itemize}
		
	\subsubsection{Riferimenti informativi}
	\begin{itemize}
		\item \textbf{\glo{Capitolato} d'appalto C4 - \textit{HD Viz}}:	 \\
			\url{https://www.math.unipd.it/~tullio/IS-1/2020/Progetto/C4.pdf};
		\item \textbf{Slide del corso di Ingegneria del Software - Gestione di Progetto}: \\
			\url{https://www.math.unipd.it/~tullio/IS-1/2020/Dispense/L06.pdf}.
    \item \textbf{Libro di Testo}: Software Engineering (10th edition) - Ian Sommerville - Pearson Education - Global Edition\\
    Sezioni:
    \begin{itemize}
        \item §2.1.2 (Software Process Model - Incremental Development);
        \item §22.1 (Project Management - Risk Management);
        \item §25.3 (Configuration Management - Change Management).
    \end{itemize}
\end{itemize}

\subsection{Scadenze}
Il gruppo \textit{\Gruppo} ha deciso di consegnare il materiale per le revisioni di avanzamento nelle date:
\begin{itemize}
	\item \textbf{Revisione dei Requisiti}: 11 gennaio 2021;
	\item \textbf{Revisione di Progettazione}: 8 marzo 2021;
	\item \textbf{Revisione di Qualifica}: 12 maggio 2021;
	\item \textbf{Revisione di Accettazione}: 11 giugno 2021.
\end{itemize}
Il gruppo sosterrà le presentazioni nelle date che vengono concordate dal \glo{committente}.