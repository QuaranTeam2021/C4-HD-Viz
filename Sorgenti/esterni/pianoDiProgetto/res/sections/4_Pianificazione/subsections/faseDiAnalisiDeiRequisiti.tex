\subsection{Analisi dei requisiti}
La fase di analisi dei requisiti ha inizio il giorno 2020-10-23, giorno successivo alla formazione dei gruppi del primo lotto, e ha come ordine fissato il giorno 2021-01-11, data della consegna dei documenti in ingresso alla Revisione dei Requisiti del 2021-01-11.

\subsubsection{Ruoli attivi}
Durante questa fase sono attivi i seguenti ruoli:
\begin{itemize}
	\item Responsabile;
	\item Amministratore;
	\item Analista;
	\item Verificatore.
\end{itemize}

\subsubsection{Periodi e attività}
La fase è suddivisa in quattro periodi:

\paragraph{Primo periodo (dal 2020-10-23 al 2020-11-23)}
\begin{itemize}
	\item \textbf{studio dei \glo{capitolati}:} studio preliminare ed esplorazione individuale dei \glo{capitolati}, per maturare una conoscenza dell'argomento tale da poter discutere sulla scelta del \glo{capitolato};
	\item \textbf{studio tecnologie:} ricerca di strumenti e tecnologie per la gestione del progetto e la coordinazione del \glo{team};
	\item \textbf{pianificazione delle attività:} suddivisione dei ruoli e relativi compiti per le attività del periodo;
	\item \textbf{normazione:} definizione delle regole da seguire per i processi di supporto relativi alla documentazione e gestione della configurazione, definizione della struttura delle \textit{\NdP}.
\end{itemize}

\paragraph{Secondo periodo (dal 2020-11-24 al 2020-12-06)}
\begin{itemize}
	\item \textbf{identità del gruppo:} scelta definitiva del nome e del logo del gruppo, creazione della casella di posta elettronica;
	\item \textbf{studio tecnologie:} inizio dello studio individuale delle tecnologie richieste per il \glo{capitolato} scelto;
	\item \textbf{studio di fattibilità:} formalizzazione della scelta del \glo{capitolato} per cui candidarsi con la stesura dello \textit{\SdF{}};
	\item \textbf{pianificazione delle attività:} suddivisione dei ruoli e dei relativi compiti per tutte le attività del periodo;
	\item \textbf{verifica:} controllo di tutti i documenti redatti durante il periodo.
\end{itemize}

\paragraph{Terzo periodo (dal 2020-12-07 al 2020-12-13)}
\begin{itemize}
	\item \textbf{normazione:} scelta delle regole da seguire per processi primari, organizzativi e per processi di supporto non relativi alla documentazione e alla gestione della configurazione (dei quali si è già discusso nel primo periodo);
	\item \textbf{norme di progetto:} stesura del documento \textit{\NdP{}} in base a quanto deciso;
	\item \textbf{glossario:} stesura del documento \textit{\Glossario{}};
	\item \textbf{ricerca:} studio e discussione di tecnologie potenzialmente utili a facilitare lo sviluppo del prodotto;
	\item \textbf{analisi dei requisiti:} estrapolazione e classificazione dei requisiti a partire dal \glo{capitolato} C4;
	\item \textbf{verifica:} controllo della qualità di tutti i documenti sviluppati durante il periodo.
\end{itemize}

\paragraph{Quarto periodo (dal 2020-12-14 al 2021-01-03)}
\begin{itemize}
	\item \textbf{incontro con il proponente:} raccolta di tutti i dubbi, curiosità e chiarimenti riguardo al \glo{capitolato} C4 da sottoporre al proponente e applicazione di quanto traspare dall'incontro;
	\item \textbf{stesura documenti:} conclusione della stesura dei documenti rimanenti per la consegna della Revisone dei Requisiti: \textit{\PdP{}}, \textit{\PdQ{}}, \textit{\AdR{}};
	\item \textbf{pianificazione delle attività:} suddivisione dei ruoli e relativi compiti per le attività del periodo;
	\item \textbf{aggiornamento dei documenti:} aggiornamento da apportare ai documenti esistenti sulla base dell'evoluzione della pianificazione (come norme o ruoli);
	\item \textbf{verifica:} controllo della qualità di tutti i prodotti sviluppati durante il periodo.
\end{itemize}

\paragraph{Quinto periodo (dal 2021-01-04 al 2021-01-11)}
\begin{itemize}
	\item \textbf{preparazione presentazione:} realizzazione della presentazione da esporre in sede di revisione e studio individuale;
	\item \textbf{lettera di presentazione:} stesura lettera di presentazione;
	\item \textbf{revisione e verifica:} revisione finale dei documenti da consegnare alla Revisione dei Requisiti, calcolo delle metriche di qualità e valutazione del risultato in base a quanto stabilito nel \textit{\PdQ{}}, stesura e revisione del consuntivo di periodo (\textit{\PdP{}});
	\item \textbf{revisione e verifica:} Approvazione dei documenti \textit{\PdPv{v1.0.0}}, \textit{\PdQv{v1.0.0}}, \textit{\AdRv{v1.0.0}}, \textit{\NdPv{v1.0.0}}.
\end{itemize}

\begin{figure}[H]
	\centering
	\includegraphics[width=\linewidth]{../images/adr_gantt.pdf}
	\caption{Diagramma di Gantt della fase di analisi dei requisiti.}
\end{figure}