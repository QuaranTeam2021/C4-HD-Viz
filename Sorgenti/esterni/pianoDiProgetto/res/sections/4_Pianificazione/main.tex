\section{Pianificazione}
Per gestire lo sviluppo del progetto secondo il modello incrementale, e rispettare le scadenze, il gruppo ha deciso di suddividere il periodo di tempo in cui si colloca lo sviluppo del prodotto (inteso come stato del suo ciclo di vita) nelle seguenti fasi:
\begin{itemize}
	\item analisi dei requisiti;
	\item consolidamento dei requisiti;
	\item progettazione e codifica della \glo{Technology Baseline};
	\item progettazione di dettaglio e codifica;
	\item validazione e collaudo.
\end{itemize}
Ogni fase può essere suddivisa in periodi, in genere per portare a termine un dato numero di incrementi. Ogni periodo è composto da attività, che possono essere svolte in parallelo solo se non presentano dipendenze tra di esse; se esistono dipendenze, è necessario svolgere tali attività in sequenza.

\subsectionInFile{faseDiAnalisiDeiRequisiti.tex}
\subsectionInFile{faseDiConsolidamentoDeiRequisiti.tex}
\subsectionInFile{faseDiProgettazioneECodificaTB.tex}
\subsectionInFile{faseDiProgettazioneDiDettaglio.tex}
\subsectionInFile{faseDiValidazioneECollaudo.tex}

\subsection{Riassunto della pianificazione degli incrementi}
La seguente tabella riporta come sono stati distribuiti i vari incrementi nelle fasi di progetto.

\begin{center}
\begin{LongTable}{Distribuzione degli incrementi}{Distribuzione degli incrementi}{p{.45\hsize}|p{.45\hsize}}
	\topline
  	\rowcolor{headercolour}
  	\textbf{Fase}       &   \textbf{Incrementi pianificati}  \\ \capsep
  	\endfirsthead
  	\continuedcaption
  	\topline
   	\rowcolor{headercolour}
  	\textbf{Fase}       &   \textbf{Incrementi pianificati}  \\ \capsep
  	\endhead
  	\rowsep
	\defaultfooter{2}
	%\rowcolor{violet!25}
	%\ohdrx{3}{Progettazione e codifica del Proof of Concept e funzionalità essenziali}\\\hline	
	%\rowcolor{violet!25}
	%\ohdrx{3}{}\\\hline	
	Analisi dei requisiti &  nessun incremento previsto  \\	\hline	
	Consolidamento dei requisiti  &  nessun incremento previsto  \\	\hline	
	Progettazione e codifica della \glo{Technology Baseline}  & Incremento I \newline  Incremento II   \\	\hline	
	Progettazione di dettaglio e codifica  &  Incremento III \newline  Incremento IV \newline Incremento V \newline  Incremento VI   \\	\hline	
	Validazione e collaudo  &  nessun incremento previsto  \\	\hline	
	\bottomline
\end{LongTable}
\end{center}