\subsection{Fase di progettazione e codifica della \glo{Technology Baseline}}
 Le ore e i costi di seguito riportati sono finalizzati alla progettazione e codifica della \glo{Technology Baseline}, articolata in due incrementi:
 \begin{enumerate}
 	\item [i.] importazione dati da file locali e visualizzazione basilare dei dati;
 	\item [ii.] implementazione di analisi automatiche, gestione di filtri sui dati e aggiunta o rimozione dei grafici da visualizzare.
 \end{enumerate}
 \subsubsection{Incremento I}
 Le modifiche orarie rispetto indicate nelle tabelle seguenti evidenziano gli scostamenti rispetto al preventivo iniziale della sezione §5.3.1 (Preventivo - Incremento I), non rispetto alle modifiche introdotte con il preventivo a finire della sezione §6.2.4 (Consuntivo - Consolidamento dei requisiti).
 \paragraph{Divisione oraria}
 \begin{tabOrariRuolo}{Incremento I  - Consuntivo della divisione oraria}{Consuntivo della divisione oraria}
 		{\SIM} & 0 & 4 & 0 & 7 (+2) & 2 & 4 & 17 \\
 		{\GIA} & 0 & 0 & 0 & 6 (+1) & 5 (+1) & 6  & 17 \\
 		
 		{\VEL} & 0 & 4  & 5 (+1) & 3 & 5 (+1) & 0 & 17 \\
 				
 		{\MAD} & 0 & 5 (+2) & 0 & 5 & 4 & 3 & 17 \\
 		
 		{\CHF} & 0 & 0 & 4 (+1) & 6 (+1) & 5 & 2  & 17 \\
 		
 		{\COF} & 6 (+1) & 0 & 3 (-2) & 2 & 0 & 6 (+3)  & 17 \\
 		
 		{\REL} & 4 (+1)  & 0 & 0 & 5 & 3 & 5(+1) & 17 \\ \hline
 		
 		{\textbf{Ore totali ruolo}} & 10 (+2) & 13 (+2) & 12 & 34 (+4) & 24 (+2)  & 26 (+4) & 119  \\                      
 \end{tabOrariRuolo}
 I dati ottenuti possono essere riassunti dal seguente istogramma:
 \begin{figure}[H]
 	\centering 
 	 \includegraphics[width=0.9\textwidth]{../images/ist_consuntivo_inc_1.pdf} \\
 	\caption{Istogramma della ripartizione ore/ruolo sostenute durante l'Incremento I }
 	\label{Incremento I - istogramma}
 \end{figure}
 \paragraph{Costo risultante}
 Il costo per ogni ruolo, circoscritto a questo periodo, è il seguente:
 \begin{tabCostiRuolo}{Incremento I  - Consuntivo di periodo dei costi per ruolo}{Consuntivo dei costi per ruolo}
 	{Responsabile} & 10 (+2) & 300,00 (+60,00)\\
 	
 	{Amministratore} & 13 (+2) & 260,00 (+40,00) \\	
 	
 	{Analista} & 12 & 300,00 \\
 	
 	{Progettista} & 34 (+4) & 748,00 (+88,00) \\
 	
 	{Programmatore} & 24 (+2) & 360,00  (+30,00) \\
 	
 	{Verificatore} & 26 (+4) & 390,00 (+60,00) \\ \hline
 	
 	{\textbf{Totale}} & 119 & 2358,00 (+278,00) \\
 \end{tabCostiRuolo}
 I dati ottenuti possono essere riassunti dal seguente areogramma:
 \begin{figure}[H]
 	\centering 
    \includegraphics[width=0.9\textwidth]{../images/aereo_consuntivo_cost_inc_1.pdf} \\
 	\caption{Areogramma della ripartizione ore/ruolo sostenute durante l'Incremento I}
 	\label{ Incremento I - areogramma}
 \end{figure}
 \paragraph{Osservazioni}
La pianificazione iniziale per questo periodo (§5.3) era stata modificata nel preventivo a finire della fase di consolidamento dei requisiti (§6.2.4), perché gli analisti avevano espresso dubbi sulla fattibilità di raggiungere gli obiettivi del periodo nel monte ore inizialmente fissato. Tali dubbi si sono concretizzati e la pianificazione avvenuta in sede di preventivo a finire della fase precedente si è rivelata adeguata. \\
In particolare rispetto a quanto stimato inizialmente in §5.3:
 \begin{itemize}
 	\item \textbf{Responsabile}: sono state richieste un numero maggiore di ore a quanto preventivate perché è stata fatta un'analisi più accurata sulla suddivisione dei compiti dovuta dalla poca esperienza accumulata sul coordinamento del gruppo durante la fase di sviluppo del prodotto;
 	\item \textbf{Amministratore}: il ruolo ha richiesto più ore del previsto. Ciò è dovuto principalmente alla configurazione dell'ambiente \glo{Node.js}, con il quale gli amministratori non avevano familiarità. Questo si è riflesso sui programmatori, i quali hanno dovuto assistere alla configurazione dell'ambiente;
 	\item \textbf{Progettista}: il numero di ore è risultato maggiore rispetto al preventivo a causa della poca esperienza nella progettazione del prodotto; 
 	\item \textbf{Programmatore}: lo studio individuale dei linguaggi e delle \glo{librerie} ha portato a un piccolo aumento del monte ore necessario per acquisire maggiore familiarità con i \glo{framework};
 	\item \textbf{Verificatore}: per correggere le segnalazioni avvenute in sede di Revisione dei Requisiti è stato necessario impiegare un numero di ore di verifica maggiore rispetto al preventivo.
 \end{itemize}
 Tali problematiche erano state previste e non hanno comportato variazione oraria rispetto al preventivo a finire della fase di consolidamento dei requisiti (§6.2.4).
 
 \paragraph{Considerazioni rispetto al preventivo}
In seguito al periodo affrontato, si è potuto constatare che le risorse utilizzate sono maggiori rispetto a quanto preventivato, causando un esito negativo sul bilancio rispetto al preventivo. Non si ritiene di attualizzare la ripianificazione poichè l'aumento dei costi non è significativo, inoltre tutti gli obiettivi sono stati raggiunti e non hanno avuto rallentamenti. 

 \subsubsection{Incremento II}
 Le modifiche orarie indicate nelle tabelle seguenti evidenziano gli scostamenti rispetto al preventivo iniziale della sezione §5.3.3 (Preventivo - Incremento II), non rispetto alle modifiche introdotte con il preventivo a finire della sezione §6.2.4 (Consuntivo - Consolidamento dei requisiti).
 \paragraph{Divisione oraria}
 \begin{tabOrariRuolo}{Incremento II  - Consuntivo della divisione oraria}{Consuntivo della divisione oraria}
 		{\SIM} & 0 & 2 (-1) & 0 & 2(-1) & 4 & 5 & 13 \\
 		
 		{\GIA} & 0 & 0 & 0 & 4 & 2(-2) & 7 & 13 \\	
 		
 		{\VEL} & 0 & 3 (-1) & 3(-1) & 0 & 3  & 4 & 13 \\
 		
 		{\MAD} & 0 & 3 & 3(-1) & 2 & 2(-1) & 3 & 13 \\
 		
 		{\CHF} & 0 & 0 & 3 & 3 (-1) & 2(-2) & 5 (+1) & 13 \\
 		
 		{\COF} & 3 & 0 & 3 & 2 (-1) & 4 & 1 (+1) & 13 \\	
 		
 		{\REL} & 4 & 0 & 0 & 3 & 1(-2) & 5 & 13 \\ \hline
 		
 		{\textbf{Ore totali ruolo}} & 7 & 8 (-2) & 12 (-2) & 16 (-5) & 18 (-7) & 30 (+2) & 91  \\                    
 \end{tabOrariRuolo}
 I dati ottenuti possono essere riassunti dal seguente istogramma:
 \begin{figure}[H]
 	\centering 
 	\includegraphics[width=\linewidth]{../images/ist_consuntivo_inc_2.pdf}
 	\caption{Istogramma della ripartizione ore/ruolo sostenute durante l'Incremento II}
 	\label{Incremento II - istogramma}
 \end{figure}
 \paragraph{Costo risultante}
 Il costo per ogni ruolo, circoscritto a questo periodo, è il seguente:
 \begin{tabCostiRuolo}{Incremento II - Consuntivo di periodo dei costi per ruolo}{Consuntivo dei costi per ruolo}
 	{Responsabile} & 7 & 210,00 \\
 	
 	{Amministratore} & 8 (-2) & 160,00 (-40,00)\\	
 	
 	{Analista} & 12 (-2) & 300,00 (-50,00) \\
 	
 	{Progettista} & 16 (-5) & 352,00 (-110,00)\\
 	{Programmatore} & 18 (-7) & 270,00 (-105,00) \\
 	
 	{Verificatore} & 30 (+2) & 450,00 (+30,00) \\ \hline
 	
 	{\textbf{Totale}} & 91 & 1742,00 (-275,00)\\
 \end{tabCostiRuolo}
 I dati ottenuti possono essere riassunti dal seguente areogramma:
 \begin{figure}[H]
 	\centering 
 	\includegraphics[width=\linewidth]{../images/aereo_consuntivo_cost_inc_2.pdf}
 	\caption{Incremento II - ore/ruolo sostenute nella fase di progettazione e codifica della \glo{Technology Baseline}}
 	\label{Incremento II - areogramma}
 \end{figure}
 \paragraph{Osservazioni}
 Gli obiettivi dell'incremento sono stati raggiunti. Rispetto al \textit{preventivo iniziale} sono state apportate alcune modifiche sulla distribuzione oraria tra alcuni ruoli: 
 \begin{itemize}
 	\item \textbf{Amministratore, Progettista e Programmatore}: le ore preventivate sono risultate minori del necessario a causa del minor lavoro da realizzare durante l'incremento;
 	\item \textbf{Verificatore}: nel preventivo a finire della sezione §6.2.4 erano state assegnate ore aggiuntive dedicate alla verifica, che sono state impiegate tutte.
\end{itemize}
Rispetto al \textit{preventivo a finire} della sezione §6.2.4 invece non sono state necessarie variazioni della distribuzione oraria \textit{tra i ruoli}, segno che la pianificazione correttiva ha centrato gli obiettivi. \\
Ad ogni modo la decisione di presentare il materiale in ingresso alla Revisione di Progettazione in data 2021-03-08 ha fatto slittare il termine dell'incremento II al 2021-03-08. Poiché la decisione è stata approvata il 2021-02-19 (a periodo in corso) essa ha comportato una modifica alla pianificazione \textit{giornaliera} dell'incremento II, raffigurata nel diagramma di Gantt per la fase di progettazione di dettaglio e codifica (§4.3.2.2 - Figura 4). Le conseguenti modifiche alla pianificazione dei periodi successivi sono mostrate nel paragrafo seguente.
 \paragraph{Considerazioni rispetto al preventivo}
Il non utilizzo di tutte le ore pianificate ha portato ad un esito positivo del bilancio rispetto al preventivo di questo incremento. Per questo motivo il gruppo ha deciso di non portare modifiche alla pianificazione in quanto l'esito positivo dell'incremento va a bilanciare l'esito negativo del precedente incremento, inoltre il team è soddisfatto sul bilancio attuale.

\subsubsection{Riepilogo della fase di progettazione e codifica della \glo{Technology Baseline}}
Al termine della fase di progettazione e codifica della \glo{Technology Baseline} il consuntivo di periodo della divisione oraria è il seguente:
\paragraph{Divisione oraria}
\begin{tabOrariRuolo}{Fase di progettazione e codifica della \glo{Technology Baseline} - Consuntivo della divisione oraria}{Consuntivo della divisione oraria}

	\SIM & 0 & 6 & 0 & 11 & 6  & 7 & 30 \\
	
	\GIA & 0 & 0 & 0 & 10 & 9 & 11 &  30 \\	
	
	\VEL & 0 & 7  & 8 & 3 & 8 & 4 &  30 \\
	
	\MAD & 0 & 8 & 3 & 7 & 6 & 6 &  30 \\
	
	\CHF & 0 & 0 & 7  & 9 & 7 & 7 &  30 \\
	
	\COF & 9 & 0 & 6 & 4 & 4 & 7  &  30 \\	
	
	\REL & 8 & 0 & 0 & 8 & 4 & 10 &  30 \\ \hline
	
	\textbf{Ore totali ruolo} & 17 & 21 & 24 & 50 & 42 & 56 & 210 \\                 
\end{tabOrariRuolo}
I dati ottenuti possono essere riassunti dal seguente istogramma:
\begin{figure}[H]
	\centering 
	 \includegraphics[width=0.9\textwidth]{../images/ist_consuntivo_inc_riassuntivo_1_e_2.pdf} \\
	\caption{Istogramma della ripartizione ore/ruolo sostenute nella fase di progettazione e codifica della \glo{Technology Baseline}}
	\label{Progettazione e codifica della Technology Baseline - istogramma}
\end{figure}
\paragraph{Costo risultante}
Il costo per ogni ruolo, circoscritto a questo periodo, è il seguente:
\begin{tabCostiRuolo}{Fase di progettazione e codifica della \glo{Technology Baseline} - Consuntivo di periodo dei costi per ruolo}{Consuntivo dei costi per ruolo}
	
	Responsabile & 17 (+2) & 510,00 (+60) \\
	
	Amministratore    &  21  &  420,00 \\	
	
	Analista    &  24 (-2)  &  600,00 (-50,00) \\
	
	Progettista    &  50 (-1) &  1100,00 (-22,00) \\
	
	Programmatore    &  42 (-5) &  630,00 (-75,00)\\
	
	Verificatore    &  56 (+4)   &  840,00 (+60,00)\\ \hline
	
	\textbf{Totale}    &  210  &  4100,00 (+3,00) \\
\end{tabCostiRuolo}
I dati ottenuti possono essere riassunti dal seguente areogramma:
\begin{figure}[H]
	\centering 
	 \includegraphics[width=0.9\textwidth]{../images/aereo_consuntivo_cost_inc_ria1e2.pdf} \\
	\caption{Areogramma della ripartizione ore/ruolo sostenute nella fase di progettazione e codifica della \glo{Technology Baseline}}
	\label{Progettazione e codifica della Technology Baseline - aerogramma}
\end{figure}