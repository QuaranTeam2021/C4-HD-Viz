\subsection{Fase di analisi dei requisiti}
Il prospetto orario riportato di seguito fa riferimento ad ore non rendicontate. Le ore investite in questa fase sono state utilizzate per l'apprendimento personale e lo studio del problema presentato dal capitolato.
\subsubsection{Divisione oraria}
\begin{tabOrariRuolo}{Fase di analisi dei requisiti - Consuntivo della divisione oraria}{Consuntivo della divisione oraria}
	\SIM & 9(-2) & 0 & 14(-1) & 0 & 0 & 7(+3) & 30 \\
	
	\GIA & 0 & 9 & 13(-2) & 0 & 0 & 8(+2) & 30 \\	
	
	\VEL & 11(-3) & 0 & 4(+3) & 0 & 0 & 15 & 30	\\
	
	\MAD & 9(-2) & 0 & 8(+1) & 0 & 0 & 13(+1) & 30 \\
	
	\CHF & 0 & 12 & 5 & 0 & 0 & 13 & 30 \\
	
	\COF & 0 & 10(-1) & 8(+1) & 0 & 0 & 12 & 30 \\	
	
	\REL & 0 & 9 & 13(-1) & 0 & 0 & 8(+1) & 30 \\ \hline
	
	\textbf{Ore totali ruolo} & 22 & 39 & 66 & 0 & 0 & 83 & 210  \\                     
\end{tabOrariRuolo}
I dati ottenuti possono essere riassunti dal seguente istogramma:
\begin{figure}[H]
	\centering 
	\includegraphics[width=0.9\textwidth]{images/ist_consuntivo_analisi.pdf} \\
	\caption{Istogramma della ripartizione ore/ruolo sostenute nella fase di analisi dei requisiti}
	\label{IstogrammaConsuntivoAnalisi}
\end{figure}
\subsubsection{Costo risultante}
Il costo per ogni ruolo, circoscritto a questo periodo, è il seguente:
\begin{tabCostiRuolo}{Fase di analisi dei requisiti - Consuntivo di periodo dei costi per ruolo}{Consuntivo dei costi per ruolo}
	Responsabile   &  29(-7)  &  870,00(-210,00) \\
	
	Amministratore    &  40(-1)  &  800,00(-20,00) \\	
	
	Analista    &  65(+1)  &  1625,00(+25,00) \\
	
	Progettista    &  0  &  0 \\
	
	Programmatore    &  0  &  0 \\
	
	Verificatore    &  76(+7)  &  1140,00(+105,00) \\ \hline
	
	\textbf{Totale}    &  210  &  4435,00(-100,00) \\
\end{tabCostiRuolo}
I dati ottenuti possono essere riassunti dal seguente areogramma:
\begin{figure}[H]
	\centering 
	\includegraphics[width=0.9\textwidth]{images/areo_consuntivo_analisi.pdf} \\
	\caption{Areogramma della ripartizione ore/ruolo sostenute nella fase di analisi dei requisiti}
	\label{AreogrammaConsuntivoAnalisi}
\end{figure}
\subsubsection{Conclusioni}
Ogni membro del gruppo, in questa fase, si è impegnato a rispettare le ore che gli sono state assegnate. A causa di alcuni problemi che si sono verificati successivamente alla fase di preventivo è stato necessario effettuare una redistribuzione oraria per alcuni ruoli. I ruoli che hanno subito modifiche sono i seguenti:
\begin{itemize}
	\item \textbf{Responsabile}: i contatti tenuti con il \glo{proponente} sono stati da subito esaustivi, i dubbi rimasti sono stati chiariti con uno scambio di mail tra proponente e responsabile. La stesura del \textit{\PdPv{v1.0.0}} e la pianificazione sono risultate più semplici di quanto preventivato. Per questi motivi il ruolo di responsabile ha richiesto meno ore di quelle pianificate;
	\item \textbf{Amministratore}: le ore richieste al ruolo di amministratore sono state ridotte, anche se di poco. Lo sviluppo di layout standard per i documenti ha semplificato la loro stesura. La riduzione è stata minima perché sono stati incontrati problemi nella formattazione delle tabelle, risolti dagli amministratori;
	\item \textbf{Analista}: il ruolo dell'analista ha richiesto un leggero aumento orario. Sono state richieste però diverse ore aggiuntive ai membri che avevano un'idea più chiara dei possibili \glo{casi d'uso};
	\item \textbf{Verificatore}: l'analisi dei documenti è risultata più complicata di quanto preventivato. L'aggiunta tardiva di nuovi termini al glossario e la necessità di uniformare le intestazioni di ogni documento ha richiesto un ulteriore controllo finale di ogni documento. Per questo motivo alcuni membri hanno visto aumentare le ore assegnate come verificatori.
\end{itemize}
\subsubsection{Considerazioni rispetto al preventivo}
La redistribuzione oraria, ed in particolare la diminuzione del lavoro richiesto ai responsabili, ha permesso di ottenere un bilancio positivo vedendo una diminuzione di euro 100,00 rispetto a quanto preventivato in §5.1.2 (Preventivo - Analisi di requisiti - Costo risultante). Poiché il totale di ore lavorative per ciascun componente è rimasto invariato, e poiché si tratta di un periodo non rendicontato non è stato necessario intraprendere alcuna contromisura. \\
Pertanto non sono state apportate modifiche al preventivo.