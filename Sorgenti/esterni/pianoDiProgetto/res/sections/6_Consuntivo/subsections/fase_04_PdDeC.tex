\subsection{Fase di progettazione di dettaglio e codifica}
Le ore e i costi di seguito riportati sono finalizzati alla progettazione di dettaglio e codifica di funzionalità aggiuntive, articolate in quattro incrementi.
\subsubsection{Incremento III}
Le modifiche orarie indicate nelle tabelle seguenti evidenziano gli scostamenti rispetto al preventivo iniziale della sezione §5.4.1 (Preventivo - Incremento III).
\paragraph{Divisione oraria}
\begin{tabOrariRuolo}{Incremento III  - Consuntivo della divisione oraria}{Consuntivo della divisione oraria}
	\SIM  & 0 & 0 & 0 & 11 (-1) & 13 (-1) & 6 (+2) & 30 \\

	\GIA &  4 & 2 & 0 &  8 & 6 (-2) & 10 (+2) & 30 \\	

	\VEL & 0 & 4 & 0 & 9 (-1) & 9 & 8 (+1) & 30 \\

	\MAD & 0 & 0 & 3  & 8 (-1) & 11 (-2) & 8 (+3) & 30 \\

	\CHF & 2 & 0 & 2 & 9 & 10 & 7  & 30 \\

	\COF & 0 & 2 & 0 & 10 (-2) & 8 & 10 (+2) & 30 \\	

	\REL & 0 & 0 & 0 &  10 & 13 (-2) & 7 (+2) & 30 \\ 
	\textbf{Ore totali ruolo} & 6 & 8 & 5 & 65 (-5) & 70 (-7) & 56 (+12) & 210 \\ \hline             
\end{tabOrariRuolo}
I dati ottenuti possono essere riassunti dal seguente istogramma:
\begin{figure}[H]
	\centering 
	\includegraphics[width=0.9\textwidth]{../images/ist_consuntivo_inc_3.pdf} \\
	\caption{Istogramma della ripartizione ore/ruolo sostenute durante l'Incremento III }
	\label{Incremento III - istogramma}
\end{figure}
\paragraph{Costo risultante}
Il costo per ogni ruolo, circoscritto a questo periodo, è il seguente:
\begin{tabCostiRuolo}{Incremento III  - Consuntivo di periodo dei costi per ruolo}{Consuntivo dei costi per ruolo}
	Responsabile   & 6  &  180,00 \\

	Amministratore    &  8  & 160,00  \\	

	Analista    &  5  &  125,00 \\

	Progettista    &  65 (-5) & 1430,00 (-110,00) \\

	Programmatore    &  70 (-7) & 1050,00 (-110,00) \\

	Verificatore    & 56 (+12)  &  840,00 (+180,00) \\ \hline

	\textbf{Totale}    &  210  &  3785,00 (-35,00) \\
\end{tabCostiRuolo}

I dati ottenuti possono essere riassunti dal seguente areogramma:
\begin{figure}[H]
	\centering 
	\includegraphics[width=0.9\textwidth]{../images/aereo_consuntivo_cost_inc_3.pdf} \\
	\caption{Areogramma della ripartizione ore/ruolo sostenute durante l'Incremento III}
	\label{ Incremento III - areogramma}
\end{figure}
\paragraph{Osservazioni}
Durante il periodo è stato portato a termine l'ampliamento previsto delle funzionalità, mentre è stato dedicato meno tempo alle attività di progettazione. In particolare i seguenti ruoli hanno subito redistribuzioni orarie:
\begin{itemize}
	\item \textbf{Programmatori:} hanno dedicato meno tempo di quanto preventivato ad inserire nuovi grafici e ad aggiungere nuove funzionalità a quelli esistenti;
	\item \textbf{Progettisti:} hanno manifestato perplessità riguardo alle scelte architetturali da adottare. Dunque hanno dedicato ore allo studio personale. Hanno richiesto al responsabile di posticipare in futuro le loro mansioni;
	\item \textbf{Verificatori:} le ore destinate alla verifica in questo periodo sono risultate decisamente poche. Perciò le ore avanzate sono state dedicate ad attività di verifica.
\end{itemize}

\paragraph{Considerazioni rispetto al preventivo}
Lo sviluppo procede in linea con quanto pianificato. L'utilizzo di un minor numero di ore ha portato ad un esito positivo del bilancio rispetto al preventivo. Poichè il gruppo si prepara ad affrontare nel prossimo periodo un incremento più impegnativo, non si ritiene necessario apportare modifiche alla ripianificazione.

\subsubsection{Incremento IV}
Le modifiche orarie indicate nelle tabelle seguenti evidenziano gli scostamenti rispetto al preventivo iniziale della sezione §5.4.3 (Preventivo - Incremento IV), non rispetto al preventivo a finire della sezione §6.4.1.4 (Incremento III - Preventivo a finire).
\paragraph{Divisione oraria}
\begin{tabOrariRuolo}{Incremento IV  - Consuntivo della divisione oraria}{Consuntivo della divisione oraria}
	\SIM  & 0 & 0 & 3 & 2(-1) & 8 (+4) & 0 & 13 \\

	\GIA &  2 & 0 & 0 & 1 (-1) & 0 & 10 (+4) & 13 \\	

	\VEL & 0 & 0 & 0 & 0 & 6 (+3) & 7 & 13 \\

	\MAD & 0 & 0 & 0  & 1 & 3 (+1) & 9 (+2) & 13 \\

	\CHF & 4 & 0 & 2 & 4 & 3 (+3) & 0  & 13 \\

	\COF & 0 & 2 & 0 & 6 & 5 (+3) & 0 & 13 \\	

	\REL & 0 & 4 (-2) & 0 &  0 & 0 & 9 (+5) & 13 \\ \hline
	\textbf{Ore totali ruolo} & 6 & 6 (-2) & 5 & 14 (-2) & 25 (+14) & 35 (+11) & 91 (+21)\\ \hline                
\end{tabOrariRuolo}
I dati ottenuti possono essere riassunti dal seguente istogramma:
\begin{figure}[H]
	\centering 
	\includegraphics[width=0.9\textwidth]{../images/ist_consuntivo_inc_4.pdf} \\
	\caption{Istogramma della ripartizione ore/ruolo sostenute durante l'Incremento IV }
	\label{Incremento IV - istogramma}
\end{figure}
\paragraph{Costo risultante}
Il costo per ogni ruolo, circoscritto a questo periodo, è il seguente:
\begin{tabCostiRuolo}{Incremento IV  - Consuntivo di periodo dei costi per ruolo}{Consuntivo dei costi per ruolo}
	Responsabile   &  6  & 180,00  \\
	
	Amministratore    & 6 (-2)  & 120,00 (-40,00)  \\	
	
	Analista    &   5  &  125,00 \\
	
	Progettista    &  14 (-2) &  308,00 (-44,00) \\
	
	Programmatore    &  25 (+14) & 375,00 (+210,00) \\
	
	Verificatore    &  35 (+11) &  525,00 (+165,00) \\ \hline
	
	\textbf{Totale}    & 91 (+21)  &  1633,00 (+291,00) \\
\end{tabCostiRuolo}
I dati ottenuti possono essere riassunti dal seguente areogramma:
	\begin{figure}[H]
	\centering 
	\includegraphics[width=0.9\textwidth]{../images/aereo_consuntivo_cost_inc_4.pdf} \\
	\caption{Areogramma della ripartizione ore/ruolo sostenute durante l'Incremento IV}
	\label{ Incremento IV - areogramma}
\end{figure}
\paragraph{Osservazioni}
Gli obiettivi dell'incremento sono stati raggiunti. Sono state affrontate alcune attività per le quali sono state richieste delle variazioni delle ore, in particolare:
\begin{itemize}
	\item \textbf{Programmatore:} è stato necessario un dispendio maggiore di ore rispetto alle 9 ore assegnate per la risoluzione del problema di integrazione tra grafici dinamici e interfaccia \glo{React};
	\item \textbf{Verificatore:} a periodo in corso sono pervenute le segnalazioni del committente relative alla Revisione di Progettazione. I verificatori sono stati impiegati nelle conseguenti attività di correzione della documentazione.
\end{itemize}	

\paragraph{Considerazioni rispetto al preventivo}
In linea con le decisioni prese in sede di preventivo a finire dell'incremento IV (finalizzate a sopperire a una minore disponibilità successiva del gruppo) l'impegno in questo incremento è risultato di 21 ore maggiore di quanto preventivato inizialmente, con conseguente aumento dei costi rispetto a quanto preventivato. Il gruppo non si ritiene soddisfatto dello scostamento dei costi, nonostante l'aumento di utilizzo delle risorse, si ritiene fiducioso e prevede di riuscire a contenere i costi e a rispettare i preventivi dei prossimi incrementi.

\subsubsection{Incremento V}
Le modifiche orarie indicate nelle tabelle seguenti evidenziano gli scostamenti rispetto al preventivo iniziale della sezione §5.4.5 (Preventivo - Incremento V).
\paragraph{Divisione oraria}
\begin{tabOrariRuolo}{Incremento V  - Consuntivo della divisione oraria}{Consuntivo della divisione oraria}
	\SIM & 2 (-1) & 1 (-1) & 0 & 0 & 1 (-1) & 3 & 7 \\
	
	\GIA & 0 & 0 & 0 & 2 (-1) & 2 (-1) & 3 (-1) & 7 \\	
	
	\VEL & 0 & 1 (-1) & 0 & 0 & 4 (-1) & 2 (-1) & 7 \\
	
	\MAD & 0 & 1 (-1) & 0 & 1 (-1) & 2 & 3 (-1) & 7 \\
	
	\CHF & 0 & 1 (-1) & 0 & 2 (-1) & 0 & 4 (-1) & 7 \\
	
	\COF & 0 & 0 & 0 & 0 & 2 (-2) & 5 (-1) & 7 \\	
	
	\REL & 2 (-1) & 0 & 0 & 0 & 2 (-1) & 3 (-1) & 7 \\ \hline
	\textbf{Ore totali ruolo} & 4 (-2) & 4 (-4) & 0 & 5 (-3) & 13 (-6) & 23 (-6) & 49 (-21)\\ \hline           
\end{tabOrariRuolo}
I dati ottenuti possono essere riassunti dal seguente istogramma:
\begin{figure}[H]
	\centering 
	\includegraphics[width=0.9\textwidth]{../images/ist_consuntivo_inc_5.pdf} \\
	\caption{Istogramma della ripartizione ore/ruolo sostenute durante l'Incremento V }
	\label{Incremento V - istogramma}
\end{figure}
\paragraph{Costo risultante}
Il costo per ogni ruolo, circoscritto a questo periodo, è il seguente:
\begin{tabCostiRuolo}{Incremento V  - Consuntivo di periodo dei costi per ruolo}{Consuntivo dei costi per ruolo}
	Responsabile   & 4 (-2)  &  120,00 (-60,00) \\
	
	Amministratore    & 4 (-4)  &  80,00 (-80,00) \\	
	
	Analista    &  0  & 0  \\
	
	Progettista    & 5 (-3)  &  110,00 (-66,00) \\
	
	Programmatore    & 13 (-6)  &  195,00 (-90,00)\\
	
	Verificatore    &   23 (-6) &  345,00 (-90,00)\\ \hline
	
	\textbf{Totale}    &  49 (-21)  & 850,00  (-386,00) \\
\end{tabCostiRuolo}
I dati ottenuti possono essere riassunti dal seguente areogramma:
\begin{figure}[H]
	\centering 
	\includegraphics[width=0.9\textwidth]{../images/aereo_consuntivo_cost_inc_5.pdf} \\
	\caption{Areogramma della ripartizione ore/ruolo sostenute durante l'Incremento V}
	\label{ Incremento V - areogramma}
\end{figure}
\paragraph{Osservazioni}
La riduzione del monte ore riflette le modifiche apportate con il preventivo a finire dell'Incremento IV (§6.4.2.4), che si sono rivelate adeguate. \\
È stata portata a termine la progettazione dell'architettura, che verrà presentata al \glo{committente} in sede di \glo{Product Baseline}.

\paragraph{Considerazioni rispetto al preventivo}
Lo sviluppo procede in modo fluido, l'impiego maggiore nel precedente periodo si è dimostrato corretto, e ha portato a un minor impiego richiesto per l'Incremento V. \\
La pianificazione si è dimostrata eccessiva rispetto a quanto preventivato ma al contempo soddisfacente, con la quale si vanno a recuperare le risorse aggiuntive impiegate nell'Incremento IV, portando il gruppo ad essere fiducioso nella pianificazione. Il preventivo rimane inalterato, così come la pianificazione dell'Incremento VI.

\subsubsection{Incremento VI}
Le modifiche orarie indicate nelle tabelle seguenti evidenziano gli scostamenti rispetto al preventivo iniziale della sezione §5.4.7 (Preventivo - Incremento V).
\paragraph{Divisione oraria}
\begin{tabOrariRuolo}{Incremento VI  - Consuntivo della divisione oraria}{Consuntivo della divisione oraria}
	\SIM & 0 & 0 & 0 & 0 & 5 (+2) & 0 & 5 \\

	\GIA & 1 & 1 & 0 & 0 & 2 (+2) & 1 & 5 \\	

	\VEL & 1 & 0 & 0 & 2 (+2) & 0 & 2 & 5 \\

	\MAD & 0 & 0 & 0 & 2  & 2 (+2) & 1 & 5 \\	

	\CHF & 0 & 1 & 0 & 0 & 4 (+2) & 0 & 5 \\

	\COF & 0 & 0 & 0 & 0 & 4 (+2) & 1 & 5 \\	

	\REL & 0 & 1 & 0 & 3 (+2) & 1 & 0 & 5 \\ \hline
	\textbf{Ore totali ruolo} & 2 & 3 & 0 & 7 (+4) & 18 (+10) & 5 & 35 (+14)\\      
\end{tabOrariRuolo}
I dati ottenuti possono essere riassunti dal seguente istogramma:
\begin{figure}[H]
	\centering 
	\includegraphics[width=0.9\textwidth]{../images/ist_consuntivo_inc_6.pdf} \\
	\caption{Istogramma della ripartizione ore/ruolo sostenute durante l'Incremento VI }
	\label{Incremento VI - istogramma}
\end{figure}
\paragraph{Costo risultante}
Il costo per ogni ruolo, circoscritto a questo periodo, è il seguente:
\begin{tabCostiRuolo}{Incremento VI  - Consuntivo di periodo dei costi per ruolo}{Consuntivo dei costi per ruolo}
	Responsabile   & 2  &  60,00 \\
	
	Amministratore    & 3  &  60,00 \\	
	
	Analista    &  0  & 0  \\
	
	Progettista    & 7 (+4)  &  154,00 (+88,00) \\
	
	Programmatore    & 18 (+10)  &  270,00 (+150,00) \\
	
	Verificatore    &   5 &  75,00 \\ \hline
	
	\textbf{Totale}    &  35 (+14) & 619,00 (+238,00)  \\
\end{tabCostiRuolo}
I dati ottenuti possono essere riassunti dal seguente areogramma:
\begin{figure}[H]
	\centering 
	\includegraphics[width=0.9\textwidth]{../images/aereo_consuntivo_cost_inc_6.pdf} \\
	\caption{Areogramma della ripartizione ore/ruolo sostenute durante l'Incremento VI}
	\label{ Incremento VI - areogramma}
\end{figure}
\paragraph{Osservazioni}
In seguito alla sospensione dell'esito alla \glo{Product Baseline}, i progettisti e programmatori hanno dovuto impiegare un numero di ore per correggere le segnalazioni fatte dal \glo{committente}. In particolare sono state impiegate:
\begin{itemize}
	\item 2 ore di progettazione per la risoluzione dei problemi concernenti l'architettura;
	\item 8 ore di programmazione per la risoluzione dei problemi concernenti l'architettura;
	\item 2 ore di progettazione per la risoluzione dei problemi concernenti il database;
	\item 2 ore di programmazione per la risoluzione dei problemi concernenti il database;
\end{itemize}
per un totale di 14 ore aggiuntive.
\paragraph{Considerazioni rispetto al preventivo}
In base al risultato ottenuto, nonostante il consumo delle risorse previsto non sia stato rispettato, il bilancio rispetto al preventivo di questo incremento risulta essere in pari. Il gruppo si prepara ad affrontare l'ultimo incremento, concentrandosi principalmente sulla parte dei test. Una parte dei test è stata già realizzata e per questo motivo il gruppo si ritiene fiducioso, e prevede che il prossimo incremento sia minore rispetto al preventivo, rientrando nel bilancio finale.
\subsubsection{Riepilogo della fase di progettazione di dettaglio e codifica}
Al termine della fase di progettazione di dettaglio e codifica il consuntivo di periodo della divisione oraria è il seguente:
\paragraph{Divisione oraria}
\begin{tabOrariRuolo}{Fase di progettazione di dettaglio e codifica - Consuntivo della divisione oraria}{Consuntivo della divisione oraria}
	
	\SIM & 2 & 1 & 3 & 13 & 27  & 9 & 55 \\
	
	\GIA & 7 & 3 & 0 & 11 & 10 & 24 &  55 \\	
	
	\VEL & 1 & 5 & 0 & 11 & 19 & 19 &  55 \\
	
	\MAD & 0 & 1 & 3 & 12 & 18 & 21 &  55 \\
	
	\CHF & 6 & 2 & 4 & 15 & 17 & 11 &  55 \\
	
	\COF & 0 & 4 & 0 & 16 & 19 & 16  &  55 \\	
	
	\REL & 2 & 5 & 0 & 13 & 16 & 19 &  55 \\ \hline
	
	\textbf{Ore totali ruolo} & 18 & 21 & 10 & 91 & 126 & 119 & 385 \\                 
\end{tabOrariRuolo}
I dati ottenuti possono essere riassunti dal seguente istogramma:
\begin{figure}[H]
	\centering 
	\includegraphics[width=0.9\textwidth]{../images/ist_consuntivo_inc_riassuntivo_3_4_5_6.pdf} \\
	\caption{Istogramma della ripartizione ore/ruolo sostenute nella fase di progettazione di dettaglio e codifica}
	\label{Progettazione e codifica della Product Baseline - istogramma}
\end{figure}
\paragraph{Costo risultante}
Il costo per ogni ruolo, circoscritto a questo periodo, è il seguente:
\begin{tabCostiRuolo}{Fase di progettazione di dettaglio e codifica - Consuntivo di periodo dei costi per ruolo}{Consuntivo dei costi per ruolo}
	
	Responsabile & 18 (-2) & 540,00 (-60) \\
	
	Amministratore    &  21 (-8)  &  420,00 (-160,00) \\	
	
	Analista    &  10  &  250,00  \\
	
	Progettista    &  91 (-6) &  2002,00 (-132,00) \\
	
	Programmatore    &  126 (+16) &  1815,00 (+240,00)\\
	
	Verificatore    &  119 (+14)   &  1830,00 (+210,00)\\ \hline
	
	\textbf{Totale}    &  385 (+14) &  6887,00 (+98,00) \\
\end{tabCostiRuolo}
I dati ottenuti possono essere riassunti dal seguente areogramma:
\begin{figure}[H]
	\centering 
	\includegraphics[width=0.9\textwidth]{../images/aereo_consuntivo_cost_inc_ria3e4e5e6.pdf} \\
	\caption{Areogramma della ripartizione ore/ruolo sostenute nella fase di progettazione di dettaglio e codifica}
	\label{Progettazione di dettaglio e codifica - aerogramma}
\end{figure}

