\subsection{Fase di validazione e collaudo}
\begin{tabAttRischi}{Riepilogo dei rischi riscontrati nella fase di validazione e collaudo}{Riepilogo dei rischi riscontrati nella fase di validazione e collaudo}
	RO1 - \newline
	Impegni accademici
	&
	Alta
	&
	Durante la fase di validazione e collaudo alcuni membri del gruppo hanno iniziato lo stage aziendale, potendo quindi dedicare meno ore al progetto. I corsi opzionali seguiti da alcuni membri del gruppo si sono conclusi, in concomitanza però con l'inizio della sessione estiva e dei primi appelli.
	&
	La soluzione proposta è stata quella di ridistribuire i compiti all'interno del gruppo, assegnando ai componenti non impegnati nello stage le attività più urgenti. I componenti impegnati nello stage si sono occupati di svolgere le attività a loro assegnate alla sera o nei giorni non impegnati dallo stage aziendale. La ridistribuzione dei compiti all'interno del gruppo ha tenuto conto anche dell'impegno in termini orari nella preparazione di appelli di alcuni membri del gruppo. \\ \hline
	RI1 - \newline
	Comunicazione interna
	&
	Alta
	&
	Gli impegni personali e accademici hanno portato a maggiori difficoltà nell'organizzazione di incontri interni, portando spesso all'assenza di alcuni membri del gruppo. Si è manifestata la necessità di intensificare la comunicazione tra i membri che hanno seguito la parte \glo{front-end} e quelli che hanno seguito la parte \glo{back-end} dell'applicazione.
	&
	Entrambe le problematiche sono state gestite intensificando l'utilizzo dell'\glo{ITS}, ricorrendo a \glo{issue} più dettagliate. I componenti non presenti agli incontri interni hanno potuto recuperare le decisioni prese attraverso i verbali e le \glo{issue} a loro assegnate. 
	\\ \hline
	RO3 - \newline
	Calcolo delle tempistiche
	&
	Media
	&
	Gli impegni accademici e l'inizio dello stage aziendale da parte di alcuni componenti del gruppo hanno portato ad un allungamento dei tempi di conclusione della fase corrente.
	&
	Il carico di lavoro è stato ripartito tra i membri che avevano più tempo a disposizione, con l'impegno da parte di tutti i componenti del gruppo a svolgere le attività a loro assegnate nel più breve tempo possibile.
	\\ \hline
	RR1 - \newline
	Comprensione errata dei requisiti
	&
	Media
	&
	In seguito al colloquio avvenuto con il \glo{proponente} è emersa la necessità di apportare delle modifiche alla visualizzazione \glo{Force Field}. La visualizzazione non evidenziava in modo soddisfacente le relazioni presenti tra i dati (spessore e lunghezza dei link) e non consentiva l'applicazione delle modifiche desiderate (cambio della forza di attrazione tra i nodi). Il \glo{proponente} ha inoltre evidenziato come l'interfaccia dell'applicazione non spiegasse le funzionalità offerte.
	&
	Un componente del gruppo ha apportato le modifiche correttive richieste alla visualizzazione \glo{Force Field}. I membri del gruppo addetti allo sviluppo della parte \glo{front-end} si sono confrontati e hanno proposto al resto del gruppo delle possibili modifiche apportabili all'interfaccia dell'applicazione. La soluzione scelta è stata quindi implementata.
	\\ \hline
\end{tabAttRischi}