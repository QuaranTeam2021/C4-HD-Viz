\subsection{Valutazione sull'efficacia delle contromisure adottate}
Dall'attualizzazione dei rischi si possono evincere le difficoltà incontrate durante lo sviluppo del progetto. Buona parte dei rischi sono stati preventivati all'inizio della pianificazione. La maggior parte dei rischi è dovuto all'inesperienza del gruppo, alla tecnologia utilizzata e allo sviluppo di un progetto accademico con un numero elevato di collaboratori.
In merito all'efficacia delle contromisure adottate si può commentare che:

\begin{itemize}
	\item l'inizio dello stage da parte di alcuni membri del gruppo e gli impegni accademici hanno portato a una dilatazione complessiva dei tempi necessari. Le contromisure adottate, attraverso una ridistribuzione dei carichi di lavoro, sono state reputate utili nel contenere le tempistiche, ma non sono riuscite ad evitare uno scostamento della data di consegna;
	\item l'intensificazione dell'utilizzo dell'\glo{ITS} ha consentito di colmare l'insufficiente comunicazione tra membri del gruppo assegnati allo sviluppo di componenti diverse. Un utilizzo maggiore di questo strumento nelle fasi precedenti avrebbe potuto limitare le incomprensioni insorte e ridurre il tempo sprecato in incontri interni;
	\item l'assenza di alcuni membri del gruppo ad alcuni incontri interni ha portato inizialmente a delle incomprensioni e ad una maggiore difficoltà nel monitoraggio degli avanzamenti. L'utilizzo di \glo{issue} e verbali interni ha solo parzialmente risolto la problematica, permangono comunque delle difficoltà ad adottare un'efficace comunicazione che possa raggiungere tutti i membri del gruppo;
	\item il riscontro tardivo delle segnalazioni del \glo{proponente} evidenzia come sarebbe stato più utile un contatto più frequente con quest'ultimo. Le segnalazioni ricevute sono state recepite dal gruppo, che in breve tempo è riuscito a correggere le problematiche più urgenti. Incontri più frequenti con il \glo{proponente} avrebbero probabilmente portato a raggiungere gli obiettivi prefissati più velocemente e a identificare subito delle interpretazioni errate dei requisiti.
\end{itemize}
La risoluzione di alcune problematiche ha comportato l'utilizzo di un maggior numero di ore rispetto a quanto preventivato, ma ha fatto comprendere quali siano le difficoltà che possono insorgere durante lo sviluppo di un prodotto e quanto sia importante la pianificazione per lo sviluppo di un prodotto di qualità. 