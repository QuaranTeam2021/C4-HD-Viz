\subsection{Fase di progettazione di dettaglio e codifica}
\begin{tabAttRischi}{Riepilogo dei rischi riscontrati nella fase di progettazione di dettaglio e codifica}{Riepilogo dei rischi riscontrati nella fase di progettazione di dettaglio e codifica}
	RO1 - \newline
	Impegni accademici
	&
	Media
	&
	Durante la fase di progettazione di dettaglio e codifica i membri del gruppo hanno continuato a seguire il corso di Ingegneria del Software, oltre che dei corsi opzionali per alcuni membri del gruppo. La difficoltà maggiore è stata però contribuire allo svolgimento del progetto nei periodi vicini la prima prova scritta di Ingegneria del Software e il compitino di Introduzione all'Apprendimento Automatico.
	&
	La soluzione proposta è stata quella di impiegare maggiormente le risorse nel periodo precedente alle prove scritte e subito dopo. Soluzione che non si è però rivelata facile da applicare in quanto tutti i membri del gruppo sono stati coinvolti nella preparazione dello scritto di Ingegneria del Software, portando un rallentamento nel raggiungimento degli obiettivi previsti. \\ \hline
	RT1 - \newline
	Inesperienza tecnologica
	&
	Media
	&
	Il gruppo che ha lavorato con la tecnologia \glo{D3.js} ha riscontrato una certa difficoltà nel gestire i grafici con animazioni in quanto il \glo{DOM} viene manipolato anche dal \glo{framework} \glo{React}. Questo rischio era stato già riscontrato nella fase precedente.
	&
	Il problema è stato finalmente risolto dopo aver condiviso il problema con gli altri membri, i quali si sono interessati facendo delle ricerche individuali e proponendo una loro soluzione.
	\\ \hline
	RI1 - \newline
	Comunicazione interna
	&
	Alta
	&
	La suddivisione dei lavori ha portato a un dislivello di avanzamento tra il \glo{front-end} e il \glo{back-end} dell'applicazione. La mancanza di comunicazione e alcune incomprensioni hanno portato a un rallentamento nello sviluppo dell'applicazione.  
	&
	Tale problematica è stata segnalata ai responsabili, i quali si sono riuniti per valutare una possibile soluzione per evitare ulteriori rallentamenti nello sviluppo del progetto. I responsabili hanno pensato di intervenire direttamente cercando di stabilire obiettivi più facilmente raggiungibili in un breve tempo, in modo tale da permettere ai gruppi di organizzarsi meglio. Una seconda soluzione proposta è stata quella di coinvolgere un altro membro del \glo{team} a supporto del gruppo che era più indietro. 
	\\ \hline
	RO3 - \newline
	Calcolo delle tempistiche
	&
	Media
	&
	Il gruppo si è trovato a dover ripensare alcuni aspetti dell'architettura dell'applicazione a seguito del colloquio con il \glo{committente} durante la \glo{Product Baseline}. Questo ha portato un allungamento dei tempi di conclusione della fase corrente e l'impiego di più ore per la progettazione.
	&
	A seguito del primo colloquio per la \glo{Product Baseline} il gruppo si è quasi interamente dedicato alla rielaborazione della progettazione per poter presentare al \glo{committente} una proposta di architettura più adeguata e corretta per il tipo di applicazione proposto dal \glo{proponente}.
	\\ \hline
	RT2 - \newline
	Impedimenti software
	&
	Bassa
	&
	Il gruppo si è trovato in difficoltà nel momento dei test del codice \glo{JavaScript}. Inizialmente si era deciso di utilizzare \glo{Mocha.js}, il quale era stato utilizzato per lo sviluppo del \glo{Proof of Concept}. Con l'avanzamento nello sviluppo ci si è trovati di fronte alla necessità di testare costrutti più complessi, che avrebbero richiesto una configurazione molto più complessa.
	&
	Per evitare ulteriori rallentamenti i membri hanno individuato e proposto un nuovo \glo{framework} per testare il codice. Il nuovo \glo{framework} utilizzato è \glo{Jest}, il quale è già integrato con \glo{React} e non è necessaria ulteriore configurazione.
	\\ \hline
\end{tabAttRischi}