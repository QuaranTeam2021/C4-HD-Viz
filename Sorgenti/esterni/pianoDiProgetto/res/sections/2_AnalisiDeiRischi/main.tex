\section{Analisi dei rischi}
\subsection{Gestione dei rischi}
L'analisi dei rischi viene effettuata con lo scopo di identificare e prevenire i rischi che possono verificarsi durante lo sviluppo di un progetto, definendo inoltre come possono essere affrontati.
Le attività che contribuiscono all'identificazione e alla risoluzione dei rischi sono le seguenti:

\begin{figure}[H]
	\centering
	\includegraphics[width=1\linewidth]{img/Gestione_dei_rischi_2.1.png}
	\caption{Attività di gestione dei rischi}
\end{figure}

\subsection{Elenco dei rischi}
I rischi vengono raggruppati a seconda della tipologia:
\begin{itemize}
	\item \textbf{RT}: Rischi Tecnologici;
	\item \textbf{RO}: Rischi Organizzativi;
	\item \textbf{RI}: Rischi Interpersonali;
	\item \textbf{RR}: Rischi legati ai Requisiti.
\end{itemize}
Ogni rischio è identificato da un codice composto dalla tipologia di appartenenza seguita da un numero progressivo.
\begin{tabRischi}
		RT1 \newline
		Inesperienza tecnologica
		&
		Le tecnologie adottate nello sviluppo del progetto sono nuove per i componenti del gruppo, i quali potrebbero riscontrare problemi durante l'utilizzo di tali tecnologie a causa dell'inesperienza.
		&
		I membri del gruppo dovranno comunicare apertamente tali difficoltà. Il responsabile dovrà rilevare eventuali lacune o conoscenze dei componenti del gruppo.
		& 
		I componenti del gruppo si impegnano ad approfondire singolarmente lo studio di tali tecnologie.
		I compiti che richiedono maggiori conoscenze verranno assegnati a più persone, in modo da favorire il confronto e la condivisione delle informazioni acquisite. \\
	\hline
		RT2 \newline
		Impedimenti software
		&
		Impossibilità di utilizzare gli strumenti software identificati per lo sviluppo del progetto a causa di incompatibilità con gli strumenti di lavoro dei singoli componenti del gruppo.
		&
		I membri del gruppo che incorreranno in questo rischio dovranno comunicarlo al responsabile.
		& 
		Identificazione di soluzioni software alternative, sfruttabili dai componenti del gruppo che hanno riscontrato tale problematica, per poter svolgere efficacemente i compiti assegnati.\\
	\hline
		RO1 \newline
		Impegni accademici
		&
		I componenti del gruppo possono essere meno disponibili in determinati momenti a causa di impegni di tipo accademico. I periodi di minore disponibilità possono differire tra i diversi membri del gruppo a causa delle diverse situazioni curriculari.
		&
		Gli impegni accademici vengono segnalati al responsabile e discussi insieme durante gli incontri interni.
		&
		L'assegnazione di incarichi avverrà nel rispetto degli impegni segnalati.\\
	\hline
		RO2 \newline
		Impegni personali
		&
		I componenti del gruppo possono essere meno disponibili in determinati momenti a causa di impegni personali. Rientrano in questa casistica eventuali impegni lavorativi o impegni ricorrenti di natura non accademica.
		&
		Gli impegni personali, insieme a eventuali impegni imprevisti devono essere segnalati tempestivamente al responsabile.
		& 
		L'assegnazione di incarichi avverrà nel rispetto degli impegni personali segnalati. All'insorgere di imprevisti il responsabile valuterà una riallocazione delle risorse.\\
	\hline
		RO3 \newline
		Calcolo delle tempistiche
		&
		La presenza di tecnologie nuove ai componenti del gruppo e l'inesperienza nella gestione di un progetto può comportare un errato calcolo delle tempistiche.
		&
		Il componente del gruppo che dovesse riscontrare il rischio di non rispettare le scadenze prestabilite nello svolgimento di un task a lui assegnato deve comunicarlo tempestivamente al responsabile.
		& 
		Il responsabile provvederà allo spostamento della scadenza, se possibile, o all'assegnazione di maggiori risorse.\\
	\hline
		RO4 \newline
		Calcolo dei costi
		&
		Strettamente correlato al rischio RO3. L'inesperienza nella gestione di un progetto può causare errate valutazioni economiche, soggette a imprecisioni comportate da eventuali calcoli imprecisi delle tempistiche.
		&
		La predisposizione di tabelle condivise consente al responsabile di monitorare le ore di lavoro di ciascun componente.
		& 
		L'insorgenza di significative variazioni orarie rispetto al preventivo iniziale comporterà un'attenta valutazione del consuntivo di periodo, al fine di identificare le cause dell'errata valutazione e possibili mitigazioni future. Si procederà quindi all'aggiornamento delle previsioni tramite la stesura del preventivo a finire.\\
	\hline
		RI1 \newline
		Comunicazione interna
		&
		I membri del gruppo potrebbero non essere reperibili in determinati momenti. La comunicazione interna potrebbe non essere efficace e causare incomprensioni.
		&
		Eventuali momenti di indisponibilità da parte di un componente del gruppo devono essere preventivamente segnalati e giustificati al responsabile. I membri del gruppo si impegnano ad utilizzare i canali di comunicazione interna predisposti.
		& 
		Il gruppo ha predisposto molteplici canali di comunicazione interna. Vengono inoltre organizzati incontri a cadenza settimanale per poter discutere dell'avanzamento del progetto.\\
	\hline
		RI2 \newline
		Comunicazione col \glo{proponente}
		&
		Il \glo{proponente} potrebbe non essere reperibile in determinati periodi o potrebbe rendersi disponibile ad effettuare incontri tramite video-chiamata solamente in determinate date o orari.
		&
		Sono stati preventivamente predisposti dei canali di comunicazione con il \glo{proponente}; le video-chiamate, su richiesta del \glo{proponente}, vengono effettuate in contemporanea con più gruppi e vengono organizzate con un adeguato preavviso, al fine di garantire la presenza di un numero adeguato di componenti del gruppo.
		& 
		Il gruppo provvederà a raggruppare domande e segnalazioni per il \glo{proponente}, cercando di rendere efficace ed efficiente la comunicazione tra le parti.\\
	\hline
		RI3 \newline
		Contrasti interni
		&
		Nel corso dello svolgimento del progetto potrebbero insorgere contrasti e tensioni tra i componenti del gruppo.
		&
		I membri del gruppo si impegnano a limitare le tensioni al fine di non inficiare lo svolgimento delle attività.
		& 
		Il responsabile ha il ruolo di mediatore in eventuali contrasti.\\
	\hline
		RR1 \newline
		Comprensione errata dei requisiti
		&
		Alcuni requisiti possono essere classificati o interpretati in modo erroneo.
		&
		L'analista deve segnalare al responsabile e agli altri analisti le perplessità riguardanti l'Analisi dei Requisiti
		& 
		Ogni dubbio deve essere chiarito mediante comunicazione con il \glo{proponente} o discussione interna. Gli errori segnalati devono essere corretti in seguito a ogni revisione.\\
	\hline
\end{tabRischi}


\subsection{Occorrenza e gravità dei rischi}
Come specificato nell'attività di analisi nella sezione §2.1 per ogni rischio viene valutato il grado del rischio, ovvero l'occorrenza e la gravità che un rischio potrebbe avere durante lo sviluppo del progetto. Viene riportata di seguito una tabella che descrive per ogni rischio, identificato dallo stesso codice indicato nella tabella precedente, la sua occorrenza di avvenimento e gravità dell'impatto.\\

\begin{tabRischi2}
		RT1 	&	Alta	&	Media\\
		RT2 	&	Bassa	&	Alta\\
		RO1 	&	Alta	&	Bassa\\
		RO2 	&	Media	&	Media\\
		RO3 	&	Alta	&	Media\\
		RO4		&	Alta	&	Alta\\
		RI1 	&	Bassa	&	Media\\
		RI2 	&	Bassa	&	Media\\
		RI3 	&	Media	&	Bassa\\
		RR1 	&	Media	&	Media\\
\end{tabRischi2}
