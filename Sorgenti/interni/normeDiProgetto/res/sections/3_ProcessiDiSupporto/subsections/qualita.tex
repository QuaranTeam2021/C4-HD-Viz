\subsection{Gestione della qualità}

\subsubsection{Descrizione}
Lo scopo della gestione della qualità è di garantire che il prodotto e i servizi offerti rispettino gli obiettivi di qualità e che i bisogni del \glo{proponente} siano soddisfatti.
Infatti con la gestione della qualità si intende ottenere: 
\begin{itemize}
	\item qualità nell'organizzazione e nei suoi processi; 
	\item qualità nel prodotto; 
	\item qualità provata oggettivamente; 
	\item soddisfazione finale di cliente e \glo{proponente}.
\end{itemize}
Nel \PdQv{v\versionPdQ{}} sono descritte le modalità utilizzate per garantire la qualità nello sviluppo del progetto. In particolare per ogni processo ed ogni prodotto vengono descritti gli obiettivi e le metriche per la valutazione del raggiungimento degli stessi.

\subsubsection{Attività}
\paragraph{Pianificazione}
Tutte le attività periodiche di controllo della qualità vengono applicate nel \PdQv{v\versionPdQ{}}. Durante tutta la durata del progetto si fa riferimento a questo documento, il quale viene costantemente aggiornato con nuove metriche e vengono rese disponibili le verifiche realizzate. 
\paragraph{Garanzia di qualità del prodotto}
La qualità del prodotto viene garantita attraverso i processi descritti in §3.5 (Verifica) e §3.6 (Validazione). Questi processi permettono di verificare il rispetto delle metriche di qualità stabilite. Un altro fattore che contribuisce alla qualità del prodotto è il confronto con il \glo{proponente}, con il quale si può avere un feedback diretto, monitorando la soddisfazione dei \glo{requisiti} concordati.
\paragraph{Garanzia di qualità dei processi}
La qualità dei processi viene garantita attraverso lo svolgimento corretto e normato delle attività che compongono tali processi. Viene monitorato lo svolgimento delle attività e il perseguimento dei principi di efficacia ed efficienza del prodotto durante tutto il suo ciclo di vita. I risultati sono quindi ottenuti rispettando le norme e gli standard che sono stati scelti come riferimento.
\paragraph{Classificazione}
\subparagraph{Classificazione delle caratteristiche del prodotto}
Le caratteristiche qualitative del prodotto vengono identificate nel modo seguente:
\begin{center}
	\textbf{QC-[Codice]}
\end{center}
Dove:
\begin{itemize}
	\item \textbf{QC}: è un'abbreviazione per \textit{Quality Characteristic};
	\item \textbf{Codice}: è un numero naturale che identifica la caratteristica di prodotto.
\end{itemize}
\subparagraph{Classificazione dei processi}
Le caratteristiche qualitative di un processo vengono identificate nel modo seguente:
\begin{center}
	\textbf{QP-[Codice]}
\end{center}
Dove:
\begin{itemize}
	\item \textbf{QP}: è un'abbreviazione per \textit{Quality Process};
	\item \textbf{Codice}: è un numero naturale che identifica il processo.
\end{itemize}
\subparagraph{Classificazione delle metriche}
Le metriche vengono identificate nel modo seguente:
\begin{center}
	\textbf{QM-[Tipo]-[Codice]}
\end{center}
Dove:
\begin{itemize}
	\item \textbf{QM}: è un'abbreviazione per \textit{Quality Metric}.
	\item \textbf{Tipo}: riguarda la tipologia della metrica e può assumere i seguenti valori:
	\begin{itemize}
		\item \textbf{PROC}: se la metrica è riferita a un processo;
		\item \textbf{PROD}: se la metrica è riferita a una caratteristica di prodotto.
	\end{itemize}
	\item \textbf{Codice}: è un numero naturale che identifica la metrica.
\end{itemize}

\subsubsection{Metriche}
\paragraph{QP-2 Gestione della qualità}
\begin{tabmetriche2}{Metriche di processo relative al processo di qualità}{Tabella delle metriche per il processo di qualità}
	QM-PROC-3 & Metriche soddisfatte (MS) & È un valore in percentuale che mira a rappresentare la qualità del processo o prodotto analizzato. & $\displaystyle\frac{\textit{\#metriche soddisfatte}}{\textit{\#totale di metriche}}\times100$ \T\\
	\hline
\end{tabmetriche2}

\subsubsection{Strumenti}
Di seguito sono descritti gli strumenti identificati per il processo di gestione della qualità.
\begin{itemize}
	\item parte dei processi forniti dallo standard ISO/IEC 12207;
	\item metriche di qualità stabilite;
\end{itemize}
