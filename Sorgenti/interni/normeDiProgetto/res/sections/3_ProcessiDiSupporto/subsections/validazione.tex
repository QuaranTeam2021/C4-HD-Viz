\subsection{\glo{Validazione}}
\subsubsection{Descrizione}
L'obiettivo del processo di \glo{validazione} è il controllo di qualità del prodotto rispetto ai \glo{requisiti} del \glo{committente} per garantirne la soddisfazione.
La \glo{validazione} è un'attività normalmente prevista sul prodotto finito ma è comunque possibile effettuare delle operazioni di \glo{validazione} anche durante il processo di sviluppo.
 
\subsubsection{Attività}
\paragraph{Pianificazione}
Per gestire il processo di \glo{validazione} è necessario effettuare una pianificazione, la quale deve focalizzarsi sui seguenti punti essenziali:
\begin{itemize}
	\item identificare gli oggetti da validare;
	\item quali operazioni eseguire;
	\item risorse, responsibilità e gestione delle scadenze legate alla pianificazione.
\end{itemize}
\paragraph{Test}
\subparagraph{Test di accettazione}
Tramite questi test si vuole stabilire se il prodotto finito possa essere accettato come soluzione alle reali necessità dell'utente finale. Viene definito anche test di collaudo, perché viene eseguito in un sistema identico o quanto più simile a quello reale, in collaborazione con il \glo{committente}. Vengono testate le funzionalità principali per attestare il soddisfacimento dei \glo{requisiti} richiesti. L'obiettivo non è più andare alla ricerca di \glo{bugs} o problemi, ma ottenere l'accettazione da parte del \glo{committente} per poter rilasciare il \glo{software}.
   	
\subsubsection{Metriche}
Non sono state attualmente individuate metriche specifiche al processo di \glo{validazione}.

\subsubsection{Strumenti}
Non sono stati attualmente individuati strumenti specifici al processo di \glo{validazione}.
