\subsection{Gestione dei cambiamenti}

\subsubsection{Descrizione}
Il processo di gestione dei cambiamenti definisce una \glo{procedura} di analisi e gestione dei problemi.
L'obiettivo del processo è fornire una risposta efficace alla rilevazione di un problema, garantendo inoltre la tracciabilità delle modifiche effettuate e dell'insorgenza del problema stesso.
Attraverso questo processo si intende inoltre tenere traccia degli errori più comuni, al fine di identificarne le fonti e mettere in atto meccanismi di prevenzione e di miglioramento.

\subsubsection{Attività}
\paragraph{Rilevazione del problema}
In seguito alla rilevazione di un problema significativo o alla ricezione di una segnalazione dovrà sempre essere aperta una \glo{issue} all'interno dell'\glo{issue tracking system}.
Una issue è caratterizzata da un codice che identifica, classifica e assegna la priorità alla \glo{issue} così strutturato:
\begin{center}
\textbf{C-[tipo]-[priorità]-[numero]}
\end{center}
dove:
\begin{itemize}
	\item \textbf{[tipo]} assume valore \textbf{P} se la issue riguarda il \textit{prodotto} o \textbf{A} se riguarda un \textit{processo};
	\item \textbf{[priorità]} assume valore \textbf{H}, \textbf{M} o \textbf{L} a seconda del fatto che la priorità con cui debba essere risolta sia alta (\textit{High}), media (\textit{Medium}) o bassa (\textit{Low});
	\item \textbf{[numero]} è un numero intero progressivo e parte da 1.
\end{itemize}
Ogni \glo{issue} deve inoltre essere fornita di una descrizione completa ed esaustiva.
La descrizione deve permettere a ogni membro del gruppo di capire la natura del problema insorto.
Nel caso in cui si ritenga che il problema rilevato abbia priorità elevata e necessiti di una rapida gestione, bisogna avvisare gli altri componenti del gruppo o le parti interessate tramite i canali di comunicazione interna.
Un componente del gruppo può prendere in carico una \glo{issue}, coerentemente con il proprio ruolo e le proprie competenze, notificandolo agli altri componenti tramite l'\glo{issue tracking system}.
Le \glo{issue} aperte verranno visionate durante gli incontri interni, al fine di monitorare l'andamento del progetto.

\paragraph{Risoluzione del problema}
I componenti del gruppo che prenderanno in carico la gestione del problema dovranno sfruttare il sistema di collegamento tra \glo{issue} e \glo{commit} integrato nell'\glo{issue tracking system}, al fine di tracciare le modifiche effettuate in risposta ad una segnalazione. 
In seguito alla risoluzione del problema si completerà la gestione dei cambiamenti tramite la chiusura dell'\glo{issue} e la storicizzazione dei cambiamenti adottati per ogni issue insorta.

\subsubsection{Metriche}
Non sono state attualmente individuate metriche specifiche al processo di gestione dei cambiamenti.
	
\subsubsection{Strumenti}
Per il processo di gestione dei cambiamenti si è deciso di sfruttare l'\glo{issue tracking system} integrato in \glo{GitHub} in quanto facilita il collegamento tra \glo{issue} e \glo{commit}, consentendo inoltre una facile identificazione dei componenti del \glo{team} coinvolti.