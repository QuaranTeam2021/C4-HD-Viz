\subsection{Scopo del documento}
Questo documento ha lo scopo di definire le regole base che tutti i membri del \glo{team} \textit{QuaranTeam} sono tenuti a seguire durante lo sviluppo del progetto, al fine di garantire un materiale uniforme e conforme alle normative. Le regole stabilite e discusse nel corpo di questo documento fanno riferimento a tutti i documenti e agli strumenti scelti per lo sviluppo del \glo{software}. 
Si utilizzerà un approccio incrementale, partendo dalla stesura del documento nei suoi punti principali e andando progressivamente a normare tutte le decisioni concordate dal \glo{team}. Tutti i membri del \glo{team} si impegnano a prendere visione del documento e a seguire le regole stabilite favorendo la coesione tra i membri.

\subsection{Scopo del prodotto}
Il \glo{capitolato} C4 ha per obiettivo la creazione di un'applicazione web chiamata \textit{HD Viz} sviluppata principalmente in \glo{HTML}, \glo{CSS}, \glo{JavaScript} e con l'utilizzo della \glo{libreria} \glo{D3.js}. La \glo{piattaforma} necessita anche di una parte \glo{server} che supporti il trasporto della parte \glo{HTML} verso il browser e l'esecuzione di \glo{query} su un \glo{database} \glo{SQL} o \glo{NoSQL}, realizzabile in \glo{Java} con \glo{server} \glo{Tomcat} o in \glo{JavaScript} con \glo{server} \glo{Node.js}. 
L'applicazione ha lo scopo di fornire una visualizzazione di dati, il cui contesto è sconosciuto, a supporto della fase esplorativa dell'analisi dei dati.
\textit{HD Viz} dovrà essere in grado di rappresentare dati in almeno 15 dimensioni e fornire minimo 4 diversi tipi di visualizzazione.

\subsection{Glossario}
Viene fornito il \Glossariov{v\versionGlossario{}}, una raccolta di tutti i termini con un significato particolare, che vengono definiti e descritti al fine di evitare ambiguità. In tutti i documenti, i termini definiti nel \Glossariov{v\versionGlossario{}} saranno identificati con una G a pedice. 

\subsection{Riferimenti}
\subsubsection{Riferimenti normativi}
\begin{itemize}
	\item \textbf{\glo{Capitolato}}\textbf{ d'appalto C4 - \textit{HD Viz}}: \\
		\url{https://www.math.unipd.it/~tullio/IS-1/2020/Progetto/C4.pdf}.
\end{itemize}
	
\subsubsection{Riferimenti informativi}
\begin{itemize}
	\item \textbf{Standard ISO/IEC 12207:1995}: \\
		\url{https://www.math.unipd.it/~tullio/IS-1/2009/Approfondimenti/ISO_12207-1995.pdf};
	\item \textbf{ISO 8601:2000}: \\
		\url{http://lists.ebxml.org/archives/ebxml-core/200104/pdf00005.pdf};
	
	\item \textbf{Slide del corso di Ingegneria del Software - Analisi dei requisiti}: \\
		\url{https://www.math.unipd.it/~tullio/IS-1/2020/Dispense/L07.pdf};
	
	\item \textbf{Documentazione \LaTeX{}}\ped{\textit{G}}: \\
		\url{https://www.latex-project.org/help/documentation/};

	\item \textbf{Spunti di studio per la Flipped Classroom sul tema della Documentazione}: \\
		\url{https://www.math.unipd.it/~tullio/IS-1/2020/Dispense/FC2.pdf};

	\item \textbf{Software Documentation, Ian Sommerville}: \\
		\url{www.literateprogramming.com/documentation.pdf};
\end{itemize}
