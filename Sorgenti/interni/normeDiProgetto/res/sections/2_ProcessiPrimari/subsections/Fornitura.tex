\subsection{Fornitura}

\subsubsection{Descrizione}
Il processo di fornitura ha lo scopo di individuare le norme che il gruppo dovrà rispettare per la realizzazione del prodotto. Il fornitore redige un documento chiamato \textit{Studio di Fattibilità}, nel quale vengono indicati e valutati i rischi della richiesta d'appalto. Il gruppo ha l'incarico di dialogare con il \glo{proponente} \textit{Zucchetti S.p.A.} e con i \glo{committenti} per discutere le richieste del progetto, redigendo in un \textit{Piano di Progetto} le attività che intende svolgere per la realizzazione del progetto.
Le aspettative sono quelle di ottenere un prodotto che soddisfi le richieste del \glo{proponente}. Per ottenere ciò il gruppo deve rimanere in stretto contatto con quest'ultimo, per definire i \glo{requisiti} e chiarire eventuali dubbi sullo sviluppo del prodotto.

\subsubsection{Attività}

\paragraph{Studio di Fattibilità}
L'obiettivo dello \textit{Studio di Fattibilità} è analizzare in dettaglio ogni \glo{capitolato} proposto dal \glo{proponente}. Nel documento vengono evidenziati gli aspetti positivi e negativi di ogni \glo{capitolato}.
Per ogni \glo{capitolato} vengono presentate le seguenti sezioni:
\begin{itemize}
	\item[•]  \textbf{Informazioni generali:} vengono indicati il nome, il \glo{proponente} e il \glo{committente};
	\item 	\textbf{Descrizione:} viene redatto un breve riassunto sulle caratteristiche del prodotto da realizzare;
	\item[•]  \textbf{Obiettivi:} vengono descritte le attività che il gruppo deve svolgere per il raggiungimento dell'obiettivo;
	\item[•]  \textbf{Tecnologie utilizzate:} viene stilata una lista delle tecnologie per la realizzazione del progetto;
	\item[•]  \textbf{Vincoli del progetto:} vengono elencati i \glo{requisiti} obbligatori e opzionali;
	\item[•]  \textbf{Valutazione finale:} vengono descritte le note positive e negative del \glo{capitolato} espresse dal gruppo;
	\item[•]  \textbf{Esito:} viene fatta una valutazione generale del \glo{capitolato}, motivando le ragioni per le quali si è deciso di intraprendere o meno la sua realizzazione.
\end{itemize}

\paragraph{Preparazione della risposta} 
Verrà scritta una lettera di presentazione per candidare il gruppo alla fornitura del prodotto relativo al \glo{capitolato} scelto.

\paragraph{Contrattazione}
In seguito ad un preventivo redatto nel \textit{Piano di Progetto} e una definizione dei \glo{casi d'uso} redatti nell'\textit{Analisi dei Requisiti}, il gruppo stipula il contratto con il \glo{committente}, impegnandosi a rispettare le sue richieste.

\paragraph{Piano di Progetto}
Il responsabile, con l'aiuto degli amministratori, redige un \textit{Piano di Progetto} a cui attenersi durante la realizzazione del progetto. Questo documento è caratterizzato dalle seguenti sezioni: 
\begin{itemize}
	\item[•] \textbf{Analisi dei rischi:} in questa sezione si analizzano le difficoltà in cui si potrebbe incorrere durante lo svolgimento del progetto e si analizza in quale modo verranno affrontate tali complicazioni;
	\item[•] \textbf{Modello di sviluppo:} si indica il modello di sviluppo che verrà adottato per l'esecuzione e il rilascio del progetto;
	\item[•] \textbf{Pianificazione:} vengono definite le attività, con relative scadenze, che caratterizzano le varie fasi di realizzazione del progetto;
	\item[•] \textbf{Preventivo e Consuntivo:} viene stipulato un preventivo totale per la realizzazione del progetto. Viene rendicontato il risultato del periodo di attività esprimendo un'analisi su quello che è stato preventivato in anticipo;
	\item[•] \textbf{Organigramma:} viene esposta la composizione del gruppo;
	\item[•] \textbf{Attualizzazione dei rischi:} viene presentato un riepilogo dei rischi riscontrati durante lo sviluppo del progetto. Per ciascuno di essi devono essere indicate le contromisure adottate. La sezione deve contenere una valutazione dell'efficacia di quest'ultime.
\end{itemize}

\paragraph{Piano di Qualifica}
Questo documento, redatto dai verificatori, viene realizzato per definire le soluzioni da adottare al fine di garantire la qualità del prodotto. Le attività definite nel \textit{Piano di Qualifica} hanno l'obiettivo di guidare ogni membro del gruppo nella realizzazione corretta del prodotto \glo{software}. Le attività definite sono le seguenti:
\begin{itemize}
\item[•] \textbf{Qualità di processo:} vengono definiti gli intervalli ottimali e accettabili entro i quali possono variare le metriche di processo;
\item[•] \textbf{Qualità di prodotto:} vengono definiti gli intervalli ottimali e accettabili entro i quali possono variare le metriche di prodotto;
\item[•] \textbf{Strategia di testing:} viene definito il modello con cui viene testato il prodotto per garantirne la corretta implementazione, permettendo il soddisfacimento di tutti i \glo{requisiti} richiesti;
\item[•] \textbf{Standard di qualità adottati:} vengono identificati gli standard di qualità adottati per lo sviluppo del prodotto;
\item[•] \textbf{Resoconto delle attività di verifica:} viene riportato un resoconto delle attività che sono state svolte.
\end{itemize}

\paragraph{Revisione e valutazione}
Il gruppo dovrà valutare le caratteristiche del prodotto per garantire la conformità con quanto deciso con il \glo{proponente} e il \glo{committente}. Durante la realizzazione il gruppo si impegnerà a coordinare le revisioni delle attività svolte. 

\paragraph{Consegna e completamento}
Al termine dello sviluppo del \glo{software} il gruppo consegna il prodotto, come stipulato nel contratto.

\subsubsection{Metriche}
In questo processo non è stata rilevata nessuna metrica da poter utilizzare.

\subsubsection{Strumenti}
Di seguito sono descritti gli strumenti identificati dal gruppo per il processo di fornitura.
\begin{itemize}
	\item \textbf{Google Sheets:} utilizzato per la realizzazione di aerogrammi, istogrammi, tabelle e grafici;
	\item \textbf{Microsoft Project:} \glo{software} per effettuare lo scheduling delle attività, utilizzato per la modellazione dei diagrammi di Gantt.
\end{itemize}
