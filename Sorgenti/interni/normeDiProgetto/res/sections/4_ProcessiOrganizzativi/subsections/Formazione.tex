\subsection{Formazione}
\subsubsection{Descrizione}
Il processo di formazione ha lo scopo di formare ogni membro del gruppo in modo che abbia le conoscenze adatte per svolgere le attività e utilizzare le tecnologie richieste. La formazione avviene in forma individuale ma il \glo{proponente} si rende disponibile per eventuali chiarimenti. Ogni membro del gruppo provvede allo studio delle tecnologie utilizzate nel progetto. Ogni persona si documenta sui linguaggi e sugli strumenti di programmazione che vengono utilizzati, operando secondo il principio del miglioramento continuo.
\subsubsection{Attività}
\paragraph{Materiale per la formazione}
Ogni membro del gruppo è autorizzato a consultare tutto il materiale che risulta necessario all'apprendimento. Viene inoltre utilizzato un canale \glo{Discord} per la condivisione di materiali e link utili, formando una base di riferimenti e conoscenze comuni che permettano una migliore collaborazione.

\begin{tabFormazione}{Tecnologie e linguaggi di programmazione utilizzati}{Tabella delle tecnologie e linguaggi di programmazione utilizzati}
	\href{https://www.latex-project.org/}{\glo{\LaTeX}} & Linguaggio di \glo{markup} per la preparazione di testi. \\ 
	\hline
	\href{https://www.javascript.com/}{\glo{JavaScript}} & Linguaggio di scripting orientato agli oggetti e agli eventi. \\ 
	\hline
	\href{https://d3js.org/}{\glo{D3.js}} & \glo{Libreria} \glo{JavaScript} per creare visualizzazioni dinamiche ed interattive. \\ 
	\hline
	\href{https://nodejs.org/it/}{\glo{Node.js}} & Runtime \glo{JavaScript} costruito sul motore \glo{JavaScript} V8 di Chrome. \\ 
	\hline
	\href{https://expressjs.com/it/}{\glo{Express.js}} & \glo{Framework} per applicazioni web \glo{Node.js}. \\ 
	\hline
	\href{https://it.reactjs.org/}{\glo{React}} & \glo{Libreria} \glo{JavaScript} per la creazione di interfacce utente interattive. \\ 
	\hline
	\href{https://mathjs.org/}{math.js} & \glo{Libreria} matematica per \glo{JavaScript} e \glo{Node.js}. \\ 
	\hline
	\href{https://mochajs.org/}{\glo{Mocha}} & \glo{Framework} di test \glo{JavaScript} eseguito su \glo{Node.js} e su browser per lo svolgimento di test asincroni. \\ 
	\hline
	\href{https://www.sonarqube.org/}{\glo{SonarQube}} & \glo{Piattaforma} utilizzata per il controllo della qualità del codice. \\ 
	\hline
	\href{https://eslint.org/}{\glo{ESLint}} & Strumento di analisi statica del codice. \\ 
\end{tabFormazione}

\subsubsection{Metriche}
Non sono state attualmente individuate metriche specifiche al processo di formazione.

\subsubsection{Strumenti}
Di seguito sono descritti gli strumenti identificati dal gruppo per il processo di formazione.
\begin{itemize}
	\item \textbf{\glo{Discord}:} utilizzo di un canale apposito per la condivisione di materiali e link utili, permette di avere una base di riferimenti comuni facilmente raggiungibili grazie ad un sistema di tag.
\end{itemize}
