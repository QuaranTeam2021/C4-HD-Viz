\subsection{Gestione organizzativa}

\subsubsection{Descrizione}
Il processo di gestione organizzativa descrive le scelte riguardanti la suddivisione e il coordinamento del lavoro all'interno del progetto.
Il fine principale di questo processo è fornire ai membri del gruppo un \textit{\PdP} che permetta di organizzare il lavoro con efficacia ed efficienza. Si tratta quindi di:
\begin{itemize}
    \item[•] identificare i rischi che si possono verificare durante lo sviluppo del progetto e definirne la gestione, creando un modello organizzativo;
    \item[•] definire un modello di sviluppo da seguire;
    \item[•] pianificare le attività da svolgere in base alle scadenze temporali;
    \item[•] calcolare un preventivo in termini di ore e costi suddiviso per ruoli;
    \item[•] effettuare un bilancio finale delle spese.
\end{itemize}
Queste attività sono raccolte nel \PdPv{v\versionPdP{}}, la cui redazione è in carico al responsabile di progetto con la collaborazione dell'amministratore.

\subsubsection{Attività}
\paragraph{Gestione dei ruoli di progetto}
Ogni membro del gruppo dovrà ricoprire a rotazione un ruolo corrispondente ad una figura aziendale. Ad ogni ruolo corrisponde un determinato profilo professionale che richiede competenza ed esperienza. 
\subparagraph{Responsabile} 
Il responsabile ha la responsabilità della pianificazione, del controllo, della gestione e del coordinamento di attività e risorse del progetto. Ricopre inoltre un ruolo di intermediario con il \glo{committente} e il \glo{proponente} del \glo{capitolato}. Si occupa principalmente di: 
\begin{itemize}
	\item[•] pianificare ed emanare scadenze;
	\item[•] approvare i documenti;
	\item[•] coordinare le risorse umane;
	\item[•] gestire e controllare le attività del gruppo;
	\item[•] gestire ed analizzare le criticità;
	\item[•] approvare l'offerta sottoposta al \glo{committente}. 
\end{itemize} 
\subparagraph{Amministratore}
L'amministratore ha il compito di gestione, supporto e controllo dell'ambiente di lavoro. Si occupa principalmente di: 
\begin{itemize}
	\item[•] controllare e gestire l'efficienza e l'operatività dell'\glo{ambiente di sviluppo};
	\item[•] dirigere le infrastrutture di supporto;
	\item[•] gestire la documentazione;
	\item[•] redigere le \textit{\NdP};
	\item[•] collaborare alla redazione del \textit{\PdP};
	\item[•] gestire la configurazione del prodotto;
	\item[•] redigere ed attuare i piani e le \glo{procedure} per la gestione della qualità.
\end{itemize} 
\subparagraph{Analista} 
L'analista è il responsabile delle attività di analisi dei problemi e del dominio applicativo. Egli non è presente all'interno del gruppo per tutta la durata del progetto in quanto si occupa solo delle attività svolte durante l'\textit{\AdR}. Si occupa principalmente di: 
\begin{itemize}
	\item[•] studio del dominio applicativo del progetto;
	\item[•] definire la complessità del problema;
	\item[•] definire i \glo{requisiti};
	\item[•] redigere l'\textit{\AdR} e lo \textit{\SdF}.
\end{itemize} 
\subparagraph{Progettista} 
Il progettista si occupa delle attività svolte durante la progettazione dell'architettura e la progettazione di dettaglio. Ha quindi competenze tecniche e tecnologiche. Si occupa principalmente di: 
\begin{itemize}
	\item[•] prendere decisioni efficienti ed efficaci sugli aspetti tecnici del progetto;
	\item[•] sviluppare l'architettura del prodotto in modo che sia efficiente, efficace e mantenibile. Utilizza apposite tecnologie note ed ottimizzate, individuate a partire dai \glo{requisiti} definiti dall'analista;
	\item[•] redigere la specifica tecnica, la definizione di prodotto e la parte pragmatica del \textit{\PdQ}.
\end{itemize} 
\subparagraph{Programmatore} 
Il programmatore è responsabile della \glo{codifica} del progetto e delle componenti che serviranno per effettuare le prove di verifica e validazione. Si occupa principalmente di: 
\begin{itemize}
	\item[•] implementare la specifica e le decisioni del progettista;
	\item[•] svolgere la parte di \glo{codifica} del prodotto; 
	\item[•] creare le componenti di supporto per la verifica e validazione del codice.
\end{itemize}  
\subparagraph{Verificatore} 
Il verificatore è il responsabile dell'attività di verifica dei prodotti di tutte le attività svolte dagli altri membri. È responsabile del controllo sia della documentazione che del codice. Ha il compito di segnalare eventuali errori affidandosi agli standard definiti nelle \textit{\NdP} e alla propria capacità di giudizio.
Si occupa principalmente di: 
\begin{itemize}
	\item[•] esaminare i prodotti in fase di revisione;
	\item[•] evidenziare eventuali errori;
	\item[•] segnalare eventuali errori all'autore dell'oggetto in fase di verifica.
\end{itemize}

\paragraph{Gestione delle comunicazioni}
Le comunicazioni possono svolgersi internamente ed esternamente. Le comunicazioni interne coinvolgono esclusivamente i membri del gruppo mentre quelle esterne coinvolgono anche \glo{proponente} e \glo{committente}.
\subparagraph{Comunicazioni interne}
Le comunicazioni interne sono gestite principalmente attraverso tre canali:
\begin{itemize}
    \item[•] \textbf{\glo{Telegram}:} utilizzando un gruppo privato nel quale è possibile scambiare rapidamente comunicazioni brevi ed informali coinvolgendo l'intero \glo{team}.
    \item[•] \textbf{\glo{Zoom}:} utilizzato per lo scambio di comunicazioni più consistenti e tecniche, come aggiornamenti sulle attività o discussioni legate a specifiche attività.
    \item[•] \textbf{\glo{GitHub}:} utilizzando l'\glo{Issue Tracking System} di \glo{GitHub} che consente di aggiornare rapidamente l'intero gruppo sull'avanzamento dei lavori.
\end{itemize}
\subparagraph{Comunicazioni esterne}
Le comunicazioni esterne sono gestite principalmente dal responsabile di progetto e avvengono attraverso \glo{Gmail}. Viene utilizzata la casella di posta elettronica creata appositamente per il gruppo \textit{\Gruppo}, con indirizzo \href{mailto:quaranteam2021@gmail.com}{quaranteam2021@gmail.com} per comunicare con il \glo{committente} e il \glo{proponente}.

\paragraph{Gestione degli incontri}
Gli incontri sono indetti dal responsabile, il quale ha il compito di:
\begin{itemize}
    \item[•] definire la data e l'orario degli incontri considerando la disponibilità dei partecipanti;
    \item[•] stabilire gli argomenti che verranno trattati durante l'incontro;
    \item[•] valutare le richieste relative agli incontri da parte dei componenti del gruppo e di eventuali soggetti esterni;
    \item[•] verificare ed approvare il verbale redatto dal segretario della riunione, il quale viene nominato ad inizio incontro.
\end{itemize}
I partecipanti sono tenuti ad essere puntuali agli incontri e, nel caso di imprevisti, comunicarli al responsabile. Le decisioni che vengono discusse durante gli incontri sono ritenute approvate nel caso di maggioranza da parte dei partecipanti. Al termine di ciascun incontro viene stilato un verbale da parte del segretario che si è preso carico di appuntare le decisioni prese, le informazioni che sono state scambiate e i programmi per le prossime attività. Tutti gli incontri che non prevedono decisioni non vengono verbalizzati. Vista e considerata la situazione relativa all'emergenza \glo{COVID-19} gli incontri, sia interni che esterni, si svolgeranno esclusivamente in modalità virtuale tramite video-conferenza. 
\subparagraph{Incontri interni}
Gli incontri interni sono aperti esclusivamente ai componenti del gruppo \textit{\Gruppo}. Affinché gli incontri siano ritenuti validi dovranno essere presenti almeno 5 componenti del gruppo. Gli incontri interni si svolgono in video-conferenza sulla piattaforma \glo{Zoom}.
\subparagraph{Incontri esterni}
Gli incontri esterni coinvolgono anche altri soggetti esterni, oltre ai membri del gruppo \textit{\Gruppo}. Data la presenza di persone esterne, questi incontri assumono una criticità maggiore, che deve essere opportunamente gestita dal responsabile tramite un'adeguata gestione delle comunicazioni che punti a mediare tra il gruppo e gli individui esterni. Ritardi e assenze da parte dei componenti del gruppo sono da evitare. Gli incontri esterni, come quelli interni, avvengono in modalità virtuale e lo strumento di comunicazione utilizzato è \glo{Skype}. Questo strumento è stato richiesto dal \glo{proponente} che si è reso disponibile a ricevere su appuntamento tutti i membri del gruppo, o in alternativa tutti i gruppi che concorrono per lo stesso \glo{capitolato} d'appalto.

\paragraph{Gestione degli strumenti di coordinamento}
È importante che tutti i componenti del gruppo sappiano, durante lo sviluppo del progetto, quali siano le attività da svolgere, quelle in corso e quelle già completate. Ognuno deve essere a conoscenza dei compiti a lui assegnati e della relativa data di scadenza per il loro completamento. Tutto ciò è utile per poter gestire in maniera efficace ed efficiente il proprio lavoro. Il responsabile deve poter assegnare nuovi compiti ai membri del gruppo e controllarne lo stato di avanzamento per verificarne la coerenza con la pianificazione. Per soddisfare queste necessità si è scelto di utilizzare l'\glo{Issue Tracking System} fornito da \glo{GitHub} per il \glo{repository} usato dal gruppo per il \glo{versionamento} del progetto. Questo strumento permette di creare delle attività assegnabili, ad una o più persone, attraverso le seguenti informazioni:
\begin{itemize}
    \item[•] \textbf{titolo:} nome del compito da eseguire; 
    \item[•] \textbf{descrizione:} descrizione dettagliata del compito da eseguire; 
    \item[•] \textbf{assegnatari:} persone a cui compete lo svolgimento del compito;
    \item[•] \textbf{bacheca:} spazio virtuale in cui il compito sarà monitorato;
    \item[•] \textbf{scadenza:} data ultima per lo svolgimento del compito. 
\end{itemize}
Ogni attività attraversa degli stati che permettono di monitorarne l'avanzamento.
\begin{figure}[H]
	\begin{center}
		\includegraphics[width=0.6\textwidth]{images/dashboard_attività_ITS_4.1.2.4.png} \\
		\caption{Stati di avanzamento di un'attività nell'Issue Tracking System}
	\end{center}
\end{figure} 
La scelta dell'\glo{Issue Tracking System} di \glo{GitHub} è data dalla possibilità di gestire il progetto in maniera più semplice e organizzata.

\paragraph{Gestione dei rischi}
Per la gestione dei rischi viene seguita la seguente \glo{procedura}.

\begin{figure}[H]
	\begin{center}
		\includegraphics[width=1.0\textwidth]{images/gestione_rischi_4.1.2.5.png} \\
		\caption{Procedura adottata per la gestione dei rischi}
	\end{center}
\end{figure} 
Questa \glo{procedura} viene applicata nel \PdPv{v\versionPdP{}}, dove viene riportata l'analisi e l'attualizzazione dei rischi. I rischi sono suddivisi principalmente in:
\begin{itemize}
	\item [•] \textbf{Rischi tecnologici:} riguardano i rischi legati allo studio, all'utilizzo e alle funzionalità delle tecnologie \glo{software} o hardware impiegate durante lo sviluppo del progetto;
	\item[•] \textbf{Rischi organizzativi:} riguardano i rischi legati all'organizzazione generale dello sviluppo e alla gestione del progetto. Sono inclusi i rischi legati al coordinamento dei membri del gruppo e alla gestione dei costi, delle risorse e delle scadenze temporali;
	\item[•] \textbf{Rischi interpersonali:} riguardano i rischi legati ai rapporti tra i membri del gruppo e alla comunicazione interna ed esterna;
	\item[•] \textbf{Rischi legati ai requisiti:} riguardano i rischi dovuti a un'analisi dei requisiti parzialmente scorretta, incompleta o ambigua.
\end{itemize}
La classificazione dei rischi prevede le seguenti caratteristiche:
\begin{itemize}
	\item [•] \textbf{Occorrenza:} indica una stima approssimativa dell'occorrenza di un rischio e può essere:
	\begin{itemize}
		\item [--] Alta;
		\item [--] Media;
		\item [--] Bassa.
	\end{itemize}
	\item[•] \textbf{Gravità:} indica il grado di pericolosità di un rischio e può essere:
	\begin{itemize}
		\item [--] Alta;
		\item [--] Media;
		\item [--] Bassa.
	\end{itemize}
\end{itemize}

\subsubsection{Metriche}
\paragraph{QP-4 Gestione organizzativa}

\begin{tabmetriche1}{Metriche di processo relative al processo di gestione organizzativa}{Tabella delle metriche per il processo di gestione organizzativa}
	QM-PROC-7 & Budget at completion (BAC) & È un numero intero che rappresenta il budget inizialmente allocato per la realizzazione del progetto. & \T\\
	\hline
	QM-PROC-8 & Estimated at completion (EAC) & È un numero intero che rappresenta il budget stimato per la realizzazione del progetto, aggiornato allo stato attuale. & $AC+ETC$ \T\\
	\hline
	QM-PROC-9 & Estimate to complete (ETC) & È un numero intero che rappresenta il budget stimato per la realizzazione delle rimanenti attività necessarie al completamento del progetto. & \T\\
	\hline
	QM-PROC-10 & Actual cost (AC) & È un numero intero che rappresenta il denaro speso fino al momento del calcolo. & \T\\
	\hline
	QM-PROC-11 & Earned value (EV) & È un numero intero che rappresenta il valore totale del lavoro fatto fino al momento del calcolo. & $\textit{\%lavoro completato}\times BAC$ \\
	\hline
	QM-PROC-12 & Planned value (PV) & È un numero intero che rappresenta il denaro che si dovrebbe aver guadagnato in quel momento. & $\textit{\%lavoro pianificato}\times BAC$ \\
	\hline
	QM-PROC-13 & Schedule variance (SV) & È un numero intero che indica lo stato dello svolgimento del progetto rispetto alla pianificazione. & $EV-PV$ \\
	\hline
	QM-PROC-14 & Cost variance (CV) & È un numero intero che indica il livello di efficienza nello sviluppo del progetto, rispetto a quanto pianificato & $EV-AC$ \\
	\hline
	QM-PROC-15 & Rischi non preventivati (RNP) & È un numero intero utilizzato per tracciare in modo incrementale i nuovi rischi che non erano precedentemente preventivati e che si presentano durante una fase del progetto. & Si incrementa di 1 unità per ogni rischio non preventivato individuato. \\
	\hline
\end{tabmetriche1}

\subsubsection{Strumenti}
Di seguito sono descritti gli strumenti identificati dal gruppo per il processo organizzativo.
\begin{itemize}
	\item[•] \textbf{\glo{Telegram}}: Utilizzato per lo scambio di messaggi informali quali avvisi, domande, consulti e confronti. Utilizzato anche per la comunicazione relativa agli incontri;
	\item[•] \textbf{\glo{Gmail}}: Utilizzato per comunicare ufficialmente con il \glo{committente} ed il \glo{proponente};
	\item[•] \textbf{\glo{Google Drive}}: Utilizzato per caricare e visualizzare documenti disponibili a tutti i membri del gruppo in ogni momento e per la stesura di materiale soggetto a molti cambiamenti;
	\item[•] \textbf{\glo{Zoom}}: Utilizzato per gli incontri a distanza tra i membri del gruppo e con il \glo{committente};
	\item[•] \textbf{\glo{Skype}}: Utilizzato per gli incontri a distanza con il \glo{proponente};
	\item[•] \textbf{\glo{GitHub}}: Utilizzato per il \glo{versionamento} e il salvataggio in remoto dei file riguardanti il progetto. Sfruttato inoltre l'\glo{Issue Tracking System} integrato, utilizzato per la gestione e il coordinamento dei compiti da svolgere;
	\item[•] \textbf{\glo{Discord}}: Utilizzato per lo scambio di messaggi tra i membri del gruppo. Facilita la comunicazione tra sottogruppi interni al \glo{team} sfruttando la possibilità di creare vari canali dedicati a temi specifici. Consente un'organizzazione più ordinata dei messaggi ed una comunicazione più veloce ed efficace.
\end{itemize}