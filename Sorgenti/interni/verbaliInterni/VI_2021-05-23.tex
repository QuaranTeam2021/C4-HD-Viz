\newcommand{\Data}{2021-05-23}
\newcommand{\TitoloDoc}{V.I. del \Data}
\newcommand{\Redattori}{\SIM}
\newcommand{\Verificatori}{\GIA}
\newcommand{\Approvatore}{\COF}
\newcommand{\Distribuzione}{\Committente{} \\& \Gruppo{}}
\newcommand{\Uso}{Interno}
\newcommand{\Stato}{Approvato}
\newcommand{\DescrizioneDoc}{Riassunto della riunione del gruppo \textit{\Gruppo} tenutasi il \Data.}
\newcommand{\pathimg}{../../immagini}
\newcommand{\VersioneDoc}{1.0.0}

% info generali 
\newcommand{\NomeProgetto}{HD Viz}

% fornitore
\newcommand{\Gruppo}{QuaranTeam}
\newcommand{\Mail}{quaranteam2021@gmail.com}

% committenti
\newcommand{\Committente}{\VT{}\\& \CR{}}
\newcommand{\VT}{Prof. Vardanega Tullio}
\newcommand{\CR}{Prof. Cardin Riccardo}

% proponenti
\newcommand{\Proponente}{Zucchetti S.p.A.}
\newcommand{\PG}{Piccoli Gregorio}

% QuaranTeam member
\newcommand{\CHF}{Chiarello Federico}
\newcommand{\COF}{Consalvo Federico}
\newcommand{\GIA}{Gibellato Alice}
\newcommand{\MAD}{Mason Damiano}
\newcommand{\REL}{Rech Elia}
\newcommand{\SIM}{Sinigaglia Matteo}
\newcommand{\VEL}{Veronese Luca}

% ruoli
\newcommand{\Responsabile}{Responsabile di Progetto}
\newcommand{\Amministratore}{Amministratore di Progetto}

% documenti
\newcommand{\SdF}{Studio di Fattibilità}
\newcommand{\SdFv}[1]{\textit{Studio di Fattibilità {#1}}}
\newcommand{\PdQ}{Piano di Qualifica}
\newcommand{\PdQv}[1]{\textit{Piano di Qualifica {#1}}}
\newcommand{\PdP}{Piano di Progetto}
\newcommand{\PdPv}[1]{\textit{Piano di Progetto {#1}}}
\newcommand{\NdP}{Norme di Progetto}
\newcommand{\NdPv}[1]{\textit{Norme di Progetto {#1}}}
\newcommand{\AdR}{Analisi dei Requisiti}
\newcommand{\AdRv}[1]{\textit{Analisi dei Requisiti {#1}}}
\newcommand{\Glossario}{Glossario}
\newcommand{\Glossariov}[1]{\textit{Glossario {#1}}}
\newcommand{\MM}{Manuale Manutentore}
\newcommand{\MMv}[1]{\textit{Manuale Manutentore {#1}}}
\newcommand{\MU}{Manuale Utente}
\newcommand{\MUv}[1]{\textit{Manuale Utente {#1}}}

% comandi generali
\newcommand{\glo}[1]{#1\textsubscript{\textit{G}}}
\newcommand{\qm}[1]{``#1''}

\newcommand{\defaultfooter}[1]{
	\rowcolor{white}
	\multicolumn{#1}{|c|}{\textit{La tabella continua a pagina seguente.}}\\
    \hline
    \endfoot
    \endlastfoot
}


\newcommand{\versionMU}{2.0.0}
\newcommand{\versionMM}{2.0.0}
\newcommand{\versionSdF}{1.0.0}
\newcommand{\versionPdQ}{4.0.0}
\newcommand{\versionPdP}{4.0.0}
\newcommand{\versionNdP}{3.0.0}
\newcommand{\versionAdR}{3.0.0}
\newcommand{\versionGlossario}{4.0.0}

\documentclass{../../Utility/stdDocument}

\begin{document}
\verbaleFronte
\section*{Registro delle modifiche} 
\begin{RegistroModifiche}
	1.0.0 & 2021-05-24 & Approvazione del documento. & \COF & \\
	\hline
	0.1.0 & 2021-05-23 & Stesura del documento. & \SIM & \GIA \\
\end{RegistroModifiche}

\verbaleIndice

\section{Informazioni generali}
\begin{itemize}
	\item[•] \textbf{Luogo:} \glo{Zoom};
	\item[•] \textbf{Data:} 2021-05-23;
	\item[•] \textbf{Ora di inizio:} 18:30;
	\item[•] \textbf{Ora di fine:} 19:30;
	\item[•] \textbf{Partecipanti del gruppo:}
	\begin{enumerate}
		\item [-] Alice Gibellato;
		\item [-] Elia Rech;
		\item [-] Federico Consalvo; 
		\item [-] Federico Chiarello; 
		\item [-] Matteo Sinigaglia;
		\item [-] Luca Veronese;
		\item [-] Damiano Mason.
	\end{enumerate}
	\item[•] \textbf{Segretario:} \COF. 
\end{itemize}

\section{Ordine del giorno}
\begin{itemize}
	\item Suddivisione del lavoro in seguito alla Revisione di Qualifica;
	\item Segnalazione errori nel calcolo delle metriche;
	\item Richiesta di un colloquio con il \glo{proponente};
	\item Organizzazione delle attività in vista del collaudo.
\end{itemize}

\section{Resoconto}
\subsection{Suddivisione del lavoro in seguito alla Revisione di Qualifica}
In seguito alla ricezione dell'esito della Revisione di Qualifica, accolto dal \glo{team} con entusiasmo ma anche come spunto di miglioramento, il gruppo si è riunito per fare il punto della situazione e per discutere delle segnalazioni effettuate dal \glo{committente}. Sono state assegnate le seguenti attività:
\begin{itemize}
	\item \SIM{} si occuperà di descrivere più compiutamente il diagramma delle classi;
	\item \REL{} provvederà ad aggiornare il Consuntivo nel documento \textit{Piano di Progetto} - §6;
	\item \MAD{} si occuperà di aggiornare i diagrammi di \glo{Gantt} in caso di un eventuale posticipo della consegna dei documenti in ingresso per la Revisione di Accettazione;
	\item \VEL{} si occuperà di sostituire opportunamente \glo{snapshot} di stralci di UI con \glo{snapshot} dell'interfaccia nella sua interezza nel \textit{Manuale Utente};
	\item \SIM{} si occuperà di redigere il Verbale Interno del 2021-05-23 e \GIA{} si occuperà di verificarlo.
\end{itemize}

\subsection{Segnalazione errori nel calcolo delle metriche}
Un membro del gruppo si è accorto che nel calcolare le metriche QM-PROC-5 e QM-PROC-6 non si è tenuto conto di una parte del codice prodotto. Per questo, i risultati indicati negli ultimi incrementi della fase di progettazione di dettaglio e codifica non coprono tutto il codice. È stato quindi proposto di abbassare la soglia delle due metriche, in quanto il codice relativo alla vista occupa tante righe e conseguentemente fa abbassare la percentuale della metrica QM-PROC-5 e condiziona anche la metrica QM-PROC-6. Inoltre, è stato incaricato un membro di riportare il risultato effettivo delle due metriche di riferimento nei grafici della fase di validazione e collaudo. 

\subsection{Richiesta di un colloquio con il \glo{proponente}}
Si è ritenuto opportuno richiedere un incontro con il \glo{proponente}, al fine di chiarire alcuni punti e mostrare il prodotto prima del collaudo, possibilmente in data antecedente alla festività del 2021-06-02.

\subsection{Organizzazione delle attività in vista del collaudo}
Il gruppo si prepara ad affrontare il collaudo. I \glo{requisiti} obbligatori, desiderabili e facoltativi sono quasi tutti soddisfatti. Ad ogni componente sono stati assegnati degli specifici compiti:
\begin{itemize}
	\item \SIM{} e \CHF{} si occuperanno di eseguire test del modello e dei controller;
	\item \MAD{} e \GIA{} si occuperanno di eseguire test con \glo{React};
	\item \VEL{} e \REL{} proseguiranno ad implementare i requisiti opzionali dei grafici;
	\item \COF{} proseguirà a scrivere il \glo{CSS} necessario per la presentazione dell'applicazione.
\end{itemize} 


\section{Riepilogo delle decisioni}
\begin{RiepilogoDecisioni}
	VI\_2021-05-23.1 & Assegnazione attività della documentazione per la Revisione di Accettazione secondo quanto indicato in §3.1 e §3.3. \\ \hline
	VI\_2021-05-23.2 & \GIA{} Aggiorna la soglia accettabile delle metriche QM-PROC-4 e QM-PROC-5 e ricalcola il risultato finale della fase di progettazione di dettaglio e codifica. \\ \hline
	VI\_2021-05-23.3 & Richiesta di un colloquio con il \glo{proponente}. \\ \hline
	VI\_2021-05-23.4 & Suddivisione dei compiti per il collaudo. \\
\end{RiepilogoDecisioni}
\end{document}