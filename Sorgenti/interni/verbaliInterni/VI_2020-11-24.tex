\newcommand{\Data}{2020-11-24}
\newcommand{\TitoloDoc}{V.I. del \Data}
\newcommand{\Redattori}{\COF}
\newcommand{\Verificatori}{\GIA}
\newcommand{\Approvatore}{\SIM}
\newcommand{\Distribuzione}{\Committente{} \\& \Gruppo{}}
\newcommand{\Uso}{Interno}
\newcommand{\Stato}{Approvato}
\newcommand{\DescrizioneDoc}{Riassunto della riunione del gruppo \textit{\Gruppo} tenutasi il \Data.}
\newcommand{\pathimg}{../../immagini}
\newcommand{\VersioneDoc}{1.0.0}

% info generali 
\newcommand{\NomeProgetto}{HD Viz}

% fornitore
\newcommand{\Gruppo}{QuaranTeam}
\newcommand{\Mail}{quaranteam2021@gmail.com}

% committenti
\newcommand{\Committente}{\VT{}\\& \CR{}}
\newcommand{\VT}{Prof. Vardanega Tullio}
\newcommand{\CR}{Prof. Cardin Riccardo}

% proponenti
\newcommand{\Proponente}{Zucchetti S.p.A.}
\newcommand{\PG}{Piccoli Gregorio}

% QuaranTeam member
\newcommand{\CHF}{Chiarello Federico}
\newcommand{\COF}{Consalvo Federico}
\newcommand{\GIA}{Gibellato Alice}
\newcommand{\MAD}{Mason Damiano}
\newcommand{\REL}{Rech Elia}
\newcommand{\SIM}{Sinigaglia Matteo}
\newcommand{\VEL}{Veronese Luca}

% ruoli
\newcommand{\Responsabile}{Responsabile di Progetto}
\newcommand{\Amministratore}{Amministratore di Progetto}

% documenti
\newcommand{\SdF}{Studio di Fattibilità}
\newcommand{\SdFv}[1]{\textit{Studio di Fattibilità {#1}}}
\newcommand{\PdQ}{Piano di Qualifica}
\newcommand{\PdQv}[1]{\textit{Piano di Qualifica {#1}}}
\newcommand{\PdP}{Piano di Progetto}
\newcommand{\PdPv}[1]{\textit{Piano di Progetto {#1}}}
\newcommand{\NdP}{Norme di Progetto}
\newcommand{\NdPv}[1]{\textit{Norme di Progetto {#1}}}
\newcommand{\AdR}{Analisi dei Requisiti}
\newcommand{\AdRv}[1]{\textit{Analisi dei Requisiti {#1}}}
\newcommand{\Glossario}{Glossario}
\newcommand{\Glossariov}[1]{\textit{Glossario {#1}}}
\newcommand{\MM}{Manuale Manutentore}
\newcommand{\MMv}[1]{\textit{Manuale Manutentore {#1}}}
\newcommand{\MU}{Manuale Utente}
\newcommand{\MUv}[1]{\textit{Manuale Utente {#1}}}

% comandi generali
\newcommand{\glo}[1]{#1\textsubscript{\textit{G}}}
\newcommand{\qm}[1]{``#1''}

\newcommand{\defaultfooter}[1]{
	\rowcolor{white}
	\multicolumn{#1}{|c|}{\textit{La tabella continua a pagina seguente.}}\\
    \hline
    \endfoot
    \endlastfoot
}


\newcommand{\versionMU}{2.0.0}
\newcommand{\versionMM}{2.0.0}
\newcommand{\versionSdF}{1.0.0}
\newcommand{\versionPdQ}{4.0.0}
\newcommand{\versionPdP}{4.0.0}
\newcommand{\versionNdP}{3.0.0}
\newcommand{\versionAdR}{3.0.0}
\newcommand{\versionGlossario}{4.0.0}

\documentclass{../../Utility/stdDocument}

\begin{document}
\verbaleFronte
\section*{Registro delle modifiche} 
\begin{RegistroModifiche}
	1.0.0 & 2020-11-29 & Approvazione del documento. & \SIM &  Responsabile \\
	\hline
	0.1.1 & 2020-11-27 & Verifica del documento. & \GIA &  Verificatore \\
	\hline
	0.1.0 & 2020-11-24 & Stesura del documento. & \COF &  Analista \\
\end{RegistroModifiche}

\verbaleIndice

\section{Informazioni generali}
\begin{itemize}
	\item[•] \textbf{Luogo:} \glo{Zoom};
	\item[•] \textbf{Data:} 2020-11-24;
	\item[•] \textbf{Ora di inizio:} 18:00;
	\item[•] \textbf{Ora di fine:} 20:45;
	\item[•] \textbf{Partecipanti del gruppo:}
	\begin{itemize}
		\item [-] Alice Gibellato;
		\item [-] Elia Rech;
		\item [-] Federico Consalvo; 
		\item [-] Federico Chiarello; 
		\item [-] Matteo Sinigaglia;
		\item [-] Luca Veronese;
		\item [-] Damiano Mason.
	\end{itemize}
	\item[•] \textbf{Segretario:} Federico Consalvo.
\end{itemize}

\section{Ordine del giorno}
\begin{itemize}
\item Nome, logo e indirizzo e-mail del gruppo;
\item Sondaggio di opinioni sui \glo{capitolati};
\item Organizzazione per la creazione del documento \textit{\SdF};
\item Preferenze dei \glo{capitolati};
\item Creazione cartella \glo{Google Drive} e \glo{repository} su \glo{GitHub};
\item Prossimo incontro.
\end{itemize}

\section{Resoconto}
\subsection{Nome, logo e indirizzo e-mail del gruppo}
Nell'intervallo di tempo tra la riunione scorsa e quella corrente, attraverso il canale di comunicazione interna, è stato scelto il nome ufficiale del gruppo "QuaranTeam". Di conseguenza sono stati creati l'indirizzo e-mail "quaranteam2021@gmail.com", per la comunicazione esterna, e il logo che viene riportato in ogni documento.
\subsection{Sondaggio di opinioni sui \glo{capitolati}}
Durante l'incontro sono state raccolte le opinioni che ciascun membro si è fatto sulle varie proposte di \glo{capitolato}. Viene quindi riportato un breve parere collettivo riguardo ogni \glo{capitolato}.
\begin{enumerate}
	\item[•] C1: I membri del gruppo non hanno mostrato particolare interesse per i temi affrontati dal \glo{capitolato} e per il \glo{proponente}. 
	\item[•] C2: Opinioni nel complesso positive. \glo{Capitolato} all'apparenza non troppo difficile, ma non entusiasma tutti i membri del gruppo. L'azienda risulta essere molto disponibile e non predispone particolari vincoli per il \glo{capitolato}.
	\item[•] C3: Nonostante il \glo{capitolato} permetta l'utilizzo di librerie già esistenti, non sembra essere molto semplice, i temi affrontati entusiasmano però molto il gruppo. L'azienda è molto diffusa ed interessa ad alcuni membri come contatto post-laurea. 
	\item[•] C4: Un membro del gruppo si è molto interessato a questo \glo{capitolato} e ha compreso a fondo cosa viene chiesto di fare. Il seminario è apparso molto positivo e le tecnologie richieste sono già parzialmente conosciute (\glo{HTML} e \glo{CSS}). Richiede inoltre l'utilizzo della \glo{libreria} \glo{JavaScript} \glo{D3.js}, che ha suscitato l'interesse dei membri del gruppo.
	\item[•] C5: I temi affrontati da questo \glo{capitolato} hanno suscitato molto interesse tra i componenti del gruppo, sono tuttavia sorte delle perplessità riguardo la complessità delle tecnologie utilizzate. 
	\item[•] C6: Il \glo{capitolato} ha riscontrato pareri contrastanti, ma l'approfondimento relativo alle tecnologie \glo{AWS} ha suscitato interesse nel gruppo. \glo{Capitolato} all'apparenza non troppo semplice. 
	\item[•] C7: Parere positivo riguardo l'utilizzo di \glo{Python} e \glo{Qt}, tecnologie che interessano molto al gruppo. Il tema affrontato non ha però suscitato particolare interesse.   
\end{enumerate} 
\subsection{Organizzazione per la creazione del documento \textit{\SdF}} 
Per quanto riguarda la creazione e lo sviluppo dei documenti da consegnare alla Revisione dei Requisiti si è deciso di partire dallo \textit{\SdF}. Per la realizzazione della prima bozza dello \textit{\SdF} abbiamo deciso di suddividerci i \glo{capitolati} in maniera randomica e ci siamo dati come limite 2 settimane per lo svolgimento della parte assegnata.
\paragraph*{Risultato assegnazione random:}
\begin{enumerate}
	\item[•] C1: Alice Gibellato;
	\item[•] C2: Federico Consalvo;
	\item[•] C3: Federico Chiarello; 
	\item[•] C4: Luca Veronese;
	\item[•] C5: Matteo Sinigaglia; 
	\item[•] C6: Elia Rech;
	\item[•] C7: Damiano Mason.
\end{enumerate} 
\subsection{Preferenze dei \glo{capitolati}} 
È stata utilizzata una tabella per rappresentare le preferenze di ogni membro. Nelle colonne si trovano i \glo{capitolati}, rappresentati dall'iniziale maiuscola C, mentre nelle righe sono presenti le posizioni di preferenza (da 1 a 7), rappresentate dall'iniziale maiuscola P. L'incrocio riga-colonna rappresenta la posizione per ogni singolo membro. \\
\begin{table}[htbp]
	\rowcolors{2}{oddcolour}{evencolour}
	\begin{tabular}{ |p{0.7cm}| p{0.4cm}| p{1.8cm}| p{3.1cm}| p{1.3cm}| p{1.2cm}| p{0.4cm}| p{1.6cm}| }
		\hline
		\rowcolor{headercolour}
		\diagbox[width=1.1cm, height=1cm]{P}{C} & C1 & C2 & C3 & C4 & C5 & C6 & C7 \\
		\hline
		\small 1 & \small & \small FC, MS & \small & \small ER, LV, F & \small AG, DM & \small & \small \\
		\hline		
		\small 2 & \small & \small AG, ER, DM & \small & \small FC & \small F & \small & \small LV, FC, MS \\
		\hline
		\small 3 & \small & \small LV & \small F & \small DM & \small MS & \small & \small AG, ER \\
		\hline
		\small 4 & \small & \small F & \small FC & \small MS, AG & \small LV, ER & \small & \small DM \\
		\hline
		\small 5 & \small & \small  & \small AG, ER, LV, MS, DM & \small & \small FC & \small & \small F \\
		\hline
		\small 6 & \small & \small & \small & \small & \small & \small T & \small \\
		\hline
		\small 7 & \small T & \small & \small & \small & \small & \small & \small \\
		\hline
	\end{tabular}
	\centering
	\caption*{Tabella 1: Tabella delle preferenze sui \glo{capitolati}}
	\raggedright
	\label*{
		\textbf{Legenda:} \\
		• T = Tutti;\\
		• AG = Alice Gibellato;\\
		• F = Federico Chiarello; \\
		• FC = Federico Consalvo; \\
		• DM = Damiano Mason \\
		• MS = Matteo Sinigaglia; \\
		• ER = Elia Rech; \\
		• LV = Luca Veronese.
	}
\end{table}
\newline
Analizzando i dati di questa tabella il \glo{capitolato} 4 rappresenta la prima scelta del gruppo, mentre il 2 e il 7 rispettivamente la seconda e la terza. 

\subsection{Creazione cartella \glo{Google Drive} e \glo{repository} su \glo{GitHub}}
Si è deciso di creare un \glo{repository} \glo{GitHub} per condividere il lavoro svolto. È stata creata anche una cartella condivisa su \glo{Google Drive} per fare delle bozze e sfruttare la praticità di Google Documenti durante le riunioni. Si è deciso inoltre di utilizzare \glo{\LaTeX} per la stesura dei documenti.
\subsection{Prossimo incontro}
Il prossimo incontro si svolgerà all'incirca tra due settimane, per permettere ai membri di prendere confidenza con gli \glo{ambienti di sviluppo} \glo{GitHub} e \glo{\LaTeX} e coordinare il lavoro di gruppo.

\section{Riepilogo delle decisioni}
\begin{RiepilogoDecisioni}
	VI\_2020-11-24.1 & Decisi il nome e il logo del gruppo. \\
	\hline
	VI\_2020-11-24.2 & Discussione e creazione \glo{repository} su \glo{GitHub}. \\
	\hline
	VI\_2020-11-24.3 & Decisione di adottare \glo{\LaTeX} per la stesura dei documenti. \\
	\hline
	VI\_2020-11-24.4 & Suddivisione iniziale dei ruoli per la stesura del documento \textit{\SdF}. \\
	\hline
	VI\_2020-11-24.5 & Riguardo la scelta del \glo{capitolato} d'appalto per cui concorrere si è deciso che la scelta finale verrà presa solo nel momento in cui i seminari tecnologici proposti saranno terminati. \\
\end{RiepilogoDecisioni}

\end{document}