\newcommand{\Data}{2021-01-16}
\newcommand{\TitoloDoc}{V.I. del \Data}
\newcommand{\Redattori}{\VEL}
\newcommand{\Verificatori}{\COF}
\newcommand{\Approvatore}{\CHF}
\newcommand{\Distribuzione}{\Committente{} \\& \Gruppo{}}
\newcommand{\Uso}{Interno}
\newcommand{\Stato}{Approvato}
\newcommand{\DescrizioneDoc}{Riassunto della riunione del gruppo \textit{\Gruppo} tenutasi il \Data.}
\newcommand{\pathimg}{../../immagini}
\newcommand{\VersioneDoc}{1.0.0}

% info generali 
\newcommand{\NomeProgetto}{HD Viz}

% fornitore
\newcommand{\Gruppo}{QuaranTeam}
\newcommand{\Mail}{quaranteam2021@gmail.com}

% committenti
\newcommand{\Committente}{\VT{}\\& \CR{}}
\newcommand{\VT}{Prof. Vardanega Tullio}
\newcommand{\CR}{Prof. Cardin Riccardo}

% proponenti
\newcommand{\Proponente}{Zucchetti S.p.A.}
\newcommand{\PG}{Piccoli Gregorio}

% QuaranTeam member
\newcommand{\CHF}{Chiarello Federico}
\newcommand{\COF}{Consalvo Federico}
\newcommand{\GIA}{Gibellato Alice}
\newcommand{\MAD}{Mason Damiano}
\newcommand{\REL}{Rech Elia}
\newcommand{\SIM}{Sinigaglia Matteo}
\newcommand{\VEL}{Veronese Luca}

% ruoli
\newcommand{\Responsabile}{Responsabile di Progetto}
\newcommand{\Amministratore}{Amministratore di Progetto}

% documenti
\newcommand{\SdF}{Studio di Fattibilità}
\newcommand{\SdFv}[1]{\textit{Studio di Fattibilità {#1}}}
\newcommand{\PdQ}{Piano di Qualifica}
\newcommand{\PdQv}[1]{\textit{Piano di Qualifica {#1}}}
\newcommand{\PdP}{Piano di Progetto}
\newcommand{\PdPv}[1]{\textit{Piano di Progetto {#1}}}
\newcommand{\NdP}{Norme di Progetto}
\newcommand{\NdPv}[1]{\textit{Norme di Progetto {#1}}}
\newcommand{\AdR}{Analisi dei Requisiti}
\newcommand{\AdRv}[1]{\textit{Analisi dei Requisiti {#1}}}
\newcommand{\Glossario}{Glossario}
\newcommand{\Glossariov}[1]{\textit{Glossario {#1}}}
\newcommand{\MM}{Manuale Manutentore}
\newcommand{\MMv}[1]{\textit{Manuale Manutentore {#1}}}
\newcommand{\MU}{Manuale Utente}
\newcommand{\MUv}[1]{\textit{Manuale Utente {#1}}}

% comandi generali
\newcommand{\glo}[1]{#1\textsubscript{\textit{G}}}
\newcommand{\qm}[1]{``#1''}

\newcommand{\defaultfooter}[1]{
	\rowcolor{white}
	\multicolumn{#1}{|c|}{\textit{La tabella continua a pagina seguente.}}\\
    \hline
    \endfoot
    \endlastfoot
}


\newcommand{\versionMU}{2.0.0}
\newcommand{\versionMM}{2.0.0}
\newcommand{\versionSdF}{1.0.0}
\newcommand{\versionPdQ}{4.0.0}
\newcommand{\versionPdP}{4.0.0}
\newcommand{\versionNdP}{3.0.0}
\newcommand{\versionAdR}{3.0.0}
\newcommand{\versionGlossario}{4.0.0}

\documentclass{../../Utility/stdDocument}

\begin{document}
	\verbaleFronte
	\section*{Registro delle modifiche} 
	\begin{RegistroModifiche}
		1.0.0 & 2021-01-20 & Approvazione del documento. & \CHF &  Responsabile \\
		\hline
		0.1.0 & 2021-01-17 & Stesura del documento. \newline 
		Verificatore: \COF & \VEL &  Analista \\
	\end{RegistroModifiche}
	
	\verbaleIndice
	
	\section{Informazioni generali}
	\begin{itemize}
		\item[•] \textbf{Luogo:} \glo{Zoom};
		\item[•] \textbf{Data:} 2021-01-16;
		\item[•] \textbf{Ora di inizio:} 18:30;
		\item[•] \textbf{Ora di fine:} 19:15;
		\item[•] \textbf{Partecipanti del gruppo:}
		\begin{enumerate}
			\item [-] Alice Gibellato;
			\item [-] Elia Rech;
			\item [-] Federico Consalvo; 
			\item [-] Federico Chiarello; 
			\item [-] Matteo Sinigaglia;
			\item [-] Luca Veronese;
			\item [-] Damiano Mason.
		\end{enumerate}
		\item[•] \textbf{Segretario:} Luca Veronese. 
	\end{itemize}
	
	\section{Ordine del giorno}
	\begin{itemize}
		\item Preventivo a finire;
		\item Classificazione dei requisiti.
	\end{itemize}
	
	\section{Resoconto}
	Il gruppo degli analisti si è reso conto che l'incremento I della fase di progettazione e \glo{codifica} della \glo{Technology Baseline} risulta essere più impegnativo di quanto preventivato. Dunque ha richiesto una riunione per proporre di apportare modifiche alla pianificazione iniziale. Inoltre il gruppo di analisti ha espresso dubbi sulla classificazione di alcuni requisiti.
	\subsection{Preventivo a finire}
	Dopo una discussione e confronto con i responsabili si è deliberato l'aumento di 14 unità del monte ore riservato all'incremento I. Per far fronte a ciò è stato decretata la riduzione di altrettante unità del monte ore riservato all'incremento II, con conseguente rimozione dell'implementazione di requisiti facoltativi nell'incremento II ("analisi automatiche sui dati").
	È stato quindi approvato il consuntivo di periodo della fase di consolidamento dei requisiti, con contestuale delibera alla stesura del preventivo a finire nella sezione §6.2.4 del \PdP{} che aggiorna i dati del preventivo iniziale (sezione §5 del \PdP{}).
	\subsection{Classificazione dei requisiti}
	In seguito a una rilettura dei resoconti delle riunioni precedenti è stata corretta la classificazione dell'obbligatorietà di alcuni requisiti. In particolare i requisiti RF-O-1.x sono stati classificati come "facoltativi", i requisiti RF-O-4.3 e RF-O-4.4 come "desiderabili".
	\section{Riepilogo delle decisioni}
	\begin{RiepilogoDecisioni}
		VI\_2021-01-16.1 & Rimozione delle "analisi automatiche" dagli obiettivi dell'incremento II della fase di Progettazione e \glo{codifica} della \glo{Technology Baseline}. \\ \hline  
		VI\_2021-01-16.2 & Aumento del monte ore destinato all'incremento I. \\ \hline 
		VI\_2021-01-16.3 & Riduzione del monte ore destinato all'incremento II. \\ \hline
		VI\_2021-01-16.4 & Delibera della stesura del consuntivo di periodo. \\ \hline
		VI\_2021-01-16.5 & Classificazione dei requisiti RF-O-1.x, RF-O-4.3 e RF-O-4.4. \\ \hline
	\end{RiepilogoDecisioni}
\end{document}