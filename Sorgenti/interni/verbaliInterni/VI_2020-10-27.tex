\newcommand{\Data}{2020-10-27}
\newcommand{\TitoloDoc}{V.I. del \Data}
\newcommand{\Redattori}{\GIA}
\newcommand{\Verificatori}{\REL}
\newcommand{\Approvatore}{\SIM}
\newcommand{\Distribuzione}{\Committente{} \\& \Gruppo{}}
\newcommand{\Uso}{Interno}
\newcommand{\Stato}{Approvato}
\newcommand{\DescrizioneDoc}{Riassunto della riunione del gruppo \textit{\Gruppo} tenutasi il \Data.}
\newcommand{\pathimg}{../../immagini}
\newcommand{\VersioneDoc}{1.0.0}

% info generali 
\newcommand{\NomeProgetto}{HD Viz}

% fornitore
\newcommand{\Gruppo}{QuaranTeam}
\newcommand{\Mail}{quaranteam2021@gmail.com}

% committenti
\newcommand{\Committente}{\VT{}\\& \CR{}}
\newcommand{\VT}{Prof. Vardanega Tullio}
\newcommand{\CR}{Prof. Cardin Riccardo}

% proponenti
\newcommand{\Proponente}{Zucchetti S.p.A.}
\newcommand{\PG}{Piccoli Gregorio}

% QuaranTeam member
\newcommand{\CHF}{Chiarello Federico}
\newcommand{\COF}{Consalvo Federico}
\newcommand{\GIA}{Gibellato Alice}
\newcommand{\MAD}{Mason Damiano}
\newcommand{\REL}{Rech Elia}
\newcommand{\SIM}{Sinigaglia Matteo}
\newcommand{\VEL}{Veronese Luca}

% ruoli
\newcommand{\Responsabile}{Responsabile di Progetto}
\newcommand{\Amministratore}{Amministratore di Progetto}

% documenti
\newcommand{\SdF}{Studio di Fattibilità}
\newcommand{\SdFv}[1]{\textit{Studio di Fattibilità {#1}}}
\newcommand{\PdQ}{Piano di Qualifica}
\newcommand{\PdQv}[1]{\textit{Piano di Qualifica {#1}}}
\newcommand{\PdP}{Piano di Progetto}
\newcommand{\PdPv}[1]{\textit{Piano di Progetto {#1}}}
\newcommand{\NdP}{Norme di Progetto}
\newcommand{\NdPv}[1]{\textit{Norme di Progetto {#1}}}
\newcommand{\AdR}{Analisi dei Requisiti}
\newcommand{\AdRv}[1]{\textit{Analisi dei Requisiti {#1}}}
\newcommand{\Glossario}{Glossario}
\newcommand{\Glossariov}[1]{\textit{Glossario {#1}}}
\newcommand{\MM}{Manuale Manutentore}
\newcommand{\MMv}[1]{\textit{Manuale Manutentore {#1}}}
\newcommand{\MU}{Manuale Utente}
\newcommand{\MUv}[1]{\textit{Manuale Utente {#1}}}

% comandi generali
\newcommand{\glo}[1]{#1\textsubscript{\textit{G}}}
\newcommand{\qm}[1]{``#1''}

\newcommand{\defaultfooter}[1]{
	\rowcolor{white}
	\multicolumn{#1}{|c|}{\textit{La tabella continua a pagina seguente.}}\\
    \hline
    \endfoot
    \endlastfoot
}


\newcommand{\versionMU}{2.0.0}
\newcommand{\versionMM}{2.0.0}
\newcommand{\versionSdF}{1.0.0}
\newcommand{\versionPdQ}{4.0.0}
\newcommand{\versionPdP}{4.0.0}
\newcommand{\versionNdP}{3.0.0}
\newcommand{\versionAdR}{3.0.0}
\newcommand{\versionGlossario}{4.0.0}

\documentclass{../../Utility/stdDocument}

\begin{document}
\verbaleFronte
\begin{RegistroModifiche}
	1.0.0 & 2020-10-30 & Approvazione del documento. & \SIM &  Responsabile \\
	\hline
	0.1.1 & 2020-10-29 & Verifica del documento. & \REL &  Verificatore \\
	\hline
	0.1.0 & 2020-10-27 & Stesura del documento. & \GIA &  Analista \\
\end{RegistroModifiche}

\verbaleIndice

\section{Informazioni generali}
A causa delle circostanze poco favorevoli dovute alla pandemia \glo{COVID-19} le riunioni interne verranno esclusivamente svolte online via \glo{Zoom}.
\begin{itemize}
	\item[•] \textbf{Luogo:} \glo{Zoom};
	\item[•] \textbf{Data:} 2020-10-27;
	\item[•] \textbf{Ora di inizio:} 17:30;
	\item[•] \textbf{Ora di fine:} 20:15;
	\item[•] \textbf{Partecipanti del gruppo:}
	\begin{enumerate}
		\item [-] Alice Gibellato;
		\item [-] Elia Rech;
		\item [-] Federico Consalvo; 
		\item [-] Federico Chiarello; 
		\item [-] Matteo Sinigaglia;
		\item [-] Luca Veronese;
		\item [-] Damiano Mason.
	\end{enumerate}
	\item[•] \textbf{Segretario:} Alice Gibellato. 
\end{itemize}

\section{Ordine del giorno}
\begin{itemize}
	\item Team Building;
	\item Creazione gruppo \glo{Telegram};
	\item Nome del gruppo;
	\item Logo del gruppo;
	\item Indirizzo e-mail del gruppo;
	\item Prossimo incontro.
\end{itemize}

\section{Resoconto}
\subsection{Team Building}
All'inizio dell'incontro ogni componente del gruppo si è presentato, spiegando quali fossero le sue esperienze nel mondo dell'informatica, le skills che ha acquisito nel corso degli anni e gli esami che deve sostenere durante la sessione invernale, in modo tale da poter gestire al meglio l'organizzazione del \glo{team}.
\subsection{Creazione gruppo \glo{Telegram}}
Federico Consalvo è stato incaricato di creare il canale \glo{Telegram} utilizzato per l'organizzazione e la comunicazione interna del gruppo.
\subsection{Nome del gruppo}
Ogni componente del \glo{team} ha proposto un'idea per il nome del gruppo. I nomi suggeriti sono: 
\begin{itemize}
	\item Sweet Code;	
	\item The Swe Club;	
	\item Sevenbuck;	
	\item QuaranTeam;	
	\item Sweppie;	
	\item AnSWErs and Solutions;	
	\item Skynet;	
	\item SWEet software;		
	\item Steve Wozniak Everywhere;	
	\item Sons of Steve Jobs.
\end{itemize} 
\subsection{Logo del gruppo}
Alice Gibellato è stata incaricata di creare un logo ufficiale per il gruppo.
\subsection{Indirizzo e-mail del gruppo}
Federico Consalvo è stato incaricato di creare un indirizzo di posta elettronica per la comunicazione esterna del gruppo con i \glo{committenti} e i \glo{proponenti} dei \glo{capitolati}.
\subsection{Prossimo incontro}
È stato concordato e fissato il prossimo meeting per il 2020-11-24. 

\section{Riepilogo delle decisioni}
\begin{RiepilogoDecisioni}
	 VI\_2020-10-27.1 &  È stato deciso che per il prossimo meeting bisogna aver stabilito in modo ufficiale il nome del gruppo tra la lista proposta durante la riunione. \\
	\hline 
	 VI\_2020-10-27.2 &  È stato deciso l'utilizzo di \glo{Telegram} come canale per la comunicazione interna del gruppo. \\
	\hline
	 VI\_2020-10-27.3 &  È stato deciso che durante la prossima riunione si discuterà dei \glo{capitolati} d'appalto. \\
\end{RiepilogoDecisioni}
\end{document}