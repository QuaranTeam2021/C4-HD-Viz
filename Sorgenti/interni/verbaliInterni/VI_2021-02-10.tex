\newcommand{\Data}{2021-02-10}
\newcommand{\TitoloDoc}{V.I. del \Data}
\newcommand{\Redattori}{\COF}
\newcommand{\Verificatori}{\CHF}
\newcommand{\Approvatore}{\REL}
\newcommand{\Distribuzione}{\Committente{} \\& \Gruppo{}}
\newcommand{\Uso}{Interno}
\newcommand{\Stato}{Approvato}
\newcommand{\DescrizioneDoc}{Riassunto della riunione del gruppo \textit{\Gruppo} tenutasi il \Data.}
\newcommand{\pathimg}{../../immagini}
\newcommand{\VersioneDoc}{1.0.0}

% info generali 
\newcommand{\NomeProgetto}{HD Viz}

% fornitore
\newcommand{\Gruppo}{QuaranTeam}
\newcommand{\Mail}{quaranteam2021@gmail.com}

% committenti
\newcommand{\Committente}{\VT{}\\& \CR{}}
\newcommand{\VT}{Prof. Vardanega Tullio}
\newcommand{\CR}{Prof. Cardin Riccardo}

% proponenti
\newcommand{\Proponente}{Zucchetti S.p.A.}
\newcommand{\PG}{Piccoli Gregorio}

% QuaranTeam member
\newcommand{\CHF}{Chiarello Federico}
\newcommand{\COF}{Consalvo Federico}
\newcommand{\GIA}{Gibellato Alice}
\newcommand{\MAD}{Mason Damiano}
\newcommand{\REL}{Rech Elia}
\newcommand{\SIM}{Sinigaglia Matteo}
\newcommand{\VEL}{Veronese Luca}

% ruoli
\newcommand{\Responsabile}{Responsabile di Progetto}
\newcommand{\Amministratore}{Amministratore di Progetto}

% documenti
\newcommand{\SdF}{Studio di Fattibilità}
\newcommand{\SdFv}[1]{\textit{Studio di Fattibilità {#1}}}
\newcommand{\PdQ}{Piano di Qualifica}
\newcommand{\PdQv}[1]{\textit{Piano di Qualifica {#1}}}
\newcommand{\PdP}{Piano di Progetto}
\newcommand{\PdPv}[1]{\textit{Piano di Progetto {#1}}}
\newcommand{\NdP}{Norme di Progetto}
\newcommand{\NdPv}[1]{\textit{Norme di Progetto {#1}}}
\newcommand{\AdR}{Analisi dei Requisiti}
\newcommand{\AdRv}[1]{\textit{Analisi dei Requisiti {#1}}}
\newcommand{\Glossario}{Glossario}
\newcommand{\Glossariov}[1]{\textit{Glossario {#1}}}
\newcommand{\MM}{Manuale Manutentore}
\newcommand{\MMv}[1]{\textit{Manuale Manutentore {#1}}}
\newcommand{\MU}{Manuale Utente}
\newcommand{\MUv}[1]{\textit{Manuale Utente {#1}}}

% comandi generali
\newcommand{\glo}[1]{#1\textsubscript{\textit{G}}}
\newcommand{\qm}[1]{``#1''}

\newcommand{\defaultfooter}[1]{
	\rowcolor{white}
	\multicolumn{#1}{|c|}{\textit{La tabella continua a pagina seguente.}}\\
    \hline
    \endfoot
    \endlastfoot
}


\newcommand{\versionMU}{2.0.0}
\newcommand{\versionMM}{2.0.0}
\newcommand{\versionSdF}{1.0.0}
\newcommand{\versionPdQ}{4.0.0}
\newcommand{\versionPdP}{4.0.0}
\newcommand{\versionNdP}{3.0.0}
\newcommand{\versionAdR}{3.0.0}
\newcommand{\versionGlossario}{4.0.0}

\documentclass{../../Utility/stdDocument}

\begin{document}
\verbaleFronte
\section*{Registro delle modifiche} 
\begin{RegistroModifiche}
	1.0.0 & 2021-02-20 & Approvazione del documento. & \REL &  Responsabile \\
	\hline
	0.1.0 & 2021-02-12 & Stesura del documento. \newline
	 Verificatore: \CHF & \COF &  Analista \\
\end{RegistroModifiche}

\verbaleIndice

\section{Informazioni generali}
\begin{itemize}
	\item[•] \textbf{Luogo:} \glo{Zoom};
	\item[•] \textbf{Data:} 2021-02-10;
	\item[•] \textbf{Ora di inizio:} 18:00;
	\item[•] \textbf{Ora di fine:} 20:30;
	\item[•] \textbf{Partecipanti del gruppo:}
	\begin{enumerate}
		\item [-] Alice Gibellato;
		\item [-] Elia Rech;
		\item [-] Federico Consalvo; 
		\item [-] Federico Chiarello; 
		\item [-] Matteo Sinigaglia;
		\item [-] Luca Veronese;
		\item [-] Damiano Mason.
	\end{enumerate}
	\item[•] \textbf{Segretario:} Federico Consalvo. 
\end{itemize}

\section{Ordine del giorno}
\begin{itemize}
	\item Problematiche rilevate;
	\item Valutazione per il miglioramento;
	\item Prossimo incontro.
\end{itemize}

\section{Resoconto}
Questo incontro si è tenuto per discutere delle problematiche che si sono incontrate durante lo svolgimento dei lavori programmati.
\subsection{Problematiche rilevate}
Riguardo l'organizzazione: 
\begin{itemize}
	\item Incontri telematici tra membri del gruppo e con il \glo{proponente}: a causa della pandemia dovuta a \glo{COVID-19} è stato impossibile svolgere incontri di persona.
	\item Problemi di connessione: durante le video chiamate su \glo{Zoom} si sono presentati problemi di comunicazione dovuti alla connessione.
\end{itemize}
Riguardo il rivestimento dei ruoli: 
\begin{itemize}
	\item Responsabile di progetto: chi ha lavorato come responsabile di progetto ha avuto delle difficoltà nel bilanciamento iniziale della suddivisione di ore e ruoli.
	\item Analista: alcuni \glo{requisiti} non risultavano molto chiari e dettagliati.
	\item Verificatori: per svolgere un'accurata verifica di ogni documento bisognava svolgere uno studio approfondito di tutta la documentazione.
\end{itemize}
Riguardo gli strumenti utilizzati: 
\begin{itemize}
	\item \glo{LaTeX}: a causa dell'inesperienza di alcuni membri del gruppo nell'utilizzo dello strumento si sono riscontrate alcune difficoltà nella creazione di tabelle.
	\item \glo{GitHub}: si sono incontrate alcune difficoltà nella gestione dei \glo{commit} e dei vari branch per alcuni membri del gruppo.
	\item \glo{StarUML}: il software non sempre risultava collaborativo nella costruzione del \glo{UML}.
\end{itemize}
\subsection{Valutazione per il miglioramento}
Per ogni problema rilevato precedentemente è stata decisa una contromisura per la sua risoluzione. 
Contromisure per le problematiche riguardo l'organizzazione: 
\begin{itemize}
	\item Incontri telematici: abbiamo utilizzato \glo{Zoom} per svolgere gli incontri di gruppo utilizzando un foglio \glo{Google Drive} condiviso per prendere nota di tutte le decisioni prese durante lo svolgimento della riunione.
	\item Incontri con il \glo{proponente}: abbiamo utilizzato la \glo{piattaforma} \glo{Skype} per svolgere gli incontri.
	Per venire incontro all’azienda, i meeting si sono tenuti anche con la presenza degli altri gruppi concorrenti per il \glo{capitolato} C4.
	\item Problemi di connessione: abbiamo utilizzato le chat di \glo{Zoom} e \glo{Telegram} quando si presentavano queste difficoltà in modo da riuscire a comunicare ugualmente.
\end{itemize}
Contromisure per le problematiche riguardo il rivestimento dei ruoli: 
\begin{itemize}
	\item Responsabile di progetto: ad ogni riunione, il responsabile di turno si occupava di controllare che la suddivisione del lavoro fosse sempre equa. Se necessario si occupava di predisporre un redistribuzione oraria per mantenere l’equità. Questo ha permesso di evitare eventuali ritardi nella consegna e sovraccarichi per alcuni membri del gruppo a differenza di altri.
	\item Analista: a causa della mancanza di dettagli e particolari alcune parti risultavano difficili da rappresentare. Abbiamo interagito con il \glo{proponente} per avere maggiori dettagli sui \glo{requisiti}. Ciò ha richiesto una maggior partecipazione come analisti ad alcuni membri del gruppo.
	\item Verificatori: si è deciso di assegnare a tutti i membri il ruolo di verificatore in modo da avere una verifica approfondita e mantenere un'equa suddivisione del lavoro.
\end{itemize}

Contromisure per le problematiche dovute dagli strumenti utilizzati: 
\begin{itemize}
	\item \glo{LaTeX}: alcuni membri del gruppo si sono resi disponibili nell'aiutare i membri meno esperti nella risoluzione delle problematiche.
	\item \glo{GitHub}: per diminuire il numero di conflitti si è deciso di utilizzare branch di lavoro separati. Quando si sono verificati conflitti molto estesi i membri più esperti hanno aiutato nella risoluzione dei problemi.
	\item StarUML: per il tracciamento dei \glo{requisiti} il \glo{software} non è sempre risultato collaborativo. Le difficoltà maggiori si sono verificate per esportare il diagramma in formato .png e dividerlo in diverse immagini. Usando strumenti per catturare l'immagine dello schermo nei documenti esse risultavano sgranate.
\end{itemize}

Anche a seguito di queste valutazioni verrà aggiornata l'analisi dei rischi.

\subsection{Prossimo incontro} 
È stato fissato il prossimo incontro in data 2021-02-19, nel quale il gruppo si aggiornerà sull'andamento dei lavori e prenderà decisioni riguardo le future consegne. 

\section{Riepilogo delle decisioni}
\begin{RiepilogoDecisioni}
	VI\_2021-02-10.1 & Valutazioni per il miglioramento. \\	\hline
	VI\_2021-02-10.2 & Deliberata la stesura della sezione del \textit{\PdP{}} relativa all'attualizzazione dei rischi. \\	\hline
	VI\_2021-02-10.3 & Convocazione del prossimo incontro in data 2021-02-19. \\	\hline
\end{RiepilogoDecisioni}
\end{document}