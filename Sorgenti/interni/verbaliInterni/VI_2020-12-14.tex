\newcommand{\Data}{2020-12-14}
\newcommand{\TitoloDoc}{V.I. del \Data}
\newcommand{\Redattori}{\GIA}
\newcommand{\Verificatori}{\CHF}
\newcommand{\Approvatore}{\MAD}
\newcommand{\Distribuzione}{\Committente{} \\& \Gruppo{}}
\newcommand{\Uso}{Interno}
\newcommand{\Stato}{Approvato}
\newcommand{\DescrizioneDoc}{Riassunto della riunione del gruppo \textit{\Gruppo} tenutasi il \Data.}
\newcommand{\pathimg}{../../immagini}
\newcommand{\VersioneDoc}{1.0.0}

% info generali 
\newcommand{\NomeProgetto}{HD Viz}

% fornitore
\newcommand{\Gruppo}{QuaranTeam}
\newcommand{\Mail}{quaranteam2021@gmail.com}

% committenti
\newcommand{\Committente}{\VT{}\\& \CR{}}
\newcommand{\VT}{Prof. Vardanega Tullio}
\newcommand{\CR}{Prof. Cardin Riccardo}

% proponenti
\newcommand{\Proponente}{Zucchetti S.p.A.}
\newcommand{\PG}{Piccoli Gregorio}

% QuaranTeam member
\newcommand{\CHF}{Chiarello Federico}
\newcommand{\COF}{Consalvo Federico}
\newcommand{\GIA}{Gibellato Alice}
\newcommand{\MAD}{Mason Damiano}
\newcommand{\REL}{Rech Elia}
\newcommand{\SIM}{Sinigaglia Matteo}
\newcommand{\VEL}{Veronese Luca}

% ruoli
\newcommand{\Responsabile}{Responsabile di Progetto}
\newcommand{\Amministratore}{Amministratore di Progetto}

% documenti
\newcommand{\SdF}{Studio di Fattibilità}
\newcommand{\SdFv}[1]{\textit{Studio di Fattibilità {#1}}}
\newcommand{\PdQ}{Piano di Qualifica}
\newcommand{\PdQv}[1]{\textit{Piano di Qualifica {#1}}}
\newcommand{\PdP}{Piano di Progetto}
\newcommand{\PdPv}[1]{\textit{Piano di Progetto {#1}}}
\newcommand{\NdP}{Norme di Progetto}
\newcommand{\NdPv}[1]{\textit{Norme di Progetto {#1}}}
\newcommand{\AdR}{Analisi dei Requisiti}
\newcommand{\AdRv}[1]{\textit{Analisi dei Requisiti {#1}}}
\newcommand{\Glossario}{Glossario}
\newcommand{\Glossariov}[1]{\textit{Glossario {#1}}}
\newcommand{\MM}{Manuale Manutentore}
\newcommand{\MMv}[1]{\textit{Manuale Manutentore {#1}}}
\newcommand{\MU}{Manuale Utente}
\newcommand{\MUv}[1]{\textit{Manuale Utente {#1}}}

% comandi generali
\newcommand{\glo}[1]{#1\textsubscript{\textit{G}}}
\newcommand{\qm}[1]{``#1''}

\newcommand{\defaultfooter}[1]{
	\rowcolor{white}
	\multicolumn{#1}{|c|}{\textit{La tabella continua a pagina seguente.}}\\
    \hline
    \endfoot
    \endlastfoot
}


\newcommand{\versionMU}{2.0.0}
\newcommand{\versionMM}{2.0.0}
\newcommand{\versionSdF}{1.0.0}
\newcommand{\versionPdQ}{4.0.0}
\newcommand{\versionPdP}{4.0.0}
\newcommand{\versionNdP}{3.0.0}
\newcommand{\versionAdR}{3.0.0}
\newcommand{\versionGlossario}{4.0.0}

\documentclass{../../Utility/stdDocument}

\begin{document}
\verbaleFronte
\section*{Registro delle modifiche} 
\begin{RegistroModifiche}
 1.0.0 & 2020-12-20 &  Approvazione del documento. & \MAD &  Responsabile \\
\hline
 0.1.1 & 2020-12-18 &  Verifica del documento. & \CHF  &  Verificatore \\
\hline
 0.1.0 & 2020-12-14 & Stesura del documento. & \GIA &  Analista \\
\end{RegistroModifiche}

\verbaleIndice

\section{Informazioni generali}
\begin{itemize}
	\item[•] \textbf{Luogo:} \glo{Zoom};
	\item[•] \textbf{Data:} 2020-12-14;
	\item[•] \textbf{Ora di inizio:} 17:30;
	\item[•] \textbf{Ora di fine:} 19:30;
	\item[•] \textbf{Partecipanti del gruppo:}
	\begin{enumerate}
		\item [-] Alice Gibellato;
		\item [-] Elia Rech; 
		\item [-] Federico Chiarello; 
		\item [-] Matteo Sinigaglia;
		\item [-] Luca Veronese;
		\item [-] Damiano Mason.
	\end{enumerate} 
	\item[•] \textbf{Assenti:}
	\begin{enumerate}
		\item [-] Federico Consalvo.
	\end{enumerate}
	\item[•] \textbf{Segretario:} Alice Gibellato.
\end{itemize}

\section{Ordine del giorno}
\begin{itemize}
	\item Revisione \textit{\NdP};
	\item Preparazione domande al \glo{proponente};
	\item Organizzazione del lavoro;
	\item Prossimo incontro.
\end{itemize}

\section{Resoconto}
\subsection{Revisione \textit{\NdP}}
Durante la settimana del 2020-12-07 sono state sviluppate le \textit{\NdP} mantenendo la suddivisone dei gruppi stabilita durante la riunione precedente. È stata svolta una veloce revisione che ha portato ad un elenco delle parti mancanti e da completare. Le \textit{\NdP} saranno completate definitivamente dopo il primo incontro con il \glo{proponente} e revisionate dopo la stesura degli altri documenti. 
\subsection{Preparazione delle domande da rivolgere al \glo{proponente}}
Dopo aver contattato gli altri gruppi è stato fissato un incontro con il \glo{proponente} previsto il 2020-12-17. A tal proposito sono state preparate delle domande riguardo il prodotto finale, la loro disponibilità in caso di necessità, il canale di comunicazione e dei chiarimenti sui dati da utilizzare nell'applicazione.
\subsection{Organizzazione del lavoro}
È stata fissata una scadenza intermedia prevista per il 3 gennaio per concludere la stesura di tutti i documenti ed avere così a disposizione una settimana prima della consegna per eventuali revisioni ed aggiustamenti. È inoltre stata fatta un'ulteriore suddivisione in gruppi per la redazione dei successivi documenti: \textit{\PdP} e \textit{\PdQ}. In merito all'\textit{\AdR} si è deciso di discuterne dopo aver chiarito i dubbi sul prodotto che si andrà a sviluppare all'incontro con il \glo{proponente}. L'\textit{\AdR} verrà creata e sviluppata progressivamente e collaborativamente da tutti i membri del gruppo. 
\subsection{Prossimo incontro}
Il prossimo incontro si terrà il 2020-12-17, assieme al \glo{proponente} dell'azienda \textit{\Proponente} e agli altri gruppi interessati al \glo{capitolato} C4. Il prossimo incontro interno verrà deciso mediante i canali di comunicazione interna successivamente all'incontro esterno.

\section{Riepilogo delle decisioni}
\begin{RiepilogoDecisioni}
	 VI\_2020-12-14.1 &  Revisione \textit{\NdP}. \\
	\hline
	 VI\_2020-12-14.2 &  Si è deciso di preparare un documento Google dove verranno raccolte le domande per l'incontro con il \glo{proponente}. \\
	\hline
	 VI\_2020-12-14.3 &  Suddivisione in gruppi per la scrittura dei documenti. \\
	\hline
	VI\_2020-12-14.4 &  Stabilita scadenza per la stesura dei documenti. \\
\end{RiepilogoDecisioni}

\end{document}